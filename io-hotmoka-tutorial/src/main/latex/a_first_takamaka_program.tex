\chapter{A first Takamaka program}\label{ch:a_first_takamaka_program}

Takamaka\index{Takamaka} is the language that can be used to write
smart contracts for Hotmoka nodes. Hotmoka
nodes and Takamaka code have exactly the same
relation as Ethereum nodes and Solidity code.

Let us start from a simple example of Takamaka code. Since we are
writing Java code, there is nothing special to learn or install
before starting writing programs in Takamaka. Just use your
preferred integrated development environment (IDE) for Java. Or even
do everything from command-line, if you prefer. Our examples below will be
shown for the Eclipse IDE, using Java 21 or later, but you can perfectly well
use the IntelliJ IDE instead.

Our goal will be to create a Java class that we will instantiate
and use in blockchain. Namely, we will learn how to create an object
of that class, that will be persisted in blockchain, and how we can later
call the \texttt{toString()} method on that instance in blockchain.

\section{Creation of the Eclipse project}\label{sec:creation_first_eclipse_project}

\begin{center}
(See the \texttt{io-hotmoka-tutorial-examples-family} project in \texttt{\hotmokaRepo{}})
\end{center}

Let us create a Maven project \texttt{io-hotmoka-tutorial-examples-family} inside Eclipse,
in the \texttt{\hotmokaTutorialDir} directory.
For that, in the Eclipse's Maven wizard
(\emph{New}$\rightarrow$\emph{Maven project}) specify the options
\emph{Create a simple project (skip archetype selection)}
and deselect the \emph{Use default Workspace directory} option,
specifying a subdirectory \texttt{io-hotmoka-tutorial-examples-family}
of the \texttt{\hotmokaTutorialDir{}} directory as \emph{Location} instead.
Hence, \emph{Location} should be something that ends
with \texttt{\ldots/\hotmokaTutorialDir{}/io-hotmoka-tutorial-examples-family}.
Do not add the project to any working set. Use \texttt{io.hotmoka}
as \emph{Group Id} and use the directory name
\texttt{io-hotmoka-tutorial-examples-family} as \emph{Artifact Id}.

\begin{commentbox}
The \emph{Group Id} can be changed as you prefer, but we will stick
to \texttt{io.hotmoka} to show the exact files that you will see in the provided code.
\end{commentbox}

By clicking \emph{Finish} in the Eclipse's Maven wizard, you should see
a new Maven project in the Eclipse's explorer.
Currently, Eclipse creates a default \texttt{pom.xml} file that uses Java 5
and has no dependencies. Replace hence
the content of the \texttt{pom.xml} file that has been created in the
\texttt{io-hotmoka-tutorial-examples-family} project with the following code:
%
\begin{codebox}\begin{xmllst}
<project xmlns="http://maven.apache.org/POM/4.0.0"
    xmlns:xsi="http://www.w3.org/2001/XMLSchema-instance"
    xsi:schemaLocation="http://maven.apache.org/POM/4.0.0 http://maven.apache.org/xsd/maven-4.0.0.xsd">

  <modelVersion>4.0.0</modelVersion>
  <groupId>io.hotmoka</groupId>
  <artifactId>io-hotmoka-tutorial-examples-family</artifactId>
  <version>|\hotmokaVersion{}|</version>

  <properties>
    <project.build.sourceEncoding>UTF-8</project.build.sourceEncoding>
    <maven.compiler.release>21</maven.compiler.release>
  </properties>

  <dependencies>
    <dependency>
      <groupId>io.hotmoka</groupId>
      <artifactId>io-takamaka-code</artifactId>
      <version>|\takamakaVersion{}|</version>
    </dependency>
  </dependencies>

  <build>
    <plugins>
      <plugin>
        <groupId>org.apache.maven.plugins</groupId>
        <artifactId>maven-compiler-plugin</artifactId>
        <version>3.11.0</version>
      </plugin>
    </plugins>
  </build>

</project>
\end{xmllst}\end{codebox}
%
It specifies to use Java 21 and provides the dependency
to \texttt{io-takamaka-code}, that is, the run-time classes of the Takamaka smart contracts.

\begin{commentbox}
We are using \texttt{\takamakaVersion{}} here, as version of the Takamaka runtime
project. You can replace that, if needed, with the latest version of the project.
\end{commentbox}

Since the \texttt{pom.xml} file has changed, Eclipse will normally show an error
in the project. To solve it,
you need to update the Maven dependencies of the project:
right-click on the project then \emph{Maven}$\rightarrow$\emph{Update Project}.
The imported dependency \texttt{io-takamaka-code},
that contains the Takamaka runtime, should be downloaded
and everything should compile without errors.
The result in Eclipse should look similar to what is
shown in Fig.~\ref{fig:family}.
%
\begin{figure}[t]
  \begin{center}
    \includegraphics[width=5cm]{pics/family}
  \end{center}
  \caption{The family Eclipse project.}
  \label{fig:family}
\end{figure}

Create a \texttt{module-info.java} file inside \texttt{src/main/java}
(right-click on the project, then \emph{Configure}$\rightarrow$\emph{Create module-info.java}$\rightarrow$\emph{Create}),
to state that this project depends on the module containing the runtime of Takamaka:
%
\begin{codebox}\begin{javalst}
module family {
  requires io.takamaka.code;
}
\end{javalst}\end{codebox}

Create a package \texttt{io.hotmoka.tutorial.examples.family}
inside \texttt{src/main/java}. Inside that package,
create a Java source \texttt{Person.java}, by copying and pasting the following code:
%
\begin{codebox}\begin{javalst}
package io.hotmoka.tutorial.examples.family;

import io.takamaka.code.lang.StringSupport;

public class Person {
  private final String name;
  private final int day;
  private final int month;
  private final int year;
  public final Person parent1;
  public final Person parent2;

  public Person(String name, int day, int month, int year, Person parent1, Person parent2) {
    this.name = name;
    this.day = day;
    this.month = month;
    this.year = year;
    this.parent1 = parent1;
    this.parent2 = parent2;
  }

  public Person(String name, int day, int month, int year) {
    this(name, day, month, year, null, null);
  }

  @Override
  public String toString() {
    return StringSupport.concat(name, " (", day, "/", month, "/", year, ")");
  }
}
\end{javalst}\end{codebox}

This is a plain old Java class and should not need any comment.
The only observation is that we concat strings
with the support class \texttt{StringSupport}\index{StringSupport},
since the standard string concatenation
of Java would end up
calling methods whose computational cost is not foreseeable in advance. In general,
a very small portion of the
Java library can be used directly in Takamaka,
and support classes\index{support class} are used to replace some common functionalities,
such as string concatenation.
