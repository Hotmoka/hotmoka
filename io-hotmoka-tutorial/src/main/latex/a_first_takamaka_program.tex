\chapter{A first Takamaka program}\label{ch:a_first_takamaka_program}

Takamaka\index{Takamaka} is the language that can be used to write
smart contracts for Hotmoka nodes. Hotmoka
nodes and Takamaka code have exactly the same
relation as Ethereum nodes and Solidity code.

Let us start from a simple example of Takamaka code. Since we are
writing Java code, there is nothing special to learn or install
before starting writing programs in Takamaka. Just use your
preferred integrated development environment (IDE) for Java. Or even
do everything from command-line, if you prefer. Our examples below will be
shown for the Eclipse IDE, using Java 21 or later, but you can perfectly well
use the IntelliJ IDE instead.

Our goal will be to create a Java class that we will instantiate
and use in blockchain. Namely, we will learn how to create an object
of that class, that will be persisted in blockchain, and how we can later
call the \texttt{toString()} method on that instance in blockchain.

\section{Creation of the Eclipse project}\label{sec:creation_first_eclipse_project}

\begin{center}
(See the \texttt{io-hotmoka-tutorial-examples-family} project in \texttt{\hotmokaRepo{}})
\end{center}

Let us create a Maven project \texttt{io-hotmoka-tutorial-examples-family} inside Eclipse,
in the \texttt{\hotmokaTutorialDir} directory.
For that, in the Eclipse's Maven wizard
(\emph{New}$\rightarrow$\emph{Maven project}) specify the options
\emph{Create a simple project (skip archetype selection)}
and deselect the \emph{Use default Workspace directory} option,
specifying a subdirectory \texttt{io-hotmoka-tutorial-examples-family}
of the \texttt{\hotmokaTutorialDir{}} directory as \emph{Location} instead.
Hence, \emph{Location} should be something that ends
with \texttt{\ldots/\hotmokaTutorialDir{}/io-hotmoka-tutorial-examples-family}.
Do not add the project to any working set. Use \texttt{io.hotmoka}
as \emph{Group Id} and use the directory name
\texttt{io-hotmoka-tutorial-examples-family} as \emph{Artifact Id}.

\begin{commentbox}
The \emph{Group Id} can be changed as you prefer, but we will stick
to \texttt{io.hotmoka} to show the exact files that you will see in the provided code.
\end{commentbox}

By clicking \emph{Finish} in the Eclipse's Maven wizard, you should see
a new Maven project in the Eclipse's explorer.
Currently, Eclipse creates a default \texttt{pom.xml} file that uses Java 5
and has no dependencies. Replace hence
the content of the \texttt{pom.xml} file that has been created in the
\texttt{io-hotmoka-tutorial-examples-family} project with the following code:
%
\begin{codebox}\begin{xmllst}
<project xmlns="http://maven.apache.org/POM/4.0.0"
    xmlns:xsi="http://www.w3.org/2001/XMLSchema-instance"
    xsi:schemaLocation="http://maven.apache.org/POM/4.0.0 http://maven.apache.org/xsd/maven-4.0.0.xsd">

  <modelVersion>4.0.0</modelVersion>
  <groupId>io.hotmoka</groupId>
  <artifactId>io-hotmoka-tutorial-examples-family</artifactId>
  <version>|\hotmokaVersion{}|</version>

  <properties>
    <project.build.sourceEncoding>UTF-8</project.build.sourceEncoding>
    <maven.compiler.release>21</maven.compiler.release>
  </properties>

  <dependencies>
    <dependency>
      <groupId>io.hotmoka</groupId>
      <artifactId>io-takamaka-code</artifactId>
      <version>|\takamakaVersion{}|</version>
    </dependency>
  </dependencies>

  <build>
    <plugins>
      <plugin>
        <groupId>org.apache.maven.plugins</groupId>
        <artifactId>maven-compiler-plugin</artifactId>
        <version>3.11.0</version>
      </plugin>
    </plugins>
  </build>

</project>
\end{xmllst}\end{codebox}
%
It specifies to use Java 21 and provides the dependency
to \texttt{io-takamaka-code}, that is, the run-time classes of the Takamaka smart contracts.

\begin{commentbox}
We are using \texttt{\takamakaVersion{}} here, as version of the Takamaka runtime
project. You can replace that, if needed, with the latest version of the project.
\end{commentbox}

Since the \texttt{pom.xml} file has changed, Eclipse will normally show an error
in the project. To solve it,
you need to update the Maven dependencies of the project:
right-click on the project then \emph{Maven}$\rightarrow$\emph{Update Project}.
The imported dependency \texttt{io-takamaka-code},
that contains the Takamaka runtime, should be downloaded
and everything should compile without errors.
The result in Eclipse should look similar to what is
shown in Fig.~\ref{fig:family}.
%
\begin{figure}[t]
  \begin{center}
    \includegraphics[width=5cm]{pics/family}
  \end{center}
  \caption{The family Eclipse project.}
  \label{fig:family}
\end{figure}

Create a \texttt{module-info.java} file inside \texttt{src/main/java}
(right-click on the project, then \emph{Configure}$\rightarrow$\emph{Create module-info.java}$\rightarrow$\emph{Create}),
to state that this project depends on the module containing the runtime of Takamaka:
%
\begin{codebox}\begin{javalst}
module family {
  requires io.takamaka.code;
}
\end{javalst}\end{codebox}

Create a package \texttt{io.hotmoka.tutorial.examples.family}
inside \texttt{src/main/java}. Inside that package,
create a Java source \texttt{Person.java}, by copying and pasting the following code:
%
\begin{codebox}\begin{javalst}
package io.hotmoka.tutorial.examples.family;

import io.takamaka.code.lang.StringSupport;

public class Person {
  private final String name;
  private final int day;
  private final int month;
  private final int year;
  public final Person parent1;
  public final Person parent2;

  public Person(String name, int day, int month, int year, Person parent1, Person parent2) {
    this.name = name;
    this.day = day;
    this.month = month;
    this.year = year;
    this.parent1 = parent1;
    this.parent2 = parent2;
  }

  public Person(String name, int day, int month, int year) {
    this(name, day, month, year, null, null);
  }

  @Override
  public String toString() {
    return StringSupport.concat(name, " (", day, "/", month, "/", year, ")");
  }
}
\end{javalst}\end{codebox}

This is a plain old Java class and should not need any comment.
The only observation is that we concat strings
with the support class \texttt{StringSupport}\index{StringSupport},
since the standard string concatenation
of Java would end up
calling methods whose computational cost is not foreseeable in advance. In general,
a very small portion of the
Java library can be used directly in Takamaka,
and support classes\index{support class} are used to replace some common functionalities,
such as string concatenation.

Package the project into a jar and install it in the local Maven repository,
by running the Maven command \texttt{mvn install} inside
the directory of the project (that is, inside the subdirectory
\texttt{\hotmokaTutorialDir{}/io-hotmoka-tutorial-examples-family}).
This should generate a file named
\texttt{io-hotmoka-tutorial-examples-family-\hotmokaVersion{}.jar},
inside the \texttt{target} directory. Only the compiled
class files will be relevant: Hotmoka nodes will ignore source files, manifest
and any resources in the jar; the same compiled
\texttt{module-info.class} is irrelevant for Hotmoka.
All such files can be removed from the jar, to reduce the gas cost of their
installation in the store of a node, but we do not care about this optimization here.
The result should look as in Fig.~\ref{fig:family_jar}.
%
\begin{figure}[t]
  \begin{center}
    \includegraphics[width=5cm]{pics/family_jar}
  \end{center}
  \caption{The family Eclipse project, exported in jar.}
  \label{fig:family_jar}
\end{figure}

\section{Installation of the Jar in a Hotmoka node}\label{sec:jar_installation}

\begin{center}
(See the \texttt{io-hotmoka-tutorial-examples-runs} project in \texttt{\hotmokaRepo{}})
\end{center}

We have generated the jar containing our code and we want to send it now to a Hotmoka node,
where it will be installed. This means that it will become available to programmers
who want to use its classes, directly or as dependencies of their programs.
In order to install a jar in the Hotmoka node that we have used in the previous chapter,
we can use the moka command-line tool, specifying which account will pay for the
installation of the jar. The cost of the installation depends on the size of the
jar and on the number of its dependencies. The moka tool uses a heuristics to
foresee this cost, that can be overridden if needed.

Move inside the \texttt{\hotmokaTutorialDir{}} directory, if you are not
there already, so that
moka will find your saved key pair there, and run the \texttt{moka jars install} command:
%
\begin{shellcommandbox}\begin{ttlst}
moka jars install ef8e65ed3b4ab9c5c57ced0ab83c56ee423d39279d6bca6cf8d748be175ddefa#0 io-hotmoka-tutorial-examples-family/target/io-hotmoka-tutorial-examples-family-1.11.4.jar --password-of-payer --uri=ws://157.90.103.251:8001
\end{ttlst}\end{shellcommandbox}
\index{jar!install}
\begin{shellbox}\begin{ttlst}
Enter value for --password-of-payer (the password of the key pair of the payer account): chocolate
The jar has been installed at 762aaa30978229e145eca5731ab4badcd99a329a8273acbe25dc84dcbe651909.

Gas consumption:
 * total: 443162
   * for CPU: 15348
   * for RAM: 5174
   * for storage: 422640
   * for penalty: 0
 * price per unit: 1 pana
 * total price: 443162 panas
\end{ttlst}\end{shellbox}

%
As you can see above, the jar has been installed at a reference
\texttt{\familyAddressShort{}}, that can be used
later to refer to that jar. This has costed some gas, paid by our account.
You can verify that the balance of the account has been decreased, through the
\texttt{moka objects show} command.

The state of the Hotmoka nodes of the network is now as in Fig.~\ref{fig:jar_installed}.
As shown there, a dependency has been automatically created\index{jar!dependency}
from \texttt{io-hotmoka-tutorial-examples-family-\hotmokaVersion{}.jar} to
\texttt{io-takamaka-code-\takamakaVersion{}.jar}. This is because all Takamaka code
will use the run-time classes of the Takamaka language,
hence the \texttt{moka jars install} command adds them, by default.
That command allows to explicitly specify other dependencies, for more complicated uses.
Note that a dependency must already be installed in the node
before it can be used as dependency of other jars.
%
\begin{figure}[t]
  \begin{center}
    \begin{tikzpicture}[scale=0.45,>=Stealth]
      \draw[rounded corners] (0,-13.5) rectangle (21,19.3) [fill=statebackgroundcolor];

      \draw[rounded corners] (0.5,16) rectangle +(20,2.8) [fill=jarbackgroundcolor];
      \draw (1,18) node[right] {jar: classes of the Takamaka runtime, such as};
      \draw (1,17) node[right] {\texttt{Contract}, \texttt{Storage}, \texttt{@View}, \texttt{@FromContract}\ldots};
      \draw (-2,18.2) node[left] {Takamaka code};
      \draw (-2,17.1) node[left] {at \texttt{\takamakaCodeShort{}}};
      \draw[->,thick] (-2,17.5) -- +(2.5,0);

      \draw[->,thick] (6,13.3) -- +(0,2.7);
      \draw (6,14.8) node[left] {classpath};
      \draw (-2,12.8) node[left] {manifest};
      \draw (-2,11.7) node[left] {at \texttt{\manifestShort{}}};
      \draw[->,thick] (-2,12.3) -- +(2.5,0);

      \draw[rounded corners] (0.5,10.8) rectangle +(11,2.8) [fill=objectbackgroundcolor];
      \draw (1,12.7) node[right] {chainId: \chainId{}};
      \draw (1,11.7) node[right] {signature: ed25519};

      \draw[<-,thick] (4,8.1) -- +(0,2.7);
      \draw (4,9.5) node[right] {gamete};
      \draw[->,thick] (2,10.8) .. controls (2,5) .. node[very near start,sloped,above] {\small gas station} (3,3.8);
      \draw[->,thick] (1,10.8) .. controls (1,2) .. node[near end,sloped,below] {\small validators} (1.5,-1.2);

      \draw[rounded corners] (3,5.3) rectangle +(8,2.8) [fill=objectbackgroundcolor];
      \draw (3.5,6.7) node[right] {balance: 9999\ldots};
      \draw (-2,7.2) node[left] {gamete};
      \draw (-2,6.1) node[left] {at \texttt{\gameteShort{}}};
      \draw[->,thick] (-2,6.7) -- +(5,0);
      \draw[->,thick] (10,8.1) .. controls (12,11) .. node[near end,sloped,below] {classpath} (12,16.1);

      \draw[rounded corners] (2,1) rectangle +(8,2.8) [fill=objectbackgroundcolor];
      \draw (2.5,2.5) node[right] {gasPrice: 1};
      \draw (-2,3) node[left] {gas station};
      \draw (-2,2) node[left] {at \texttt{\gasStationShort{}}};
      \draw[->,thick] (-2,2.5) -- +(4,0);
      \draw[->,thick] (9,3.8) .. controls (13,6) .. node[near end,sloped,below] {classpath} (13,16.1);

      \draw[rounded corners] (0.5,-4) rectangle +(10.5,2.8) [fill=objectbackgroundcolor];
      \draw (1,-2) node[right] {number of validators: 0};
      \draw (1,-3.1) node[right] {current supply: 1209\ldots};
      \draw (-2,-2) node[left] {validators};
      \draw (-2,-3) node[left] {at \texttt{\validatorsShort{}}};
      \draw[->,thick] (-2,-2.5) -- +(2.5,0);
      \draw[->,thick] (9.5,-1.2) .. controls (14,5) .. node[near end,sloped,below] {classpath} (14,16.1);

      \draw[rounded corners] (3,-8.5) rectangle +(11,2.8) [fill=objectbackgroundcolor];
      \draw (3.5,-7) node[right] {balance: \accountOneBalance};
      \draw (-2,-6.6) node[left] {account1};
      \draw (-2,-7.7) node[left] {at \texttt{\accountOneShort{}}};
      \draw[->,thick] (-2,-7) -- +(5,0);
      \draw[->,thick] (10,-5.7) .. controls (15,0) .. node[near end,sloped,below] {classpath} (15,16.1);

      \draw[rounded corners] (0.5,-13) rectangle +(20,2.8) [fill=jarbackgroundcolor];
      \draw (1,-11.1) node[right] {file \texttt{io-hotmoka-\ldots-family-\hotmokaVersion{}.jar},};
      \draw (1,-12.1) node[right] {that defines some classes, such as \texttt{Person}};
      \draw (-2,-11.1) node[left] {jar file};
      \draw (-2,-12.1) node[left] {at \texttt{\familyAddressShort{}}};
      \draw[->,thick] (-2,-11.6) -- +(2.5,0);
      \draw[->,thick] (15,-10.2) .. controls (16,-5) .. node[near end,sloped,below] {dependency} (16,16.1);
    \end{tikzpicture}
  \end{center}
  \caption{The state of the network nodes after the installation of our jar.}
  \label{fig:jar_installed}
\end{figure}

What we have done above is probably enough for most users, but sometimes you need
to perform the same operation in code, for instance in order to implement a software
application that connects to a Hotmoka node and runs some transactions.
Therefore, we describe below how you can write a Java program that installs the
same jar in the Hotmoka node, without using the \texttt{moka jars install} command.
A similar translation in code can be performed for all examples in this tutorial,
but we will report it only for a few of them.

Let us hence create another Eclipse Maven project
\texttt{io-hotmoka-tutorial-examples-runs}, inside
the directory \texttt{\hotmokaTutorialDir{}},
exactly as we did in the previous section for the \texttt{family} project.
Specify Java 21 (or later) in its build configuration.
Use \texttt{io.hotmoka} as \emph{Group Id} and
\texttt{io-hotmoka-tutorial-examples-runs} as \emph{Artifact Id}.
This is specified in the following \texttt{pom.xml}, that you should copy inside
the \texttt{io-hotmoka-tutorial-examples-runs} project, replacing that generated by Eclipse:
%
\begin{codebox}\begin{xmllst}
<project xmlns="http://maven.apache.org/POM/4.0.0"
  xmlns:xsi="http://www.w3.org/2001/XMLSchema-instance"
  xsi:schemaLocation="http://maven.apache.org/POM/4.0.0 http://maven.apache.org/xsd/maven-4.0.0.xsd">

  <modelVersion>4.0.0</modelVersion>
  <groupId>io.hotmoka</groupId>
  <artifactId>io-hotmoka-tutorial-examples-runs</artifactId>
  <version>|\hotmokaVersion{}|</version>

  <properties>
    <project.build.sourceEncoding>UTF-8</project.build.sourceEncoding>
    <maven.compiler.release>21</maven.compiler.release>
  </properties>

  <build>
    <plugins>
      <plugin>
        <groupId>org.apache.maven.plugins</groupId>
        <artifactId>maven-compiler-plugin</artifactId>
        <version>3.11.0</version>
      </plugin>
    </plugins>
  </build>

  <dependencies>
    <dependency>
      <groupId>io.hotmoka</groupId>
      <artifactId>io-hotmoka-node-remote</artifactId>
      <version>|\hotmokaVersion{}|</version>
    </dependency>
    <dependency>
      <groupId>io.hotmoka</groupId>
      <artifactId>io-hotmoka-helpers</artifactId>
      <version>|\hotmokaVersion{}|</version>
    </dependency>
    <dependency>
      <groupId>io.hotmoka</groupId>
      <artifactId>io-hotmoka-node-tendermint</artifactId>
      <version>|\hotmokaVersion{}|</version>
    </dependency>
    <dependency>
      <groupId>io.hotmoka</groupId>
      <artifactId>io-hotmoka-node-disk</artifactId>
      <version>|\hotmokaVersion{}|</version>
    </dependency>
    <dependency>
      <groupId>io.hotmoka</groupId>
      <artifactId>io-hotmoka-node-service</artifactId>
      <version>|\hotmokaVersion{}|</version>
    </dependency>
    <dependency>
      <groupId>io.hotmoka</groupId>
      <artifactId>io-hotmoka-constants</artifactId>
      <version>|\hotmokaVersion{}|</version>
    </dependency>
    <dependency>
      <groupId>io.hotmoka</groupId>
      <artifactId>io-takamaka-code-constants</artifactId>
      <version>|\takamakaVersion{}|</version>
    </dependency>
  </dependencies>

</project>
\end{xmllst}\end{codebox}
%
This \texttt{pom.xml} specifies a few dependencies. We do not need all of them now,
but we will need them along the next sections, hence let us insert them all already.
These dependencies get automatically downloaded from the Maven repository.

Since we have just modified the file \texttt{pom.xml}, Eclipse could show an error
for the project \texttt{io-hotmoka-tutorial-examples-runs}. To fix it,
you need to update the Maven dependencies of the project:
right-click on the \texttt{io-hotmoka-tutorial-examples-runs} project
and then select the menu \emph{Maven}$\rightarrow$\emph{Update Project\ldots}

Create a \texttt{module-info.java} inside \texttt{src/main/java}, containing:
%
\begin{codebox}\begin{javalst}
module io.hotmoka.tutorial.examples.runs {
  requires io.hotmoka.helpers;
  requires io.hotmoka.node.remote;
  requires io.hotmoka.node.disk;
  requires io.hotmoka.node.tendermint;
  requires io.hotmoka.node.service;
  requires io.hotmoka.constants;
  requires io.takamaka.code.constants;
}
\end{javalst}\end{codebox}
%
Again, we do not need all such dependencies already, but we will need them later.

Create a package \texttt{io.hotmoka.tutorial.examples.runs} inside
\texttt{src/main/java} and add the following class \texttt{Family.java} inside it:
%
\begin{codebox}\begin{javalst}
package io.hotmoka.tutorial.examples.runs;

import static java.math.BigInteger.ONE;

import java.math.BigInteger;
import java.net.URI;
import java.nio.file.Files;
import java.nio.file.Path;
import java.nio.file.Paths;
import java.security.KeyPair;

import io.hotmoka.constants.Constants;
import io.hotmoka.crypto.api.Signer;
import io.hotmoka.helpers.GasHelpers;
import io.hotmoka.helpers.SignatureHelpers;
import io.hotmoka.node.Accounts;
import io.hotmoka.node.MethodSignatures;
import io.hotmoka.node.StorageValues;
import io.hotmoka.node.TransactionRequests;
import io.hotmoka.node.api.Node;
import io.hotmoka.node.api.requests.SignedTransactionRequest;
import io.hotmoka.node.api.transactions.TransactionReference;
import io.hotmoka.node.api.values.StorageReference;
import io.hotmoka.node.remote.RemoteNodes;

public class Family {

  public static void main(String[] args) throws Exception {

    // the path of the user jar to install
    var familyPath = Paths.get(System.getProperty("user.home")
      + "/.m2/repository/io/hotmoka/io-hotmoka-tutorial-examples-family/"
      + Constants.HOTMOKA_VERSION
      + "/io-hotmoka-tutorial-examples-family-" + Constants.HOTMOKA_VERSION + ".jar");

    var dir = Paths.get(args[1]);
    var payer = StorageValues.reference(args[2]);
    var password = args[3];

    try (var node = RemoteNodes.of(new URI(args[0]), 150000)) {
      // we get a reference to where io-takamaka-code-X.Y.Z.jar has been stored
      TransactionReference takamakaCode = node.getTakamakaCode();

      // we get the signing algorithm to use for requests
      var signature = node.getConfig().getSignatureForRequests();

      KeyPair keys = loadKeys(node, dir, payer, password);

      // we create a signer that signs with the private key of our account
      Signer<SignedTransactionRequest<?>> signer = signature.getSigner
      (keys.getPrivate(), SignedTransactionRequest::toByteArrayWithoutSignature);

      // we get the nonce of our account: we use the account itself as caller and
      // an arbitrary nonce (ZERO in the code) since we are running
      // a @View method of the account
      BigInteger nonce = node
      .runInstanceMethodCallTransaction(TransactionRequests.instanceViewMethodCall
      (payer, // payer
      BigInteger.valueOf(100_000), // gas limit
      takamakaCode, // class path for the execution of the transaction
      MethodSignatures.NONCE, // method
      payer)).get() // receiver of the method call
      .asBigInteger(__ -> new ClassCastException());

      // we get the chain identifier of the network
      String chainId = node.getConfig().getChainId();
      var gasHelper = GasHelpers.of(node);

      // we install the family jar in the node: our account will pay
      TransactionReference family = node
      .addJarStoreTransaction(TransactionRequests.jarStore
      (signer, // an object that signs with the payer's private key
      payer, // payer
      nonce, // payer's nonce: relevant since this is not a call to a @View method!
      chainId, // chain identifier: relevant since this is not a call to a @View method!
      BigInteger.valueOf(1_000_000), // gas limit: enough for this small jar
      gasHelper.getSafeGasPrice(), // gas price: at least the current gas price of the network
      takamakaCode, // class path for the execution of the transaction
      Files.readAllBytes(familyPath), // bytes of the jar to install
      takamakaCode)); // dependencies of the jar that is being installed

      // we increase our copy of the nonce, ready for further
      // transactions having the account as payer
      nonce = nonce.add(ONE);

      System.out.println("jar installed at " + family);
    }
  }

  private static KeyPair loadKeys(Node node, Path dir, StorageReference account, String password) throws Exception {
    return Accounts.of(account, dir)
    .keys(password, SignatureHelpers.of(node).signatureAlgorithmFor(account));
  }
}
\end{javalst}\end{codebox}
%
As you can see, the above \texttt{main} method requires four values to be provided from the caller:
the server to contact,
the directory \texttt{dir} where the key pair of the payer account can be found,
the storage reference of that payer account, and the password of the key pair.

The code above creates an instance of a \texttt{RemoteNode}\index{RemoteNode},
that represents a Hotmoka
node installed in a remote host. By specifying the URI of the host, the \texttt{RemoteNode} object
exposes all methods of a Hotmoka node. It is an \texttt{AutoCloseable} object,
hence it is placed inside
a try-with-resource statement that guarantees its release at the end of the \texttt{try} block.
By using that remote node, our code collects
some information about the node: the reference to the \texttt{io-takamaka-code} jar already
installed inside it (\texttt{takamakaCode}) and the \texttt{signature} algorithm used
by the node,
that it uses to construct a \texttt{signer} object that signs with the private key of our account,
loaded from disk.

The \texttt{loadKeys} method accesses the \texttt{.pem} file that we have previously created
with moka, that should be inside a directory \texttt{dir}.

Like every Hotmoka node, the observable state of the remote node can only evolve through
\emph{transactions}, that modify its state in an atomic way.
Namely, the code above performs two transactions:
%
\begin{enumerate}
\item A call to the \texttt{nonce()} method of our account\index{account!nonce}:
  this is a progressive counter of the number
  of transactions already performed with our account. It starts from zero, but our account has been
  already used for other transactions (through the moka tool). Hence we better ask the
  node about it. As we will see later, this transaction calls a \texttt{@View} method. All calls
  to \texttt{@View} methods have the nice feature of being \emph{for free}:
  nobody will pay for them.
  Because of that, we do not need to sign this transaction, or to provide a correct nonce,
  or specify a gas price. The limitation of such calls is that their transactions are not
  checked by consensus, hence we have to trust the node we ask. Moreover, they can only
  read, never write the data in the store of the node.
\item The addition of our jar in the node. This time the transaction has a cost and our
  account is specified as payer. The signer of our account signs the transaction.
  The nonce of our account and the chain identifier of the network are relevant, as well as the
  gas price, that must at least match that of the network.
  The code uses the \texttt{addJarStoreTransaction()} method, that executes a new transaction
  on the node, whose goal is to install a jar inside it. The jar is provided as a sequence of bytes
  (\texttt{Files.readAllBytes(familyPath)}, where
  \texttt{familyPath} looks inside the local Maven repository
  of the machine, where the jar to install in the node should already be present).
  The request passed to \texttt{addJarStoreTransaction()}
  specifies that the transaction can cost up
  to $300000$ units of gas, that can be bought at the price returned by the
  \texttt{gasHelper} object. The request
  specifies that its class path is \texttt{node.getTakamakaCode()}: this is the reference to the
  \texttt{io-takamaka-code} jar already installed in the node.
  Finally, the request specifies that
  \texttt{io-hotmoka-tutorial-examples-family-\hotmokaVersion{}.jar} has only
  a single dependency: \texttt{io-takamaka-code}. This means that when, later, we will refer to
  \texttt{io-hotmoka-tutorial-examples-family-\hotmokaVersion{}.jar}
  in a class path, this class path will indirectly include its dependency
  \texttt{io-takamaka-code} as well (see Fig.~\ref{fig:jar_installed}).
\end{enumerate}

\begin{commentbox}
As in Ethereum, transactions in Hotmoka are paid
in terms of gas\index{gas} consumed for their execution.
Calls to \texttt{@View}\index{@View} methods do not actually modify the state of the node
and are executed locally, on the
node that receives the request of the transaction. Hence, they can be considered
as run \emph{for free}. Instead, we have used an actual gas price for the last
transaction that installs the jar in blockchain. This could be computed with
a sequence of calls to \texttt{@View} methods (get the manifest, then the gas station
inside the manifest, then the gas price inside the gas station). In order to
simplify the code, we have used the \texttt{GasHelper} class, that does exactly that for us.
\end{commentbox}

You can run the program with Maven, specifying the server to contact,
the directory where the key pairs can be found, the payer account and its password:
%
\begin{shellcommandbox}\begin{ttlst}
mvn compile exec:java -Dexec.mainClass="io.hotmoka.tutorial.examples.runs.Family" -Dexec.args="ws://157.90.103.251:8001 hotmoka_tutorial 95712b56da4e96f5c0fe523260b6c572904ac17c7a304121e455368bd29ae657#0 chocolate"
\end{ttlst}\end{shellcommandbox}

\begin{shellbox}\begin{ttlst}
jar installed at 81933e89de8767540798c8e4f4d7fe9ac9c42c99d63d5c4c4e7645a253335fac
\end{ttlst}\end{shellbox}

%
The exact address where the jar gets installed
will change in your machine. In any case, note that this reference
to the jar is functionally equivalent to that
obtained before with the \texttt{moka jars install} command: they point to equivalent jars.

\section{Creation of an object of our class}\label{sec:creation_object}

\begin{center}
  (See the \texttt{io-hotmoka-tutorial-examples-family-storage} and the
  \texttt{io-hotmoka-tutorial-examples-runs} projects in \texttt{\hotmokaRepo{}})
\end{center}

The jar of our program is in the store of the node now: the \texttt{moka jars install} command
has installed it at \texttt{\familyAddressShort{}} and our code at
\texttt{\codeFamilyAddressShort{}}.
We can use either of them, interchangeably, as class path for the execution of a transaction that
tries to run the constructor of our class \texttt{Person} and add a brand
new \texttt{Person} object into the store of the node. We can perform this through the moka tool:
%
\begin{shellcommandbox}\begin{ttlst}
moka objects create ef8e65ed3b4ab9c5c57ced0ab83c56ee423d39279d6bca6cf8d748be175ddefa#0 io.hotmoka.tutorial.examples.family.Person Einstein 14 4 1879 null null --classpath=762aaa30978229e145eca5731ab4badcd99a329a8273acbe25dc84dcbe651909 --password-of-payer --uri=ws://157.90.103.251:8001
\end{ttlst}\end{shellcommandbox}

\begin{shellbox}\begin{ttlst}
Enter value for --password-of-payer (the password of the key pair of the payer account): chocolate
Do you really want to call constructor
  public ...Person(java.lang.String,int,int,int,...Person,...Person)
  spending up to 1000000 gas units at the price of 1 pana per unit (that is, up to 1000000 panas) [Y/N] Y
Timeout: I waited for 250000ms but the remote service at ws://157.90.103.251:8001 didn't answer! [java.util.concurrent.TimeoutException: null]
\end{ttlst}\end{shellbox}

%
The \texttt{moka objects create}\index{object!creation}
command requires to specify who pays for the object creation
(our account \texttt{\accountOneShort{}}),
then the fully-qualified name of the class that we want to instantiate
(\texttt{io.hotmoka.tutorial.examples.Person})
followed by the actual arguments passed to its constructor.
The classpath refers to the jar that we have installed previously.
The \texttt{moka objects create} command
asks for the password of the payer account and
checks if we really want to proceed (and pay). But then it ends up in failure
(\texttt{SerializationException}). Note that all offered gas has been spent.
This is a sort of \emph{penalty}\index{gas!penalty} for running a transaction that fails.
The rationale is that this penalty should discourage
potential denial-of-service attacks, when a huge number of failing transactions are thrown at a
node. At least, that attack will cost a lot. Moreover, note that the transaction, although
failed, does exist. Indeed, the nonce of the caller has been increased, as you can check
with \texttt{moka objects show} on our account.

But we still have not understood why the transaction failed. The reason is in the exception
message: \texttt{An object of class ...Person cannot be serialized into a storage value since it does not implement io.takamaka.code.lang.Storage}\index{Storage}.
Takamaka requires
that all objects stored in a node extend the \texttt{io.takamaka.code.lang.Storage} class.
That superclass
provides all the machinery needed in order to keep track of updates to such objects and persist
them in the store of the node, automatically.

\begin{commentbox}
Do not get confused here. Takamaka does \emph{not} require all objects to extend
the class \texttt{io.takamaka.code.lang.Storage}.
You can use objects that do not extend that superclass in your
Takamaka code, both instances of your classes and instances of library classes
from the \texttt{java.*} hierarchy, for instance.
What Takamaka does require, instead, is that objects
\emph{that must be kept in the store of a node} do extend
\texttt{io.takamaka.code.lang.Storage}. This
must be the case, for instance, for objects created by the constructor invoked through the
\texttt{moka objects create} command.
\end{commentbox}

Let us then modify the \texttt{io.hotmoka.tutorial.examples.Person.java}
source code, inside the \texttt{io-hotmoka-tutorial-examples-family} project:
%
\begin{codebox}\begin{javalst}
package io.hotmoka.tutorial.examples.family;

import io.takamaka.code.lang.Storage;

public class Person extends Storage {
  ... unchanged code ...
}
\end{javalst}\end{codebox}
%
\begin{commentbox}
  Extending \texttt{io.takamaka.code.lang.Storage}
  is all a programmer needs to do in order to let instances
  of a class be stored in the store of a node. There is no explicit method to call to keep track
  of updates to such objects and persist them in the store of the node:
  Hotmoka nodes will automatically deal with them.
\end{commentbox}
%
\begin{commentbox}
  We can use the \texttt{io.takamaka.code.lang.Storage} class
  and we can run the resulting compiled code
  since that class is inside \texttt{io-takamaka-code}, that has been included in the
  class path as a dependency of \texttt{io-hotmoka-tutorial-examples-family-\hotmokaVersion{}.jar}
  (Fig.~\ref{fig:jar_installed}).
\end{commentbox}

Since class \texttt{Person} has been modified, regenerate
\texttt{io-hotmoka-tutorial-examples-family-\hotmokaVersion{}.jar},
by running \texttt{mvn install} again
inside the \texttt{io-hotmoka-tutorial-examples-family} project.
Install this new version in blockchain:
%
\begin{shellcommandbox}\begin{ttlst}
cd io-takamaka-code-examples-family
mvn clean install
cd ..
moka jars install 95712b56da4e96f5c0fe523260b6c572904ac17c7a304121e455368bd29ae657#0 io-hotmoka-tutorial-examples-family/target/io-hotmoka-tutorial-examples-family-1.11.4.jar --password-of-payer --uri=ws://157.90.103.251:8001
\end{ttlst}\end{shellcommandbox}

\begin{shellbox}\begin{ttlst}
Enter value for --password-of-payer (the password of the key pair of the payer account): chocolate
The jar has been installed at f9dbebb0469aafd089eedefabd676af09728073ed2cf29cea0a30635a54b7728.

Gas consumption:
 * total: 464477
   * for CPU: 15318
   * for RAM: 5159
   * for storage: 444000
   * for penalty: 0
 * price per unit: 1 pana
 * total price: 464477 panas
\end{ttlst}\end{shellbox}

%
then run the \texttt{moka objects create} command again. This time, the execution should
complete without exception:
%
\begin{shellcommandbox}\begin{ttlst}
moka objects create 95712b56da4e96f5c0fe523260b6c572904ac17c7a304121e455368bd29ae657#0 io.hotmoka.tutorial.examples.family.Person Einstein 14 4 1879 null null --classpath=e56db9d16ab9ed7a2fa5bcca49586505a34934479dcf4d91d3b1bf2bd6af1e33 --password-of-payer --uri=ws://157.90.103.251:8001
\end{ttlst}\end{shellcommandbox}

\begin{shellbox}\begin{ttlst}
Enter value for --password-of-payer (the password of the key pair of the payer account): chocolate
Do you really want to call constructor
  public ...Person(java.lang.String,int,int,int,...Person,...Person)
  spending up to 1000000 gas units at the price of 1 pana per unit (that is, up to 1000000 panas) [Y/N] Y
A new object 443a7bb67488d416d9992f09906ba3c55c845bfe0d9642e813b8ef12f7d1a160#0 has been created.

Gas consumption:
 * total: 58700
   * for CPU: 14945
   * for RAM: 5035
   * for storage: 38720
   * for penalty: 0
 * price per unit: 1 pana
 * total price: 58700 panas
\end{ttlst}\end{shellbox}

%
The new object has been allocated at a storage reference \texttt{\personObjectShort{}}
that can be used to refer to it, also in the future.
You can verify that it is actually there and that its fields are correctly initialized,
by using the \texttt{moka objects show} command:
%
\begin{shellcommandbox}\begin{ttlst}
moka objects show 443a7bb67488d416d9992f09906ba3c55c845bfe0d9642e813b8ef12f7d1a160#0 --uri ws://panarea.hotmoka.io:8001
\end{ttlst}\end{shellcommandbox}

\begin{shellbox}\begin{ttlst}
class io.hotmoka.tutorial.examples.family.Person (from jar installed at ec59a4b330b8037152647e2c44c42577bad8f107f8b1c72fcff5234bfeebce2b)
  day:int = 14
  month:int = 4
  name:java.lang.String = "Einstein"
  parent1:io.hotmoka.tutorial.examples.family.Person = null
  parent2:io.hotmoka.tutorial.examples.family.Person = null
  year:int = 1879
\end{ttlst}\end{shellbox}

%
\begin{commentbox}
  Compared with Solidity, where contracts and accounts are just untyped \emph{addresses},
  objects (and hence accounts) are strongly-typed in Takamaka.
  This means that they are tagged with their run-time type (see the output
  of \texttt{moka objects show} above), in a boxed representation,
  so that it is possible to check that they are used correctly, \ie{}, in accordance
  with the declared type of variables, or to check their run-time type with checked casts
  and the \texttt{instanceof} operator. Moreover, Takamaka has information to check
  that such objects have been created by using the same
  jar that stays in the class path later, every time an object gets used
  (see the information \texttt{from jar installed at} in the output of
  \texttt{moka objects show} above).
\end{commentbox}

We can perform the same object creation in code, instead of using the
\texttt{moka objects create} command.
Namely, the following code builds on the previous example and installs a jar by adding
a further transaction that calls the constructor of \texttt{Person}:
%
\begin{codebox}\begin{javalst}
package io.hotmoka.tutorial.examples.runs;

import static io.hotmoka.helpers.Coin.panarea;
import static io.hotmoka.node.StorageTypes.INT;
import static java.math.BigInteger.ONE;

import java.math.BigInteger;
import java.net.URI;
import java.nio.file.Files;
import java.nio.file.Path;
import java.nio.file.Paths;
import java.security.KeyPair;

import io.hotmoka.constants.Constants;
import io.hotmoka.crypto.api.Signer;
import io.hotmoka.helpers.GasHelpers;
import io.hotmoka.helpers.SignatureHelpers;
import io.hotmoka.node.Accounts;
import io.hotmoka.node.ConstructorSignatures;
import io.hotmoka.node.MethodSignatures;
import io.hotmoka.node.StorageTypes;
import io.hotmoka.node.StorageValues;
import io.hotmoka.node.TransactionRequests;
import io.hotmoka.node.api.Node;
import io.hotmoka.node.api.requests.SignedTransactionRequest;
import io.hotmoka.node.api.transactions.TransactionReference;
import io.hotmoka.node.api.types.ClassType;
import io.hotmoka.node.api.values.StorageReference;
import io.hotmoka.node.remote.RemoteNodes;

public class FamilyStorage {

  private final static ClassType PERSON = StorageTypes.classNamed("io.hotmoka.tutorial.examples.family.Person");

  public static void main(String[] args) throws Exception {
    // the path of the user jar to install
    var familyPath = Paths.get(System.getProperty("user.home")
    + "/.m2/repository/io/hotmoka/io-hotmoka-tutorial-examples-family/"
    + Constants.HOTMOKA_VERSION
    + "/io-hotmoka-tutorial-examples-family-" + Constants.HOTMOKA_VERSION + ".jar");

    var dir = Paths.get(args[1]);
    var payer = StorageValues.reference(args[2]);
    var password = args[3];

    try (var node = RemoteNodes.of(new URI(args[0]), 150000)) {
      // we get a reference to where io-takamaka-code-X.Y.Z.jar has been stored
      TransactionReference takamakaCode = node.getTakamakaCode();

      // we get the signing algorithm to use for requests
      var signature = node.getConfig().getSignatureForRequests();

      KeyPair keys = loadKeys(node, dir, payer, password);

      // we create a signer that signs with the private key of our account
      Signer<SignedTransactionRequest<?>> signer = signature.getSigner
      (keys.getPrivate(), SignedTransactionRequest::toByteArrayWithoutSignature);

      // we get the nonce of our account: we use the account itself as caller and
      // an arbitrary nonce (ZERO in the code) since we are running
      // a @View method of the account
      BigInteger nonce = node
      .runInstanceMethodCallTransaction(TransactionRequests.instanceViewMethodCall
      (payer, // payer
      BigInteger.valueOf(100_000), // gas limit
      takamakaCode, // class path for the execution of the transaction
      MethodSignatures.NONCE, // method
      payer)).get() // receiver of the method call
      .asBigInteger(__ -> new ClassCastException());

      // we get the chain identifier of the network
      String chainId = node.getConfig().getChainId();

      var gasHelper = GasHelpers.of(node);

      // we install the family jar in the node: our account will pay
      TransactionReference family = node
      .addJarStoreTransaction(TransactionRequests.jarStore
      (signer, // an object that signs with the payer's private key
      payer, // payer
      nonce, // payer's nonce: relevant since this is not a call to a @View method!
      chainId, // chain identifier: relevant since this is not a call to a @View method!
      BigInteger.valueOf(1_000_000), // gas limit: enough for this small jar
      gasHelper.getSafeGasPrice(), // gas price: at least the current gas price of the network
      takamakaCode, // class path for the execution of the transaction
      Files.readAllBytes(familyPath), // bytes of the jar to install
      takamakaCode)); // dependencies of the jar that is being installed
      
      // we increase our copy of the nonce, ready for further
      // transactions having the account as payer
      nonce = nonce.add(ONE);

      // call the constructor of Person and store in einstein the new object in blockchain
      StorageReference einstein = node.addConstructorCallTransaction
      (TransactionRequests.constructorCall
      (signer, // an object that signs with the payer's private key
      payer, // payer
      nonce, // payer's nonce: relevant since this is not a call to a @View method!
      chainId, // chain identifier: relevant since this is not a call to a @View method!
      BigInteger.valueOf(300_000), // gas limit: enough for a small object
      panarea(gasHelper.getSafeGasPrice()), // gas price, in panareas
      family, // class path for the execution of the transaction
      
      // constructor Person(String,int,int,int)
      ConstructorSignatures.of(PERSON, StorageTypes.STRING, INT, INT, INT),

      // actual arguments
      StorageValues.stringOf("Einstein"), StorageValues.intOf(14),
      StorageValues.intOf(4), StorageValues.intOf(1879)
      ));

      System.out.println("new object allocated at " + einstein);

      // we increase our copy of the nonce, ready for further
      // transactions having the account as payer
      nonce = nonce.add(ONE);
    }
  }

  private static KeyPair loadKeys(Node node, Path dir, StorageReference account, String password) throws Exception {
    return Accounts.of(account, dir)
    .keys(password, SignatureHelpers.of(node).signatureAlgorithmFor(account));
  }
}
\end{javalst}\end{codebox}
%
The new transaction is triggered by the
\texttt{addConstructorCallTransaction()} call,
that expands the node with a new transaction that calls
a constructor. We use our account as payer for the transaction, hence we sign
the request with its private key.
The class path includes \texttt{io-hotmoka-tutorial-examples-family-\hotmokaVersion{}.jar}
and its dependency \texttt{io-takamaka-code}.
The signature of the constructor specifies that we are referring to the second
constructor of \texttt{Person}, the one that assumes \texttt{null} as parents.
The actual parameters
are provided; they must be instances of the \texttt{io.hotmoka.node.api.values.StorageValue}
interface.
We provide $300000$ units of gas, which should be enough for a constructor
that just initializes a few fields.
We are ready to pay \texttt{panarea(gasHelper.getSafeGasPrice())}
units of coin for each unit of gas.
This price could have been specified simply as \texttt{gasHelper.getSafeGasPrice()},
but we used the static method \texttt{io.hotmoka.helpers.Coin.panarea()}
to generate a \texttt{BigInteger} corresponding to the smallest coin unit of Hotmoka nodes,
a \emph{panarea}.
Namely, Hotmoka uses the units of coin shown in Fig.~\ref{fig:units}.
%
\begin{figure}[t]
  \begin{center}
    \begin{tabular}{|c||c|c|c|}
      \hline
        value (in panas) & exponent & name & short name\\\hline\hline
        $1$ & $1$ & panarea & pana\\\hline
        $1000$ & $10^3$ & alicudi & ali\\\hline
        $1000000$ & $10^6$ & filicudi & fili\\\hline
        $1000000000$ & $10^9$ & stromboli & strom\\\hline
        $1000000000000$ & $10^{12}$ & vulcano & vul\\\hline
        $1000000000000000$ & $10^{15}$ & salina & sali\\\hline
        $1000000000000000000$ & $10^{18}$ & lipari & lipa\\\hline
        $1000000000000000000000$ & $10^{21}$ & moka & moka\\\hline
    \end{tabular}
  \end{center}
  \caption{The coin units of Hotmoka.}\label{fig:units}
\end{figure}
%
with corresponding static methods in \texttt{io.hotmoka.helpers.Coin}.

You can run \texttt{FamilyStorage} now, and see that, this time,
the object of class \texttt{Person} gets created in blockchain:
%
\begin{shellcommandbox}\begin{ttlst}
mvn compile exec:java -Dexec.mainClass="io.hotmoka.tutorial.examples.runs.Family" -Dexec.args="ws://157.90.103.251:8001 hotmoka_tutorial ef8e65ed3b4ab9c5c57ced0ab83c56ee423d39279d6bca6cf8d748be175ddefa#0 chocolate"
\end{ttlst}\end{shellcommandbox}

\begin{shellbox}\begin{ttlst}
new object allocated at c2dec457df97bf3f4b30585bfdaa3be0eaab2b460034e43056a5065ea01b1ea7#0
\end{ttlst}\end{shellbox}

%
The exact address where the \texttt{Person} object gets allocated will change at each run.
