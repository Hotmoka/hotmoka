\chapter{Hotmoka nodes}\label{ch:hotmoka_nodes}

A Hotmoka node is a device that implements an interface for running Java code
remotely. It can be any kind of device, such as a device of an IoT network,
but also a node of a blockchain. We have already used instances of Hotmoka nodes,
namely, instances of \texttt{RemoteNode}. But there are other examples of nodes, that we
will describe in this chapter.

The interface \texttt{io.hotmoka.node.api.Node}\index{Node@{\texttt{Node}}}
is shown in the topmost part of Fig.~\ref{fig:node_hierarchy}.
That interface can be split in five parts:
%
\begin{enumerate}
\item A \emph{get} part, that includes methods for querying the
	state of the node and for accessing the objects contained in its store.
\item An \emph{add} part, that expands the store of the node with the result of a transaction.
\item A \emph{run} part, that runs transactions that execute
	\texttt{@View}\index{View@{\texttt{View}}} methods and hence do not
   	expand the store of the node.
\item A \emph{post} part, that expands the store of the node with the result of a transaction,
	without waiting for its result; instead, a future is returned.
\item A \emph{contextual} part, that allows users to subscribe listeners of events generated during
	the execution of the transactions, or listeners called when the node gets closed, or
	to close the node itself.
\end{enumerate}

\begin{figure}[th]
  \begin{center}
    \includegraphics[width=0.55\textwidth]{pics/nodes}
  \end{center}
  \caption{The hierarchy of Hotmoka nodes.}
  \label{fig:node_hierarchy}
\end{figure}

If a node belongs to a blockchain, then all nodes of the blockchain have the same vision
of the state, so that it is equivalent to call a method on a node or on any other node of the
network. The only methods that are out of consensus, since they deal with information specific
to each node, are \texttt{getInfo()}\index{getInfo()@{\texttt{getInfo()}}}
and the four contextual methods
\texttt{subscribeToEvents()}\index{subscribeToEvents()@{\texttt{subscribeToEvents()}}},
\texttt{addOnCloseHandler()}\index{addOnCloseHandler()@{\texttt{addOnCloseHandler()}}},
\texttt{removeOnCloseHandler()}\index{removeOnCloseHandler()@{\texttt{removeOnCloseHandler()}}}
and \texttt{close()}\index{close()@{\texttt{close()}}}.

Looking at Fig.~\ref{fig:node_hierarchy}, it is possible to see that
the \texttt{Node} interface has many implementations, that we describe below.

\begin{description}
\item[Local implementations.]\index{node!local}
These are actual nodes that run on the machine
where they have been started. For instance, they can be a node
of a larger blockchain network. Among them,
\texttt{MokamintNode}\index{MokamintNode@{\texttt{MokamintNode}}}
implements a node of a Mokamint\index{Mokamint} blockchain
and will be presented in Sec.~\ref{subsec:mokamint_nodes};
\texttt{TendermintNode}\index{TendermintNode@{\texttt{TendermintNode}}}
implements a node of a Tendermint\index{Tendermint} blockchain
and will be presented in Sec.~\ref{subsec:tendermint_nodes};
\texttt{DiskNode}\index{DiskNode@{\texttt{DiskNode}}}
implements a single-node blockchain in disk memory: this
is useful for debugging, testing and learning, since it allows
one to inspect the content of blocks, transactions and store;
it will be presented in Sec.~\ref{subsec:disk_nodes}.
Local nodes can be instantiated through the static
factory methods of their supplier classes
\texttt{MokamintNodes}\index{MokamintNodes@{\texttt{MokamintNodes}}},
\texttt{TendermintNodes}\index{TendermintNodes@{\texttt{TendermintNodes}}} and
\texttt{DiskNodes}\index{DiskNodes@{\texttt{DiskNodes}}}.
Those methods requires to specify
parameters that are specific to the given node
of the network that is being started
and can be different from node to node
(\texttt{MokamintNodeConfig}\index{MokamintNodeConfig@{\texttt{MokamintNodeConfig}}} and similar).
Some implementations have to ability to \emph{resume}.
This means that they recover the state at the end of a previous execution, reconstruct the
consensus parameters from that state and resume the execution from there, downloading
and verifying blocks already processed by the network.
%
\item[Decorators.]\index{node!decorator}
The \texttt{Node}\index{Node@{\texttt{Node}}} interface is implemented by some decorators as well.
Typically, these decorators run some transactions on the decorated node,
to simplify some tasks, such as the initialization of the node, the installation of jars into the node
or the creation of accounts in the node. These decorators are views of the decorated node, in the sense
that any method of the \texttt{Node} interface, invoked on the decorator, is forwarded
to the decorated node, with the exception of the contextual methods that are executed locally
on the specific node where they are invoked.
We will discuss them in Sec.~\ref{sec:node_decorators}.
%
\item[Adaptors.]\index{node!adaptor}
Very often, one wants to \emph{publish} a node online,
so that he (and other programmers who need its service) can use it concurrently.
This should be possible for all implementations of the
\texttt{Node}\index{Node@{\texttt{Node}}} interface,
such as \texttt{DiskNode}\index{DiskNode@{\texttt{DiskNode}}},
\texttt{MokamintNode}\index{MokamintNode@{\texttt{MokamintNode}}},
\texttt{TendermintNode}\index{TendermintNode@{\texttt{TendermintNode}}}
and all present and future implementations.
In other words, one would like to publish \emph{any}
Hotmoka node as a service, accessible through the internet. This will be the subject
of Sec.~\ref{sec:hotmoka_services}.
Conversely, once a Hotmoka node has been published at some URI, say
\url{ws://my.company.com}, it will be accessible through websockets. This complexity
might make it complex, for a programmer, to use the published node.
In that case, we can create an instance of the node that operates as
a proxy to the network service, helping programmers integrate
their software to the service in a seamless way. This \emph{remote} node still implements
the \texttt{Node} interface, but simply forwards all its calls to the remote service
(with the exception of the contextual methods, that are executed locally on
the remote node itself). By programming against
the same \texttt{Node} interface, it becomes easy for a programmer
to swap a local node with a remote node, or vice versa. This mechanism is described
in Sec.~\ref{sec:remote_nodes},
where the adaptor interface \texttt{RemoteNode} in Fig.~\ref{fig:node_hierarchy} is presented.
\end{description}

\section{Tendermint nodes}\label{sec:tendermint_nodes}

Tendermint~\cite{Tendermint}\index{Tendermint}, now Ignite, is a
Byzantine-fault tolerant engine for building blockchains, that
replicates a finite-state machine on a network of nodes across the world.
The finite-state machine is often referred to as a \emph{Tendermint app}.
The nice feature of Tendermint is that it takes care of all
issues related to networking and consensus, leaving to the
developer only the task to develop the Tendermint app.

\begin{figure}[th]
  \begin{center}
    \includegraphics[width=0.8\textwidth]{pics/hotmoka_tendermint}
  \end{center}
  \caption{The architecture of the Hotmoka node based on Tendermint.}
  \label{fig:hotmoka_tendermint}
\end{figure}

There is a Hotmoka node that implements such a Tendermint app,
for programming in Takamaka over Tendermint. We have already used that node
in the previous chapter, since that installed at
\texttt{\serverTendermint{}} is a node of that type.
Fig.~\cite{fig:hotmoka_tendermint}
shows the architecture of a Tendermint Hotmoka node.
It consists of a few components.
The Hotmoka component is the Tendermint app that
implements the transactions on the state, that is the installation
of jars and the execution of code written in the Takamaka subset of Java. This part is the same in every
implementation of a Hotmoka node, not only for this one based on Tendermint.
The database that contains the state is implemented by
using the Xodus transactional database by IntelliJ.
What is specific here, however, is that transactions are put inside a blockchain
implemented by Tendermint. The communication occurs, internally, through the two TCP ports
26657 and 26658, that are the standard choice of Tendermint for communicating with an app.
Clients can contact the Hotmoka node
through any port, typically but not exclusively~8001 or~8002,
as a service that implements the interface \texttt{Node} in Fig.~\ref{fig:node_hierarchy}.
The node can live alone but is normally integrated with other Hotmoka nodes based on Tendermint, so that
they execute and verify the same transactions, reaching the same state at the end. This happens through
the TCP port 26656, that allows Tendermint instances to \emph{gossip}:
they exchange transactions and information on peers and finally reach consensus.
Each node can be configured to use a different port to communicate with clients,
which is useful if, for instance, ports 8001 or 8002 (or both)
are already used by some other service.
Port 26656 must be the same for all nodes in the network, since they must communicate on
a standard port.

We can use \texttt{\serverTendermint{}} to play with
accounts and Takamaka contracts. However, we might want to
install our own node, part of the same blockchain network of \texttt{\serverTendermint{}}
or part of a brand new blockchain. In the former case, our own node will execute
the same transactions of \texttt{\serverTendermint{}}, so that we can be sure that they are
executed according to the rules. In the latter case, we can have our own blockchain that
executes our transactions only, instead of using a shared blockchain such as that
at \texttt{\serverTendermint{}}.

This section shows how you can start your own Hotmoka Tendermint node,
consisting of a single validator node, hence part of its own blockchain.
The process is not difficult but is a bit tedious,
because it requires one to install Tendermint and to create
its configuration files. Sec.~\ref{sec:tendermint_docker}
provides a simpler alternative for reaching the same goal, by using the Docker tool.

\begin{commentbox}
We strongly suggest you to use Docker to install Hotmoka nodes, instead of the instructions
in this section, hence please
follow the instructions in Sec.~\ref{sec:tendermint_docker}.
The current section only exists in order to understand what happens inside the Docker container.
If you are not a developer, or if you are not interested in the topic, please skip this section and
jump directly to the next chapter.
\end{commentbox}

In order to use a Tendermint Hotmoka node, the Tendermint executable must be
installed in our machine, or our experiments will fail. The Hotmoka node
works with Tendermint version \tendermintVersion{}, that can be downloaded in executable
form from \url{https://github.com/tendermint/tendermint/releases/tag/v\tendermintVersion}.
Be sure that you download the executable for the architecture of your computer
and install it at a place that is
part of the command-line path of your computer. You can then verify that Tendermint is
correctly installed:
%
\begin{shellcommandbox}\begin{ttlst}
tendermint version
\end{ttlst}\end{shellcommandbox}

\begin{shellbox}\begin{ttlst}
0.34.15
\end{ttlst}\end{shellbox}


Before starting a local node of a Hotmoka blockchain based on Tendermint, you
need to create the Tendermint configuration file. For instance, in order
to run a single validator node with no non-validator nodes, you can create
its configuration files as follows:
%
\begin{shellcommandbox}\begin{ttlst}
tendermint testnet --v 1 --n 0
\end{ttlst}\end{shellcommandbox}

\begin{shellbox}\begin{ttlst}
I[2026-01-21|19:56:15.870] Generated private validator                  module=main keyFile=mytestnet/node0/config/priv_validator_key.json stateFile=mytestnet/node0/data/priv_validator_state.json
I[2026-01-21|19:56:15.870] Generated node key                           module=main path=mytestnet/node0/config/node_key.json
I[2026-01-21|19:56:15.870] Generated genesis file                       module=main path=mytestnet/node0/config/genesis.json
Successfully initialized 1 node directories
\end{ttlst}\end{shellbox}

%
This has created a directory \texttt{mytestnet/node0}
for a single Tendermint node, that includes the configuration
of the node and its private and public validator keys.

Once this is done, you can create a key pair for the \emph{gamete}\index{gamete}
of the node that you are going to start. This is an account that holds all initial
crypto coins, if any. You perform this with moka:
%
\index{moka!keys create@{\texttt{keys create}}}
\begin{shellcommandbox}\begin{ttlst}
moka keys create --name=gamete.pem --password
\end{ttlst}\end{shellcommandbox}

\begin{shellbox}\begin{ttlst}
Enter value for --password (the password that will be needed later to use the key pair): mypassword
The new key pair has been written into "gamete.pem":
* public key: B3Hv7hNMBHmfQSarxrMNqCRTj6DahKicJU6sCGPXEy2g (ed25519, base58)
* public key: lSsoWbqm+vbte4Z63UljMkCuI3CzK2Q6YFR1Y0OxISk= (ed25519, base64)
* Tendermint-like address: 16EF002D06F6477AFEDF7F0FD4F7D9CCA498BA99
\end{ttlst}\end{shellbox}


You can start now a Hotmoka node based on Tendermint,
that uses the Tendermint configuration
directory that you have just created, and with a gamete controlled
by the \texttt{gamete.pem} key pair,
by using the \texttt{moka nodes tendermint init} command. You need
to specify the jar of the runtime of Takamaka, that will
be stored inside the node as \texttt{takamakaCode}: we use the
local Maven's cache for that but you can alternatively download the
\texttt{io.takamaka-code-\takamakaVersion{}.jar} file from Maven and refer to
it in the following command line:
%
\index{moka!nodes tendermint init@{\texttt{nodes tendermint init}}}
\begin{shellcommandbox}\begin{ttlst}
moka nodes tendermint init ~/.m2/repository/io/hotmoka/io-takamaka-code/1.8.0/io-takamaka-code-1.8.0.jar --public-key-of-gamete=AAjrhYeadydfe1jrifuqRvfL5k642H98QoqCxXK1oAJp --tendermint-config=mytestnet/node0
\end{ttlst}\end{shellcommandbox}

\begin{shellbox}\begin{ttlst}
The following service has been published:
 * ws://localhost:8001: the API of this Hotmoka node

The validators are the following accounts:
 * 8974f612d9433cca36331355f7bcb65aedcb4dd32815ef8a41a4d9c5a828cdec#0 with public key GVqwAKLKxGuP96D8RfREy4BQqj8y7cymTZSy3Qy27yHx (ed25519, base58)

The owner of the key pair of the gamete can bind it now to its address with:
  moka keys bind file_containing_the_key_pair_of_the_gamete --password --url url_of_this_Hotmoka_node
or with:
  moka keys bind file_containing_the_key_pair_of_the_gamete --password --reference ee0fe2458fff5881c00975508808d5a4ec3894483935a8d7fd5d0c14cd7ef8ea#0

Press the enter key to stop this process and close this node: \end{ttlst}\end{shellbox}


This command has done a lot! It has created an instance
of \texttt{TendermintNode}; it has stored the
\texttt{io-takamaka-code-\takamakaVersion{}.jar} file
inside it; it has created
a Java object, called manifest, that contains other objects, including
an externally-owned account gamete, whose public key is
that provided with \texttt{-{}-public-key-of-gamete};
it has initialized the balance of the gamete to
the a default initial supply. Finally, this command
has published an internet service at the URI \url{ws://localhost:8001},
reachable through websocket connections, that exports the API of the node.

\begin{commentbox}
By default, \texttt{moka nodes tendermint init} publishes the service at port 8001.
This can be changed with its \texttt{-{}-port} option. Moreover, it uses a default
initial supply for the gamete that can be changed with the \texttt{-{}-initial-supply} option.
Finally, it uses a Hotmoka chain identifier identical to that of the underlying Tendermint
network, specified inside the Tendermint configuration files created by \texttt{tendermint testnet}.
If you want to override it, you can either
edit such configuration files or use the \texttt{-{}-chain-id} option (in the latter case,
the chain identifiers of Hotmoka and Tendermint may be different, which is perfectly fine).
\end{commentbox}

In order to use the gamete, you should bind its key to its actual storage
reference in the node, on your local machine. Open another shell,
move inside the directory holding the keys of the gamete and run:
%
\index{moka!nodes keys bind@{\texttt{nodes keys bind}}}
\begin{shellcommandbox}\begin{ttlst}
moka keys bind gamete.pem --password
\end{ttlst}\end{shellcommandbox}

\begin{shellbox}\begin{ttlst}
Enter value for --password (the password of the key pair): mypassword
The key pair of f14806f1597faef1622467f5dc05c64e7ef845bbe907bf519d837e1de449fd69#0 has been saved as "f14806f1597faef1622467f5dc05c64e7ef845bbe907bf519d837e1de449fd69#0.pem".\end{ttlst}\end{shellbox}

%
This operation has created a pem file whose name is that of the storage reference of the gamete.
With this file, it is possible to run transactions on behalf of the gamete.
%
\begin{commentbox}
You do not need the \texttt{-{}-uri} option above, since \url{ws://localhost:8001} is the default
URI for moka.
\end{commentbox}

Your computer exports a Hotmoka node now, running on Tendermint. You can verify this with
\texttt{moka nodes manifest show}.
Moreover, if your computer is reachable at some address \texttt{my.machine}
and if its 8001 port is open to the outside world,
then anybody can contact
your node at \url{ws://my.machine:8001}, query your node and run transactions on it.
However, what has been created is a Tendermint node where all initial coins are inside
the gamete. By using the gamete, \emph{you} can fill the node with objects
and accounts now, and in general run all transactions you want.
However, other users, who do not know the keys of the gamete,
will not be able to run any non-\texttt{@View} transaction on your node.
If you want to open a faucet\index{faucet}, so that other users can gain droplets of coins
(see examples in Sec.~\ref{sec:creation_account}),
you must add the \texttt{-{}-open-unsigned-faucet} option to the \texttt{moka nodes tendermint init}
command above. If you do that, you can then go into another shell (since the previous one is busy with the
execution of the node), in a directory holding the key pair file of the gamete, and type:
%
\index{moka!nodes faucet@{\texttt{nodes faucet}}}
\begin{shellcommandbox}\begin{ttlst}
moka nodes faucet 5000000 --password
\end{ttlst}\end{shellcommandbox}

\begin{shellbox}\begin{ttlst}
Enter value for --password (the password of the key pair): mypassword
The threshold of the faucet has been set.\end{ttlst}\end{shellbox}

%
This set the maximal amount of coins that
the faucet is willing to give away at each request (its \emph{flow}). You can re-run the
\texttt{moka nodes faucet}
command many times, in order to change the flow of the faucet, or close it completely.
Needless to say, only the owner of the keys of the gamete can run the \texttt{moka nodes faucet} command,
which is why the key pair file of the gamete must be in the directory where you run it.

After opening a faucet with a sufficient flow, anybody can
re-run, for instance, the examples of Ch.~\ref{ch:getting_started_with_hotmoka} by replacing
\texttt{\serverMokamint{}} with \url{ws://my.machine:8001}: your computer will serve
the requests and run the transactions.

If you turn off your Hotmoka node based on Tendermint, its state remains saved inside the
\texttt{chain} directory: the \texttt{chain/tendermint} subdirectory is where Tendermint stores the blocks
of the chain; while \texttt{chain/hotmoka} contains the Xodus database,
consisting of the storage objects created in blockchain.
Try for instance to stop the Tendermint node that we initialized before
(press enter in the window where it was running).
You can subsequently resume that node from its latest state, by typing:
%
\index{moka!nodes tendermint resume@{\texttt{nodes tendermint resume}}}
\begin{shellcommandbox}\begin{ttlst}
moka nodes tendermint resume
\end{ttlst}\end{shellcommandbox}

\begin{shellbox}\begin{ttlst}
The following service has been published:
 * ws://localhost:8001: the API of this Hotmoka node

Press the enter key to stop this process and close this node: \end{ttlst}\end{shellbox}


There is a log file that can be useful to check the state of our Hotmoka-Tendermint node.
Namely, \texttt{tendermint.log} contains the log of Tendermint itself. It can be interesting
to inspect which blocks are committed and when:
%
\begin{shellbox}\begin{ttlst}
I[2025-06-11|10:13:24.143] Version info, module=main
  tendermint_version=@tendermint_version block=11 p2p=8
I[2025-06-11|10:13:24.169] Started node module=main
  nodeInfo="{ProtocolVersion:{P2P:8 Block:11 App:0}
I[2025-06-11|10:13:25.234] executed block module=state
  height=630 num_valid_txs=2 num_invalid_txs=0
I[2025-06-11|10:13:25.408] committed state module=state
  height=630 num_txs=0
  app_hash=A30F89457141AB7E94F71456871396FD9D30CA8E9F66998C6E3E3079D40849F
\end{ttlst}\end{shellbox}
%
In this log, the block height increases and the application hash changes,
reflecting the fact that the state has been modified.