
% Save this as tutorial.tex for the lwarp package tutorial.

\documentclass[12pt,oneside,openright]{book}

\usepackage{iftex}

% --- LOAD FONT SELECTION AND ENCODING BEFORE LOADING LWARP ---

\ifPDFTeX
\usepackage{lmodern}            % pdflatex or dvi latex
\usepackage[T1]{fontenc}
\usepackage[utf8]{inputenc}
\else
\usepackage{fontspec}           % XeLaTeX or LuaLaTeX
\fi

% --- LWARP IS LOADED NEXT ---
\usepackage[
%   HomeHTMLFilename=index,     % Filename of the homepage.
%   HTMLFilename={node-},       % Filename prefix of other pages.
%   IndexLanguage=english,      % Language for xindy index, glossary.
%   latexmk,                    % Use latexmk to compile.
%   OSWindows,                  % Force Windows. (Usually automatic.)
%   mathjax,                    % Use MathJax to display math.
]{lwarp}
% \boolfalse{FileSectionNames}  % If false, numbers the files.

% --- LOAD PDFLATEX MATH FONTS HERE ---

% --- OTHER PACKAGES ARE LOADED AFTER LWARP ---
\usepackage[
    left=0.5in,
    right=0.5in,
    top=0.75in,
    bottom=1.0in,
    paperwidth=8.25in,
    paperheight=11in,
    bindingoffset=0.375in
]{geometry}

\usepackage[pdftex]{graphicx}

\usepackage{fancyhdr}

\pagestyle{fancy}

\fancyhf{}%
%\renewcommand{\headrulewidth}{1pt}
\fancyhead[L]{\includegraphics[width=1.5cm]{pics/hotmoka-logo}}
\fancyhead[R]{Chapter~\thechapter}
\fancyfoot[L]{}
\fancyfoot[C]{\thepage}
\fancyfoot[R]{}

\usepackage{makeidx} \makeindex
\usepackage{xcolor}             % (Demonstration purposes only.)
\usepackage{hyperref,cleveref}  % LOAD THESE LAST!
\usepackage{url}
\DeclareUrlCommand\email{\urlstyle{tt}}
\DeclareUrlCommand\website{\urlstyle{same}}
\DeclareUrlCommand\directory{\urlstyle{same}}
\usepackage{mdframed}

% --- DOCUMENT PARAMETERS ---

\newcommand{\hotmokaVersion}{1.11.4}
\newcommand{\takamakaVersion}{1.8.0}
\newcommand{\tendermintVersion}{0.34.15}
\newcommand{\faustoEmail}{\email{fausto.spoto@hotmoka.io}}
\newcommand{\hotmokaRepo}{git@github.com:Hotmoka/hotmoka.git}
\newcommand{\hotmokaTutorialDir}{hotmoka\_tutorial}
\newcommand{\gametePublicKeyBaseFiftyeight}{GvH6hJjFNPYdDsnCvTjz83oWX2TeU5djjDSoMUBr7nKR}


% --- LATEX AND HTML CUSTOMIZATION ---
\title{Programming Hotmoka\\{\small A tutorial on Hotmoka and smart contracts in Takamaka}}
\author{Fausto Spoto (\faustoemail{})}
\date{\today\\{\small (updated to Hotmoka \hotmokaversion{} and Takamaka \takamakaversion{})} \\
  \vspace*{40ex}
  {\small\begin{minipage}{\textwidth*4/5}
    \includegraphics[width=8ex]{pics/CC_license} This document is licensed under a Creative Commons Attribution 4.0 International License.
    Its latest version is available for free in various formats at \texttt{\hotmokarepo/releases}.
  \end{minipage}}
}
\setcounter{tocdepth}{2}        % Include subsections in the \TOC.
\setcounter{secnumdepth}{2}     % Number down to subsections.
\setcounter{FileDepth}{1}       % Split \HTML\ files at sections
\booltrue{CombineHigherDepths}  % Combine parts/chapters/sections
\setcounter{SideTOCDepth}{1}    % Include subsections in the side\TOC
\HTMLTitle{Programming Hotmoka} % Overrides \title for the web page.
\HTMLAuthor{Fausto Spoto}       % Sets the HTML meta author tag.
\HTMLLanguage{en-US}            % Sets the HTML meta language.
\HTMLDescription{A tutorial on Hotmoka and smart contracts in Takamaka} % Sets the HTML meta description.
\HTMLFirstPageTop{\begin{center}\includegraphics[width=10cm]{pics/hotmoka-logo}\end{center}}
\HTMLPageTop{\begin{center}\includegraphics[width=7cm]{pics/hotmoka-logo}\end{center}}
\HTMLPageBottom{Copyright 2022 by Fausto Spoto (\faustoemail{})}
\CSSFilename{lwarp_sagebrush.css}

\definecolor{commentboxbackgroundcolor}{rgb}{0.85,0.85,0.85}
\newmdenv[linecolor=black,leftmargin=2em,rightmargin=2em,innertopmargin=1em,innerbottommargin=1em,backgroundcolor={commentboxbackgroundcolor}]{commentbox}

\newcommand{\graybox}[1]{
\fcolorbox{black}{lightgray}{
\begin{minipage}{0.9\textwidth}
  {#1}
\end{minipage}}}

\newcommand{\ie}{\emph{i.e.}\@}

\begin{document}

\maketitle                      % Or titlepage/titlingpage environment.

% An article abstract would go here.

\tableofcontents                % MUST BE BEFORE THE FIRST SECTION BREAK!
%\listoffigures

\chapter{Introduction}\label{ch:introduction}

\section{Origin of Hotmoka, Takamaka and Mokamint}

Almost two decades ago, Bitcoin~\cite{Nakamoto08}
swept the computer industry
as a revolution, providing, for the first time, a reliable technology
for building trust over an inherently untrusted computing
infrastructure, such as a distributed network of computers.
Trust immediately translated into money and Bitcoin became
an investment target, exactly at the moment of one of the worst
economical turmoil of recent times. \emph{Centralized} banks,
fighting against the crisis, looked like dinosaurs in comparison
to the \emph{decentralized} nature of Bitcoin.

Nevertheless, the novelty of Bitcoin was mainly related to its
\emph{consensus} mechanism based on a \emph{proof of work}, while the
programmability of Bitcoin transactions was limited due
to the use of a non-Turing-equivalent scripting
bytecode~\cite{Antonopoulos23}.

The next step was hence the use of a Turing-equivalent
programming language (up to \emph{gas limits}) over an abstract
store of key/value pairs, that can be
efficiently kept in a Merkle-Patricia trie.
That was Ethereum~\cite{AntonopoulosWPMP25}, whose
Solidity programming language allows one
to code any form of \emph{smart contract}, that is, code
that becomes an agreement between parties, thanks to
the underlying consensus enforced by the blockchain.

Solidity looks familiar to most programmers. Conditionals, loops and
structures are there since more than half a century. Programmers
assumed that they \emph{knew} Solidity. However, the intricacies of
its semantics made learning Solidity harder than expected.
Finding good Solidity programmers is still difficult and
they are consequently expensive. It is, instead, way too easy
to write buggy code in Solidity, that \emph{seems} to work perfectly,
up to \emph{that} day when things go wrong, very
wrong~\cite{AtzeiBC17}.

It is ungenerous to blame Solidity for all recent attacks to smart contracts
in blockchain. That mainly happened because of the same success of Solidity,
that made it the natural target of the attacks. Moreover, once the
Pandora's box of Turing equivalence has been opened, you cannot expect anymore to
keep the devils at bay, that is, to be able to
decide and understand, exactly, what your code will do at run time.
And this holds for every programming language, past, present or future.

I must confess that my first encounter with Solidity
was a source of frustration. Why was I expected to learn another programming
language? and another development environment? and another testing framework?
Why was I expected to write code without a support library that provides
proved solutions to frequent problems?
What was so special with Solidity after all? Things became even more difficult when
I tried to understand the semantics of the language. After twenty-five years of studying
and teaching programming languages, compilation, semantics and code analysis
(or, possibly, just because of that) I still cannot explain exactly why there
are structures and contracts instead of a single composition mechanism in Solidity;
nor what is indeed the meaning of \texttt{memory} and \texttt{storage} and why
it is not the compiler that takes care of such gritty details; nor why
externally owned accounts are not just a special kind of contracts;
nor why Solidity needs such low-level (and uncontrollable)
call instructions, that make Java's (horrible) reflection, in comparison, look like
a monument to clarity;
nor why types are weak in Solidity, so that contracts are held in
\texttt{address}
variables, whose actual type is unknown and cannot be easily
enforced at run time~\cite{CrafaPZ19}, with all consequent
programming monsters, such as unchecked casts. It seems that the evolution
of programming languages has brought us back to C's \texttt{void*} type.

Hence, when I first met people from Ailia SA in fall 2018, I was not surprised
to realize that they were looking for a new way of programming smart contracts
over the new blockchain that they were developing. I must thank them and our useful
discussions, that pushed me to dive in blockchain technology and
study many programming languages for smart contracts. The result
is Takamaka, a Java framework for writing smart contracts.
This means that it allows programmers to use a subset of Java for writing code
that can be installed and run in blockchain. Programmers will not have
to deal with the storage of objects in blockchain: this is completely
transparent to them. This makes Takamaka completely different from other
attempts at using Java for writing smart contracts, where programmers
must use explicit method calls to persist data to blockchain.

Writing smart contracts in Java entails that programmers
do not have to learn yet another programming language.
Moreover, they can use a well-understood and stable development
platform, together with all its modern tools. Programmers can use
features from the latest versions of Java, including lambda
expressions.
There are, of course, limitations to the kind of code that can
be run inside a blockchain. The most important limitation is
that programmers can only call a portion of the huge Java library,
whose behavior is deterministic, whose cost is predictable and whose methods are guaranteed
to terminate.

The runtime of the Takamaka programming language
is included in the Hotmoka project, a software layer for the
execution of smart contracts.
The more scientific aspects of Hotmoka and Takamaka have been published
in the last years~\cite{BeniniGMS21,CrosaraOST21,OlivieriST21,Spoto19,Spoto20}.
Hotmoka is only the application layer of a blockchain. The networking and
consensus layers of a blockchain are not part of Hotmoka. Instead, these
are either provided by Mokamint, a generic engine for building blockchains
based on proof of space~\cite{Spoto25}, or by the Tendermint generic engine
for building byzantine fault-tolerant networks~\cite{Kwon14}.
In the former case, Hotmoka becomes a blockchain, completely decentralized,
based on proof of space: the more disk space a miner allocates for mining, the
more blocks it will create. In the latter case, Hotmoka becomes a less
decentralized proof of
stake blockchain, where a dynamic set of validators decides the creation
of the next blocks.

\section{Intended audience}

This book is for software developers who want to use Hotmoka nodes and program smart contracts in Takamaka.
It goes deep into the inner working of Hotmoka. For instance, it shows how transactions can be
triggered in code, not just with the moka client of Hotmoka. Less experienced readers, or
developers not interested in writing code that interacts with Hotmoka nodes, can just skip these
parts and concentrate on the use of the moka command-line client only. Non-technical users might
just be happy with the use of Mokito, the mobile and web clients of Hotmoka, whose
functionalities are limited of course.

\section{Contributing to Hotmoka}

Hotmoka is a complex project, that requires many and different skills. After years of development,
it is ready for the general public. This does not mean that it is bug-free, nor perfect:
we expect our users to find all sort of bugs and to suggest improvements. Hence, feel
free to write to us at \faustoEmail{},
with bugs and improvement requests.
If you are a developer, consider the possibility of helping us with the development
of the project. In particular, the whole ecosystem of applications running
over Hotmoka is missing at the moment (that is, applications, typically web-based, that
use Hotmoka as their backend storage). Hotmoka is open-source and non-proprietary,
licensed under the terms of the Apache~2.0 License. Therefore, feel free to clone and fork the code.

\section{The example projects of this book}

The experiments that we will perform in this book will
require one to create Java projects. We suggest that you create and
experiment with these projects yourself.
However, if you have no time and want to jump immediately to the result,
or if you want to compare your work
with the expected result, we provide you with the completed examples of this book in
a module of the Hotmoka distribution repository, that you can clone.
Each section of this book will report
the project of the repository where you can find the related code.
You can clone the code as follows:

\begin{shellbox}
  git clone --branch v\hotmokaVersion{} \hotmokaRepo{}
\end{shellbox}

\noindent
The examples will be inside the Maven module
\texttt{io-hotmoka-tutorial-examples}.

\section{Acknowledgments}

I thank the people at Ailia SA, in particular Giovanni Antino, Mario Carlini,
Iris Dimni and Francesco Pasetto, who decided to invest in the Takamaka project and who are building their own
open-source blockchain that can be programmed in Takamaka. My thank goes also to all students and
colleagues who have read and proof-checked this book and its examples, finding
bugs and inconsistencies; in particular to
Luca Olivieri and Fabio Tagliaferro.
Chapter~\cite{ch:tokens} is shared work with Marco Crosara,
Filippo Fantinato, Luca Olivieri and Fabio Tagliaferro.
Chapter~\cite{ch:hotmoka_nodes} has been
inspired by previous work with Dinu Berinde.
Section~\cite{sec:shared_entities} is shared work
with Andrea Benini, Mauro Gambini and Sara Migliorini.

\begin{center}
  \includegraphics[width=3cm]{pics/docker-hub}
\end{center}

Hotmoka enjoys being a Docker-sponsored open source
project (\url{https://docs.docker.com/trusted-content/dsos-program/}).
DockerHub (\url{https://hub.docker.com/}) provides for free
a repository for the distribution of the docker images of Hotmoka.

\begin{center}
  \includegraphics[width=3cm]{pics/github}
\end{center}

GitHub (\url{https://github.com})
is hosting the code of Hotmoka for free, running tests and
packaging actions at each commit and hosting its releases for download.

\begin{center}
  \includegraphics[width=3cm]{pics/YourKit}
\end{center}

Hotmoka benefits from the use of a free license of the YourKit profiler
for Java. YourKit supports open source projects with innovative and
intelligent tools for monitoring and profiling Java and .NET applications.
YourKit is the creator of YoutKit Java Profiler
(\url{https://www.yourkit.com/java/profiler/}),
YourKit .NET profiler
(\url{https://www.yourkit.com/.net/profiler/})
and YourKit YouMonitor (\url{https://www.yourkit.com/youmonitor/}).

\chapter{Getting started with Hotmoka}\label{ch:getting_started_with_hotmoka}

\section{Hotmoka in a nutshell}\label{hotmoka_in_a_nutshell}

Hotmoka is the abstract definition of a device that can store
objects (data structures) in its persistent memory (its \emph{state} or \emph{storage})
and can execute, on those objects,
code written in a subset of Java called Takamaka. Such a device is
called a \emph{Hotmoka node} and such programs are known as
\emph{smart contracts}, taking that terminology from programs that run inside
a blockchain. It is well true that Hotmoka nodes can be different from the nodes
of a blockchain (for instance, they can be an Internet of Things device);
however, the most prominent application of Hotmoka nodes is, at the
moment, the construction of blockchains whose nodes are Hotmoka nodes.

Every Hotmoka node has its own persistent state, that contains code and
objects. Since Hotmoka nodes are made for running Java code, the code
inside their state is kept in the standard jar format used by Java, while objects
are just a collection of values for their fields, with a class tag that identifies
whose class they belong to and a reference (the \emph{classpath})
to the jar where that class is defined.
While a device of an Internet of Thing network is the sole responsible
for its own state, things are different if a Hotmoka node that is part of a blockchain.
There, the state is synchronized and identical across all nodes of the blockchain.

In object-oriented programming, the units of code that can be run
on an object are called \emph{methods}.
When a method must be run on an object,
that object is identified as the \emph{receiver} of the execution of the method.
The same happens in Hotmoka. That is, when one wants to run a method
on an object, that object must have been already allocated
in the state of the node and must be marked as the receiver of the execution
of the method. Assume for instance that one wants to run a method
on the object in Fig.~\ref{fig:receiver_payer}, identified as receiver.
The code of the method is contained in a jar, previously installed in the state
of the node, and referred to as \emph{classpath}. This is the jar where the class of
the receiver is defined.

\begin{figure}
  \begin{center}
    \includegraphics[width=\textwidth/2]{pics/receiver_payer}
  \end{center}
  \caption{Receiver, payer and classpath for a method call in a Hotmoka node.}
  \label{fig:receiver_payer}
\end{figure}

The main difference with standard object-oriented programming is that Hotmoka requires one
to specify a further object, called \emph{payer}. This is because a Hotmoka node is
a public service, that can be used by everyone has an internet connection
that can reach the node. Therefore, that service must be paid with the
internal cryptocurrency of the node, by providing a measure of execution
effort known as \emph{gas}. The payer is therefore a sort of bank account, whose
balance gets decreased in order to pay for the gas needed for the execution of the method.
The payer is accessible inside the method as its \emph{caller}.

\begin{commentbox}
There are many similarities with what happens in Ethereum: the notion of
receiver, payer and gas are taken from there. There are, however, also
big differences. The first is that the code of the methods is inside
a jar \emph{referenced} by the objects, while Ethereum requires to reinstall
the code of the contracts each time a contract is instantiated.
More importantly, Hotmoka keeps an explicit class tag inside the objects,
while contracts are untyped in Ethereum~\cite{CrafaPZ19}
and are referenced through the untyped \texttt{address} type.
\end{commentbox}

Receiver and payer have different roles but are treated identically in Hotmoka:
they are objects stored in state at their respective state locations, known as
their \emph{storage references}. For instance the payer in
Fig.~\ref{fig:receiver_payer} might be allocated at the storage
reference \texttt{@account1}. A storage reference has two parts, separated
by a \texttt{\#} sign. The first part are 64 hexadecimal digits (\ie, 32 bytes)
that identify the
transaction that created the object; the second part is a progressive number
that identifies an object created during that transaction: the first object
created during the transaction has progressive zero, the second has progressive
one, and so on. When a method is called on a Hotmoka node, what is actually specified
in the call request are the storage references of the receiver and of the payer
(plus the actual arguments to the method, if any).

In Hotmoka, a \emph{transaction} is either
%
\begin{enumerate}
\item the installation of a jar, that modifies the state of the node, and is paid by a payer account, or
\item the execution of a constructor, that yields the storage reference of a new object, or
\item the execution of a method on a receiver, that yields a returned
  value and/or has side-effects that modify the state of the node, and is paid by a payer account.
\end{enumerate}

A Hotmoka node can keep track
of the transactions that it has executed, so that it is possible, for instance,
to recreate its state by running all the transactions executed in the past, starting from
the empty state.

It is very important to discuss at this moment a significant difference with what
happens in Bitcoin, Ethereum and most other blockchains. There, an account
is not an object, nor a contract,
but just a key in the key/value store of the blockchain, whose value is its balance.
The key used for an account is typically computed by hashing the public key derived from
the private key of the account. In some sense, accounts, in those blockchains, exist
independently from the state of the blockchain and can be computed offline: just
create a random private key, compute the associated public key and hence its hash.
Hotmoka is radically different: an account is an object that must be allocated in
state by an explicit construction transaction (that must be paid, as every transaction).
The public key is explicitly stored inside the constructed object
(Base64-encoded in its \texttt{publicKey} field, see Fig.~\ref{fig:receiver_payer}.
That public key was passed as a parameter at the creation of the payer object and
can be passed again for creating more accounts. That is, it is well possible, in Hotmoka,
to have more accounts in the state of a node, all distinct, but controlled by the same key.
It is also possible to rotate the key of an account
(that is, replace it with another key), since the
\texttt{publicKey} field is not \texttt{final}.


\bibliographystyle{plain}

\begin{warpprint}   % For print output ...
\cleardoublepage    % ... a common method to place index entry into TOC.
\phantomsection
\addcontentsline{toc}{chapter}{\indexname}
\printindex
\cleardoublepage    % ... a common method to place bibliography entry into TOC.
\phantomsection
\addcontentsline{toc}{chapter}{Bibliography}
\bibliography{biblio}
\end{warpprint}

\begin{warpHTML}
\ForceHTMLPage      % HTML index will be on its own page.
\ForceHTMLTOC       % HTML index will have its own toc entry.
\printindex
\ForceHTMLPage      % HTML bibliography will be on its own page.
\ForceHTMLTOC       % HTML bibliography will have its own toc entry.
\bibliography{biblio}
\end{warpHTML}

\end{document}
