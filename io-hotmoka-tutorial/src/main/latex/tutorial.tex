% document class options for Amazon's paperback
\documentclass[12pt,oneside,openright]{book}

\usepackage{iftex}

% --- LOAD FONT SELECTION AND ENCODING BEFORE LOADING LWARP ---

\ifPDFTeX
\usepackage{lmodern}            % pdflatex or dvi latex
\usepackage[T1]{fontenc}
\usepackage[utf8]{inputenc}
\else
\usepackage{fontspec}           % XeLaTeX or LuaLaTeX
\fi

% --- LWARP IS LOADED NEXT ---
\usepackage[
%   HomeHTMLFilename=index,     % Filename of the homepage.
%   HTMLFilename={node-},       % Filename prefix of other pages.
%   IndexLanguage=english,      % Language for xindy index, glossary.
%   latexmk,                    % Use latexmk to compile.
%   OSWindows,                  % Force Windows. (Usually automatic.)
%   mathjax,                    % Use MathJax to display math.
]{lwarp}
% \boolfalse{FileSectionNames}  % If false, numbers the files.

% --- LOAD PDFLATEX MATH FONTS HERE ---

% --- OTHER PACKAGES ARE LOADED AFTER LWARP ---

% geometry for Amazon paperback
\usepackage[
    left=0.5in,
    right=0.5in,
    top=0.75in,
    bottom=1.0in,
    paperwidth=8.25in,
    paperheight=11in,
    bindingoffset=0.375in
]{geometry}

\usepackage{amsmath}
\usepackage{amssymb}
\usepackage{amsfonts}
\usepackage{cmap} % Improves copy-paste functionality
\usepackage{alltt}
\usepackage{fancyvrb}

\usepackage[pdftex]{graphicx}

\usepackage{tikz}
%\usetikzlibrary {arrows.meta,graphs,graphdrawing}
%\usegdlibrary{layered}
%\usegdlibrary{trees}
\usetikzlibrary{decorations.pathreplacing}
\usetikzlibrary{arrows.meta}

\usepackage{fancyhdr}
\setlength{\headheight}{19pt} % because the logo uses more space than usual

\newcommand{\myincludegraphics}[2]{\includegraphics[width={#1}]{"tutorial-images/#2"}}
\newcommand{\inputCommand}[1]{\input{../../src/main/latex/generated/#1_command}}
\newcommand{\inputOutput}[1]{\input{../../src/main/latex/generated/#1_output}}

\pagestyle{fancy}

\fancyhf{}%
%\renewcommand{\headrulewidth}{1pt}
\fancyhead[L]{\myincludegraphics{1.5cm}{hotmoka_logo}}
\fancyhead[R]{Chapter~\thechapter}
\fancyfoot[L]{}
\fancyfoot[C]{\thepage}
\fancyfoot[R]{}

\usepackage{makeidx} \makeindex
\usepackage{xcolor}
\usepackage{hyperref,cleveref}  % LOAD THESE LAST!
\usepackage{url}
\DeclareUrlCommand\email{\urlstyle{tt}}
\DeclareUrlCommand\website{\urlstyle{same}}
\DeclareUrlCommand\directory{\urlstyle{same}}
\usepackage{mdframed}
\usepackage{caption}
\usepackage{subcaption}

% --- DOCUMENT PARAMETERS ---

\newcommand{\hotmokaVersion}{1.11.4}
\newcommand{\takamakaVersion}{1.8.0}
\newcommand{\tendermintVersion}{0.34.15}
\newcommand{\faustoEmail}{\email{fausto.spoto@hotmoka.io}}
\newcommand{\hotmokaRepo}{git@github.com:Hotmoka/hotmoka.git}
\newcommand{\hotmokaTutorialDir}{hotmoka\_tutorial}
\newcommand{\gametePublicKeyBaseFiftyeight}{GvH6hJjFNPYdDsnCvTjz83oWX2TeU5djjDSoMUBr7nKR}

\newcommand{\serverMokamint}{ws://panarea.hotmoka.io:8001}
\newcommand{\takamakaCode}{{0d0f8f8232e4931a7f2f2e0fae7c5f63c6c31d35b1fcd22494e7c3a21fb8d2af}}
\newcommand{\takamakaCodeShort}{{0d0f8f8232e4931a\ldots}}
\newcommand{\manifest}{{5aa66b315323008e3f33b51fd74bf3f4ab2b85471fad48a1938e56dbe6e4c344\#0}}
\newcommand{\manifestShort}{{5aa66b315323008e\ldots\#0}}
\newcommand{\gamete}{{40910f188bef89b9d9ce0a66d3019de1b1b6d1a59ac0faf8a02cd0a0e2a3fd4b\#0}}
\newcommand{\gameteShort}{{40910f188bef89b9\ldots\#0}}
\newcommand{\gasStation}{{5aa66b315323008e3f33b51fd74bf3f4ab2b85471fad48a1938e56dbe6e4c344\#e}}
\newcommand{\gasStationShort}{{5aa66b315323008e\ldots\#14}}
\newcommand{\validators}{{5aa66b315323008e3f33b51fd74bf3f4ab2b85471fad48a1938e56dbe6e4c344\#1}}
\newcommand{\validatorsShort}{{5aa66b315323008e\ldots\#1}}
\newcommand{\maxFaucet}{{10000000000000000}}
\newcommand{\chainId}{{octopus}}
\newcommand{\accountOnePublicKeyBaseFiftyeight}{{3tRgiuBeMD2eGwU6XpfUcCnyprysJ6qmvsabiibGPUtr}}
\newcommand{\accountOnePublicKeyBaseSixtyfour}{{KuPoCOon+PMj9CrdrdJ1zFjQsclw9dJfxP4EbJjQi+s=}}
\newcommand{\accountOneTendermintAddress}{{4F572978E8A66BC54378A45DDB7FE2BA5164B197}}
\newcommand{\accountOnePublicKeyBaseSixtyfourShort}{{KuPoCOon+PMj9Crd\ldots}}
\newcommand{\accountOneBalance}{{50000000000000}}
\newcommand{\accountOneTransaction}{{c9f9b29fa48298db4a881096b5870fa0ca7371a846cf96b30fe420c661d3dbe1}}
\newcommand{\accountOne}{{c9f9b29fa48298db4a881096b5870fa0ca7371a846cf96b30fe420c661d3dbe1\#0}}
\newcommand{\accountOneShort}{{c9f9b29fa48298db\ldots\#0}}
\newcommand{\accountOneRecharge}{{200000}}
\newcommand{\accountMokito}{{826b150ccd5bf7ac6d8b07a7d3d12eba2c0eada93a91318e3b8a61397c702412\#0}}
\newcommand{\accountMokitoShort}{{826b150ccd5bf7ac\ldots\#0}}
\newcommand{\publicKeyBaseFifthyeightAnonymous}{{E3ch7Cwko98bgFWwu9woQfDpqZQwnuWuA9P5DnjGmFHm}}
\newcommand{\sentToAnonymous}{{10000}}
\newcommand{\accountAnonymous}{{63cde844a3732bbc61f0ceade21dabea61c4ac4cb8d41b93d821583d1a3236ca\#0}}
\newcommand{\familyAddress}{{1d41e26e1aa396c0a423c24b850e1e45ef58064a0ddab31aef3db5307046ee8b}}
\newcommand{\familyAddressShort}{{1d41e26e1aa396c0\ldots}}
\newcommand{\codeFamilyAddress}{{96793d242344b6f9e3105ba778458f32717b947fbfcff13369305b5c2284c6d5}}
\newcommand{\codeFamilyAddressShort}{{96793d242344b6f9\ldots}}
\newcommand{\familyTwoAddress}{{b2bca897be990178b2f2adcabaad71ee57db342ffff38a359824816699900821}}
\newcommand{\familyTwoAddressShort}{{b2bca897be990178\ldots}}
\newcommand{\personObject}{{443a7bb67488d416d9992f09906ba3c55c845bfe0d9642e813b8ef12f7d1a160\#0}}
\newcommand{\personObjectShort}{{443a7bb67488d416\ldots\#0}}
\newcommand{\personObjectTwo}{{ae5d9f7618c7d0c3026f55824a689c48f74d294b80b5234f89b1c98a7ea47246\#0}}
\newcommand{\personObjectTwoShort}{{ae5d9f7618c7d0c3\ldots\#0}}
\newcommand{\familyExportedAddress}{{1f5be838d40beb17d0695b40b49345b5add0968c8d96a0d50a2e28e28daa28d9}}
\newcommand{\familyExportedAddressShort}{{1f5be838d40beb17\ldots}}
\newcommand{\personExportedObject}{{ae1e3a206ef20917f1a1aa41f42b2851bfa01e56d2e218f2ed0d093ed37c0b6d\#0}}
\newcommand{\personExportedObjectShort}{{ae1e3a206ef20917\ldots\#0}}
\newcommand{\gradualPonziAddress}{{3a3fd6bac9326459ff449e672b9b89dd50eae97aade0247934e18214897dcc1a}}
\newcommand{\gradualPonziAddressShort}{{3a3fd6bac9326459\ldots}}
\newcommand{\accountTwo}{{9ca86f1726c4b292574988163c393674fd61081c6fbf5139239cbc51acd4f4b4\#0}}
\newcommand{\accountTwoShort}{{9ca86f1726c4b292\ldots\#0}}
\newcommand{\accountThree}{{d934daa0590d57cc801e8d09dfd9a31b8335de6f0c231355b83e52454a6b2471\#0}}
\newcommand{\accountThreeShort}{{d934daa0590d57cc\ldots\#0}}
\newcommand{\gradualPonziObject}{{0445f38cfb110f876a555c9540f53b365fdd43b0d697053fc5a832538509ea9e\#0}}
\newcommand{\gradualPonziObjectShort}{{0445f38cfb110f87\ldots\#0}}
\newcommand{\ticTacToeAddress}{{1aec967cc6221c72dcc7fc289470174575932e18e7229f0635d64dec20b0d7f3}}
\newcommand{\ticTacToeAddressShort}{{1aec967cc6221c72\ldots}}
\newcommand{\ticTacToeObject}{{06eda349956fbd428d9861885c2667d31c53b679b30932653509198c018bd723\#0}}
\newcommand{\ticTacToeObjectShort}{{06eda349956fbd42\ldots\#0}}
\newcommand{\ercTwentyAddress}{{465a5c54f29d39199d1374cc52af5b4ab6f5dd5f6735b7325b0ee2b3af54e29d}}
\newcommand{\ercTwentyAddressShort}{{465a5c54f29d3919\ldots}}
\newcommand{\ercTwentyObject}{{15974182b4c291e09377f1a5ba9954424e74d2572a335306831f783d8ccc30cb\#0}}
\newcommand{\ercTwentyObjectShort}{{15974182b4c291e0\ldots\#0}}

\newcommand{\serverTendermintChainId}{mriya}
\newcommand{\accountFour}{8173982155b864f1a12ebe1f9febade73bd9ebf42fd66fef6283bff2038bf728\#0}
\newcommand{\accountFourShort}{8173982155b864f1\ldots\#0}
\newcommand{\accountFive}{97aa0e09a2d59f6c37895cce0b7bd3d4b21c77e7851d8c73ac527553fba773ea\#0}
\newcommand{\accountFiveShort}{97aa0e09a2d59f6c\ldots\#0}
\newcommand{\accountSix}{048998b700cb947fe3cebe37d83d5b103d12d60d80e6611ca780408f12cddac2\#0}
\newcommand{\accountSixShort}{048998b700cb947f\ldots\#0}

\newcommand{\accountSeven}{a5522145eab808a984ba659fb9c52778bf11567119359eafb3af7a5e99901038\#0}
\newcommand{\accountSevenShort}{a5522145eab808a9\ldots\#0}
\newcommand{\accountEight}{9b7158bc38140c3fa36f3ddb6a1919c5c81d295d9624727c8fade9225bc24500\#0}
\newcommand{\accountEightShort}{9b7158bc38140c3f\ldots\#0}
\newcommand{\accountNine}{1ed8d6903f5fe0a97a3f588f3b500a49abc6627ea8d0046510259948b4752199\#0}
\newcommand{\accountNineShort}{1ed8d6903f5fe0a9\ldots\#0}
\newcommand{\accountTen}{0840a5a5963aa2c5b3e4150c8c74d91ed56dcd2ff79636e970c1d3f5e47f6ce1\#0}
\newcommand{\accountTenShort}{0840a5a5963aa2c5\ldots\#0}
\newcommand{\familyQuantumAddress}{97d5dce010b802a49d49946ac3be92ef241f34b17f7aa48cc2363c7c154e1e74}
\newcommand{\familyQuantumAddressShort}{97d5dce010b802a4\ldots}


% --- LATEX AND HTML CUSTOMIZATION ---
\title{Programming Hotmoka\\{\small A tutorial on Hotmoka and smart contracts in Takamaka}}
\author{Fausto Spoto (\faustoEmail{})}
\date{\today\\{\small (updated to Hotmoka \hotmokaVersion{}, Mokamint \mokamintVersion{} and Takamaka \takamakaVersion{})} \\
  \vspace*{10ex}
  \hotmokaWeb{} \\
  \vspace*{30ex}
  {\small\begin{minipage}{\textwidth*4/5}
    \begin{center}\myincludegraphics{11ex}{CC_license}\end{center}
    This document is licensed under a Creative Commons Attribution 4.0 International License.
    Its latest version is available for free in various formats at \hotmokaRepoReleases{}.
  \end{minipage}}
}
\setcounter{tocdepth}{2}        % Include subsections in the \TOC.
\setcounter{secnumdepth}{2}     % Number down to subsections.
\setcounter{FileDepth}{1}       % Split \HTML\ files at sections
\booltrue{CombineHigherDepths}  % Combine parts/chapters/sections
\setcounter{SideTOCDepth}{2}    % Include subsections in the side\TOC
\HTMLTitle{Programming Hotmoka} % Overrides \title for the web page.
\HTMLAuthor{Fausto Spoto}       % Sets the HTML meta author tag.
\HTMLLanguage{en-US}            % Sets the HTML meta language.
\HTMLDescription{A tutorial on Hotmoka and smart contracts in Takamaka} % Sets the HTML meta description.
\HTMLFirstPageTop{\begin{center}\myincludegraphics{10cm}{hotmoka_logo}\end{center}}
\HTMLPageTop{\begin{center}\myincludegraphics{7cm}{hotmoka_logo}\end{center}}
\HTMLPageBottom{Copyright 2022 by Fausto Spoto (\faustoEmail{})}
\CSSFilename{lwarp_sagebrush.css}

\definecolor{objectbackgroundcolor}{rgb}{1,0.6,0.7}
\definecolor{jarbackgroundcolor}{rgb}{0.9,0.9,0.3}
\definecolor{statebackgroundcolor}{rgb}{0.6,0.6,1.0}

\definecolor{commentboxbackgroundcolor}{rgb}{0.9,0.85,1.0}
\newmdenv[linecolor=black,leftmargin=1em,rightmargin=1em,innertopmargin=1em,innerbottommargin=1em,backgroundcolor={commentboxbackgroundcolor}]{commentbox}

\definecolor{shellboxbackgroundcolor}{rgb}{0.8,1.0,0.8}
\newmdenv[linecolor=black,leftmargin=1em,rightmargin=1em,backgroundcolor={shellboxbackgroundcolor}]{shellbox}

\definecolor{shellcommandboxbackgroundcolor}{rgb}{1.0,0.9,0.9}
\newmdenv[linecolor=black,leftmargin=1em,rightmargin=1em,backgroundcolor={shellcommandboxbackgroundcolor}]{shellcommandbox}

\definecolor{codeboxbackgroundcolor}{rgb}{0.9,0.9,0.9}
\newmdenv[linecolor=black,leftmargin=1em,rightmargin=1em,backgroundcolor={codeboxbackgroundcolor}]{codebox}

\usepackage{listings} 

\lstset{% 
  basicstyle=\ttfamily\small, 
  breaklines=true,
  columns=fullflexible,
  %escapeinside = {§}{§},
  breakindent=0pt} 
\lstnewenvironment{ttlst}{}{} 

\lstdefinestyle{myJava}{
tabsize = 2, %% set tab space width
showstringspaces = false, %% prevent space marking in strings, string is defined as the text that is generally printed directly to the console
%numbers = left, %% display line numbers on the left
commentstyle = \color{violet}, %% set comment color
keywordstyle = \color{blue}, %% set keyword color
stringstyle = \color{red}, %% set string color
rulecolor = \color{black}, %% set frame color to avoid being affected by text color
basicstyle = \small \ttfamily , %% set listing font and size
breaklines = true, %% enable line breaking
%numberstyle = \tiny,
%escapeinside = ||,
language = Java,
showstringspaces=false,
morekeywords={module,requires,exports,provides}
%frame = trBL,
%firstnumber = last
}

\lstnewenvironment{javalst}[1][]{
  \lstset{style=myJava, #1} % Apply the style and any extra options
}{}

\lstdefinestyle{myXML}{
  language=XML,
  escapeinside = ||,
  tabsize = 2, %% set tab space width
  showstringspaces=false,
  commentstyle = \color{violet}, %% set comment color
  keywordstyle = \color{blue}, %% set keyword color
  stringstyle = \color{red}, %% set string color
  rulecolor = \color{black}, %% set frame color to avoid being affected by text color
  basicstyle = \small \ttfamily , %% set listing font and size
  morekeywords={encoding,project,modelVersion,groupId,artifactId,properties,dependencies,dependency,build,plugins,plugin,maven,compiler,release,sourceEncoding,xs:schema,xs:element,xs:complexType,xs:sequence,xs:attribute}
}

\lstnewenvironment{xmllst}[1][]{
  \lstset{style=myXML, #1} % Apply the style and any extra options
}{}

\newcommand{\ie}{\emph{i.e.}\@}
\newcommand{\wrt}{\emph{wrt.}\@}
\newcommand{\leftplug}{\myincludegraphics{0.7cm}{plug_left}}
\newcommand{\rightplug}{\myincludegraphics{0.7cm}{plug_right}}
\newcommand{\ssd}{\myincludegraphics{0.8cm}{ssd}}

\colorlet{lightred}{red!50!white}
\colorlet{lightviolet}{violet!50!white}
\colorlet{verylightgray}{gray!25!white}

\begin{document}

\maketitle                      % Or titlepage/titlingpage environment.

% An article abstract would go here.

\tableofcontents                % MUST BE BEFORE THE FIRST SECTION BREAK!
%\listoffigures

\input{1 introduction}
\input{2 getting_started}
\input{3 first_takamaka_program}
\input{4 smart_contracts}
\chapter{The support library}\label{ch:support_library}

This chapter presents the support library of the Takamaka language,
that contains classes for simplifying the definition of smart contracts.

In Sec.~\ref{sec:storage_types_and_classes},
we said that storage objects must obey to some constraints.
The strongest of them is that their fields of reference type, in turn, can only hold
storage objects. In particular, arrays are not allowed there. This can
be problematic, in particular for contracts that deal with a
dynamic, variable, potentially unbound number of other contracts.

Therefore, most classes of the support library deal
with such constraints, by providing fixed or variable-sized collections
that can be used in storage objects, since they are storage objects themselves.
Such utility classes implement lists, arrays and maps and are
consequently generally described as \emph{collections}. They have the
property of being storage classes, hence their instances can be kept in
the store of a Hotmoka node,
\emph{as long as only storage objects are added as elements of the collection}.
As usual with collections, these utility classes
have generic type, to implement collections of arbitrary, but fixed
types. This is not problematic, since Java (and hence Takamaka) allows generic types.

\section{Storage lists}\label{sec:storage_lists}

Lists are an ordered sequence of elements. In a list, it is typically
possible to access (read or write) the first element in constant time, while accesses
to the \emph{n}th element require to scan the list from its head and
consequently have a cost proportional to \emph{n}. Because of this,
lists are \emph{not}, in general, random-access data structures, whose \emph{n}th
element should be accessible in constant time.
The size of a list is not fixed: lists grow in size as more elements are added.

\begin{figure}[th]
  \begin{center}
	\begin{tikzpicture}[scale=1]
	\scriptsize
	\externalclasscolor
	\begin{interface}[text width=2cm]{Iterable\string<E\string>}{0,0}
	\end{interface}
	\interfacecolor
	\begin{interface}[text width=5.6cm]{StorageListView\string<E\string>}{0,-1.5}
		\inherit{Iterable\string<E\string>}
		\operation{@View contains(e:Object):boolean}
		\operation{@View first():E}
		\operation{@View last():E}
		\operation{@View get(index:int):E}
		\operation{@View size():int}
		\operation{toArray(generator:IntFunction\string<E[]\string>):E[]}
		\operation{forEach(action:Consumer\string<? super E\string>)}
	\end{interface}
	\begin{interface}[text width=5.3cm]{SnapshottableStorageListView\string<E\string>}{4.5,-5.5}
		\inherit{StorageListView\string<E\string>}
		\operation{snapshot():StorageListView\string<E\string>}
	\end{interface}
	\begin{interface}[text width=4cm]{StorageList\string<E\string>}{-4,-5.5}
		\inherit{StorageListView\string<E\string>}
		\operation{addFirst(element:E)}
		\operation{addLast(element:E)}
		\operation{add(element:E)}
		\operation{clear()}
		\operation{remove(e:Object):boolean}
		\operation{view():StorageListView\string<E\string>}
	\end{interface}
	\begin{interface}[text width=5.6cm]{SnapshottableStorageList\string<E\string>}{4.5,-8}
		\inherit{StorageList\string<E\string>}
		\inherit{SnapshottableStorageListView\string<E\string>}
		\operation{view():SnapshottableStorageListView\string<E\string>}
	\end{interface}
	\externalclasscolor
	\begin{class}[text width=2cm]{Storage}{-0.5,-9}
	\end{class}
	\classcolor
	\begin{class}[text width=3.5cm]{StorageLinkedList\string<E\string>}{0,-10.5}
		\inherit{Storage}
		\implement{SnapshottableStorageList\string<E\string>}
		\operation{StorageLinkedList()}
	\end{class}
	\end{tikzpicture}
  \end{center}
  \caption{The hierarchy of storage lists.}
  \label{fig:lists_hierarchy}
\end{figure}

Java has many classes for implementing lists, all subclasses
of \texttt{java.util.List<E>}. They cannot be used in Takamaka that, instead,
provides an implementation of lists with the storage class
\texttt{io.takamaka.code.util.StorageLinkedList<E>}\index{StorageLinkedList@{\texttt{StorageLinkedList}}}.
Its instances are storage objects and
can consequently be held in fields of storage classes and
can be stored in a Hotmoka node,
\emph{as long as only storage objects are added to the list}. Takamaka lists provide
constant-time access and addition to both ends of a list.
We refer to the JavaDoc of \texttt{StorageLinkedList<E>} for a full description of its methods.
They include methods for adding elements to either end of the list, for accessing and
removing elements, for iterating on a list and for building a Java array
\texttt{E[]} holding the elements of a list.

Fig.~\ref{fig:lists_hierarchy} shows the hierarchy of the storage lists.
A storage list implements the interface \texttt{StorageList<E>}\index{StorageList@{\texttt{StorageList}}},
that defines the methods that modify a list.
That interface extends the interface
\texttt{StorageListView<E>}\index{StorageListView@{\texttt{StorageListView}}}
that, instead, defines the methods
that read data from a list, but do not modify it. This distinction between the \emph{read-only}
interface and the \emph{modification} interface is typical of all collection classes in the
Takamaka library, as we will see. For the moment, note that this distinction is useful
for defining methods \texttt{snapshot()}\index{snapshot()@{\texttt{snapshot()}}}
and \texttt{view()}\index{view()@{\texttt{view()}}}. Both return
a \texttt{StorageListView<E>} but there
is an important difference between them. Namely, \texttt{snapshot()} yields a
\emph{frozen} view of the list, that cannot and will never be modified,
also if the original list gets subsequently updated. Instead,
\texttt{view()} yields a \emph{view} of a list, that is, a read-only list that changes whenever
the original list changes and exactly in the same way: if an element is added to the original
list, the same automatically occurs to the view.
In this sense, a view is just a read-only alias of the original list.
Both methods can be useful to export data, safely,
from a node to the outside world, since both methods
return an \texttt{@Exported}\index{Exported@{\texttt{Exported}}}
object without modification methods.
Method \texttt{snapshot()} runs in linear time (in the length of the list)
while method \texttt{view()} runs in constant time.
%
\begin{commentbox}
Differently from the other collection classes that we will describe in this chapter,
the \texttt{snapshot()} method of lists runs in linear time on the length of the list.
This is inherently related to the structure of a linked list, that requires a full
copy in order to create an unmodifiable clone. Other collections, later, will instead
allow the creation of snapshots in constant time.
\end{commentbox}
%
\begin{commentbox}
It might seem that \texttt{view()} is just an upwards cast to the
interface \texttt{StorageListView<E>}. This is wrong, since that method
does much more. Namely, it applies the façade design pattern
to provide a \emph{distinct} list that lacks any modification method
and implements a façade of the original list.
To appreciate the difference to a cast, assume to have a \texttt{StorageList<E> list} and to write
%
\begin{codebox}\begin{javalst}
StorageListView<E> view = (StorageListView<E>) list;
\end{javalst}\end{codebox}
%
This upwards cast will always succeed.
Variable \texttt{view} does not allow to call any modification method, since they
are not in its static type \texttt{StorageListView<E>}. But a downwards cast back
to \texttt{StorageList<E>} is enough to circumvent
that constraint: \texttt{StorageList<E> list2 = (StorageList<E>) view}.
This way, the original \texttt{list}
can be modified by modifying \texttt{list2} and it would not be safe to export
\texttt{view}, since it
is a Trojan horse for the modification of \texttt{list}. With method \texttt{view()}, the
problem does not arise, since the cast \texttt{StorageList<E> list2 = (StorageList<E>) list.view()}
fails: method \texttt{view()} actually returns another list object without modification methods.
The same is true for method \texttt{snapshot()} that, moreover, yields a frozen view of the
original list. These same considerations hold for the other Takamaka collections that we will
see in this chapter.
\end{commentbox}

Next section shows an example of use for \texttt{StorageLinkedList}.

\subsection{A gradual Ponzi contract}\label{subsec:gradual_ponzi}

\begin{center}
(See the \texttt{io-hotmoka-tutorial-examples-tictactoe} project in \texttt{\hotmokaRepo{}})
\end{center}

Consider our previous Ponzi contract from Ch.~\ref{ch:smart_contracts}
It is somehow irrealistic, since
an investor gets its investment back in full. In a more realistic scenario,
the investor will receive the investment back gradually, as soon as new
investors arrive. This is more complex to program, since
the Ponzi contract must take note of all investors that invested up to now,
not just of the current one as in \emph{SimplePonzi.java}. This requires a
list of investors, of unbounded size. An implementation of this gradual
Ponzi contract is reported below and has been
inspired by a similar Ethereum contract from Iyer and Dannen,
shown at page~150 of~\cite{IyerD18}.
Write its code inside package \texttt{io.hotmoka.tutorial.examples.ponzi} of
the \texttt{io-hotmoka-tutorial-examples-ponzi} project, as a new class
\texttt{GradualPonzi.java}:
%
\begin{codebox}\begin{javalst}
package io.hotmoka.tutorial.examples.ponzi;

import static io.takamaka.code.lang.Takamaka.require;

import java.math.BigInteger;

import io.takamaka.code.lang.Contract;
import io.takamaka.code.lang.FromContract;
import io.takamaka.code.lang.Payable;
import io.takamaka.code.lang.PayableContract;
import io.takamaka.code.lang.StringSupport;
import io.takamaka.code.math.BigIntegerSupport;
import io.takamaka.code.util.StorageLinkedList;
import io.takamaka.code.util.StorageList;

public class GradualPonzi extends Contract {
  public final BigInteger MINIMUM_INVESTMENT = BigInteger.valueOf(1_000L);

  /**
   * All investors up to now. This list might contain the same investor many
   * times, which is important to pay him back more than investors
   * who only invested once.
   */
  private final StorageList<PayableContract> investors = new StorageLinkedList<>();

  public @FromContract(PayableContract.class) GradualPonzi() {
    investors.add((PayableContract) caller());
  }

  public @Payable @FromContract(PayableContract.class) void invest(BigInteger amount) {
    // new investments must be at least 10% greater than current
    require(BigIntegerSupport.compareTo(amount, MINIMUM_INVESTMENT) >= 0,
      () -> StringSupport.concat("you must invest at least ", MINIMUM_INVESTMENT));
    BigInteger eachInvestorGets = BigIntegerSupport.divide
      (amount, BigInteger.valueOf(investors.size()));
    investors.forEach(investor -> investor.receive(eachInvestorGets));
    investors.add((PayableContract) caller());
  }
}
\end{javalst}\end{codebox}

The constructor of \texttt{GradualPonzi} is annotated as \texttt{@FromContract}. Therefore,
it can only be called by a contract and the latter that gets added, as first investor,
inside the field \texttt{investors}, of type \texttt{io.takamaka.code.util.StorageLinkedList}.
This list, that implements an unbounded list of objects,
is a storage object, as long as only storage objects are
added inside it. \texttt{PayableContract}s are storage objects, hence
its use is correct here.
Subsequently, other contracts can invest by calling method \texttt{invest()}.
A minimum investment is required, but this remains constant over time.
The \texttt{amount} invested gets split by the number of the previous investors
and sent back to each of them. Note that Takamaka allows programmers to use Java's lambdas.
Old fashioned Java programmers, who don't feel at home with such treats,
can exploit the fact that
storage lists are iterable and replace the single-line \texttt{forEach()} call
with a more traditional (but gas-hungrier):
%
\begin{codebox}\begin{javalst}
for (PayableContract investor: investors)
  investor.receive(eachInvestorGets);
\end{javalst}\end{codebox}
%
It is instead \emph{highly discouraged} to iterate the list as if it were an
array. Namely, \emph{do not write}
%
\begin{codebox}\begin{javalst}
for (int pos = 0; pos < investors.size(); pos++)
  investors.get(i).receive(eachInvestorGets);
\end{javalst}\end{codebox}
%
since linked lists are not random-access data structures and the complexity of the
last loop is quadratic in the size of the list. This is not a novelty: the
same occurs with many traditional Java lists that do not implement
\texttt{java.util.RandomAccess} (such as \texttt{java.util.LinkedList}).
In Takamaka, code execution costs gas and
computational complexity does matter, more than in other programming contexts.

\subsection{A note on re-entrancy}\label{subsec:reentrancy}

The \texttt{GradualPonzi.java} class pays back previous investors immediately:
as soon as a new investor invests something, his investment gets
split and forwarded to all previous investors. This should
make Solidity programmers uncomfortable, since the same approach,
in Solidity, might lead to the infamous re-entrancy\index{re-entrancy} attack, when the
contract that receives his investment back has a
fallback function redefined in such a way to re-enter the paying contract and
re-execute the distribution of the investment.
As it is well known, such an attack has made some people rich and other
desperate. You can find more detail
in~\cite{AntonopoulosWPMP25}.
Even if such a frightening scenario does not occur,
paying back previous investors immediately is discouraged in Solidity
also for other reasons. Namely, the contract that receives his
investment back might have a redefined fallback function that
consumes too much gas or does not terminate. This would hang the
loop that pays back previous investors, actually locking the
money inside the \texttt{GradualPonzi} contract. Moreover, paying back
a contract is a relatively expensive operation in Solidity, even if the
fallback function is not redefined, and this cost is paid by the
new investor that called \texttt{invest()}, in terms of gas. The cost is linear
in the number of investors that must be paid back.

As a solution to these problems, Solidity programmers do not pay previous
investors back immediately, but let the \texttt{GradualPonzi} contract take
note of the balance of each investor, through a map.
This map is updated as soon as a new investor arrives, by increasing the
balance of every previous investor. The cost of updating the balances
is still linear in the number of previous investors, but it is cheaper
(in Solidity) than sending money back to each of them, which
requires expensive inter-contract calls that trigger new sub-transactions.
With this technique, previous investors are
now required to withdraw their balance explicitly and voluntarily,
through a call to some function, typically called \texttt{widthdraw()}.
This leads to the \emph{withdrawal pattern}, widely used for writing Solidity contracts.

We have not used the withdrawal pattern in \texttt{GradualPonzi.java}. In general,
there is no need for such pattern in Takamaka, at least not for simple
contracts like \texttt{GradualPonzi.java}. The reason is that the
\texttt{receive()}\index{receive()@{\texttt{receive()}}}
method of a payable contract (corresponding to the
fallback function of Solidity) are \texttt{final} in Takamaka and very cheap
in terms of gas. In particular, inter-contract calls are not
especially expensive in Takamaka, since they are just a method
invocation in Java bytecode (one bytecode instruction). They are \emph{not} inner transactions.
They are actually cheaper than
updating a map of balances. Moreover, avoiding the \texttt{withdraw()} transactions
reduces the overall number of transactions;
without using the map supporting the withdrawal pattern, Takamaka contracts
consume less gas and less storage.
Hence, the withdrawal pattern is both
useless in Takamaka and more expensive than paying back previous contracts immediately.

\subsection{Running the gradual Ponzi contract}\label{subsec:running_gradual_ponzi}

Let us play with the \texttt{GradualPonzi} contract now.
We can now start by installing its jar in the node:
%
\index{moka!jars install@{\texttt{jars install}}}
\inputCommand{moka_jars_install_gradual_ponzi}
\inputOutput{moka_jars_install_gradual_ponzi}
%
Create two more keys now, for two more accounts that we are going to create soon:
%
\index{moka!keys create@{\texttt{keys create}}}
\inputCommand{moka_keys_create_account2}
\inputOutput{moka_keys_create_account2}
%
and then
%
\inputCommand{moka_keys_create_account3}
\inputOutput{moka_keys_create_account3}
%
We can create the two new accounts now:
%
\index{moka!accounts create@{\texttt{accounts create}}}
\inputCommand{moka_accounts_create_account2}
\inputOutput{moka_accounts_create_account2}
%
and
%
\inputCommand{moka_accounts_create_account3}
\inputOutput{moka_accounts_create_account3}

We let our first account create an instance of \texttt{GradualPonzi} in the node now
and become the first investor of the contract:
%
\index{moka!objects create@{\texttt{objects create}}}
\inputCommand{moka_objects_create_gradual_ponzi}
\inputOutput{moka_objects_create_gradual_ponzi}
%
We let the other two players invest, in sequence, in this new \texttt{GradualPonzi} contract.
First investment:
%
\index{moka!objects call@{\texttt{objects call}}}
\inputCommand{moka_objects_call_invest_1}
\inputOutput{moka_objects_call_invest_1}
%
Second investment:
%
\inputCommand{moka_objects_call_invest_2}
\inputOutput{moka_objects_call_invest_2}

We let the first player try to invest again in the contract now, this time
with a too small investment, which leads to an exception,
since the code of the contract requires a minimum investment:
%
\inputCommand{moka_objects_call_invest_3}
\inputOutput{moka_objects_call_invest_3}
%
This exception states that a transaction failed because the last
investor invested less than 1000 units of coin. Note that the
exception message reports the cause (a \texttt{require} failed)
and includes the source program line
of the contract where the exception occurred:
line $65$ of the source file \texttt{GradualPonzi.java}, that is line
%
\index{BigIntegerSupport@{\texttt{BigIntegerSupport}}}
\begin{codebox}\begin{javalst}
require(BigIntegerSupport.compareTo(amount, MINIMUM_INVESTMENT) >= 0,
  () -> StringSupport.concat("you must invest at least ", MINIMUM_INVESTMENT));
\end{javalst}\end{codebox}
%
Finally, we can check the state of the contract:
%
\index{moka!objects show@{\texttt{objects show}}}
\inputCommand{moka_objects_show_gradual_ponzi}
\inputOutput{moka_objects_show_gradual_ponzi}
%
As you can see, the contract keeps no balance. Moreover, its \texttt{investors} field is bound to an
object, whose state can be further investigated:
%
\inputCommand{moka_objects_show_investors}
\inputOutput{moka_objects_show_investors}
%
As you can see, it is a \texttt{StorageLinkedList} of size three, since it contains
our three accounts that interacted with the \texttt{GradualPonzi} contract instance.

\section{Storage arrays}\label{sec:storage_arrays}

Arrays are an ordered sequence of elements, with constant-time access
to such elements, both for reading and for writing. The size of the arrays is typically
fixed, although there are programming languages with limited forms of dynamic arrays.

Java has native arrays, of type \texttt{E[]}, where \texttt{E} is the
type of the elements of the array. They can be used in Takamaka, but not
as fields of storage classes. For that, Takamaka provides class
\texttt{io.takamaka.code.util.StorageTreeArray<E>}\index{StorageTreeArray@{\texttt{StorageTreeArray}}}
and class
\texttt{SnapshottableStorageTreeArray<E>}\index{SnapshottableStorageTreeArray@{\texttt{SnapshottableStorageTreeArray}}},
whose instances are storage objects and
can consequently be held in fields of storage classes and
can be stored in the store of a Hotmoka node, \emph{as long as only
storage objects are added inside them}. Their size is fixed and decided
at the time of construction. Although we consider such classes
as the storage replacement for Java arrays, it must be stated that the complexity of
accessing their elements is logarithmic in the size of the array, which is
a significant deviation from the standard definition of arrays. Nevertheless,
logarithmic complexity is much better than the linear complexity for
accessing elements of a \texttt{StorageLinkedList<E>} that, instead, has the advantage
of being dynamic in size. The difference between the two classes
is that instances of \texttt{SnapshottableStorageTreeArray<E>} allow the construction
of snapshots, which increases their cost in space, time and consequently gas.
Therefore, it is better to use a \texttt{StorageTreeArray<E>} if the
snapshotting feature is not used.

\begin{figure}[th]
  \begin{center}
	\begin{tikzpicture}[scale=1]
	\scriptsize
	\externalclasscolor
	\begin{interface}[text width=2cm]{Iterable\string<E\string>}{0,0}
	\end{interface}
	\interfacecolor
	\begin{interface}[text width=8.5cm]{StorageArrayView\string<E\string>}{0,-1.5}
		\inherit{Iterable\string<E\string>}
		\operation{@View length():int}
		\operation{@View get(index:int):E}
		\operation{@View getOrDefault(index:int, \string_default:E):E}
		\operation{getOrDefault(index:int, \string_default:Supplier\string<? extends E\string>):E}
		\operation{toArray(generator:IntFunction\string<E[]\string>):E[]}
		\operation{forEach(action:Consumer\string<? super E\string>)}
	\end{interface}
	\begin{interface}[text width=6cm]{SnapshottableStorageArrayView\string<E\string>}{4.5,-6.5}
		\inherit{StorageArrayView\string<E\string>}
		\operation{snapshot():StorageArrayView\string<E\string>}
	\end{interface}
	\begin{interface}[text width=9cm]{StorageArray\string<E\string>}{-4,-5.5}
		\inherit{StorageArrayView\string<E\string>}
		\operation{set(index:int, value:E)}
		\operation{update(index:int, how:UnaryOperator\string<E\string>)}
		\operation{setIfAbsent(index:int, value:E):E}
		\operation{computeIfAbsent(index:int, supplier:IntFunction\string<? extends E\string>):E}
		\operation{view():StorageArrayView\string<E\string>}
	\end{interface}
	\begin{interface}[text width=5.2cm]{SnapshottableStorageArray\string<E\string>}{4.5,-9}
		\inherit{StorageArray\string<E\string>}
		\inherit{SnapshottableStorageArrayView\string<E\string>}
		\operation{view():SnapshottableStorageArrayView\string<E\string>}
	\end{interface}
	\externalclasscolor
	\begin{class}[text width=2cm]{Storage}{0,-9}
	\end{class}
	\classcolor
	\begin{class}[text width=6cm]{StorageTreeArray\string<E\string>}{-5,-10}
		\inherit{Storage}
		\implement{StorageArray\string<E\string>}
		\operation{StorageTreeArray(length:int)}
		\operation{StorageTreeArray(length:int, initialValue:E)}
	\end{class}
	\begin{class}[text width=7.5cm]{SnapshottableStorageTreeArray\string<E\string>}{3,-11}
		\inherit{Storage}
		\implement{SnapshottableStorageArray\string<E\string>}
		\operation{SnapshottableStorageArray(length:int)}
		\operation{SnapshottableStorageArray(length:int, initialValue:E)}
	\end{class}
	\end{tikzpicture}
  \end{center}
  \caption{The hierarchy of storage arrays.}
  \label{fig:arrays_hierarchy}
\end{figure}

We refer to the JavaDoc of \texttt{StorageTreeArray<E>}
and \texttt{SnapshottableStorageTreeArray<E>} for a full list of their methods.
There are methods for adding elements, for accessing and
removing elements, for iterating on an array and for building a Java array
\texttt{E[]} with the elements of a \texttt{StorageTreeArray<E>}
or \texttt{SnapshottableStorageTreeArray<E>}.
Fig.~\ref{fig:arrays_hierarchy} shows the hierarchy of such array classes.
They implement the interface
\texttt{StorageArray<E>}\index{StorageArray@{\texttt{StorageArray}}},
that defines the methods that modify an array.
That interface extends the interface
\texttt{StorageArrayView<E>}\index{StorageArrayView@{\texttt{StorageArrayView}}}
that, instead, defines the methods
that read data from an array, but do not modify it.
This distinction between the \emph{read-only}
interface and the \emph{modification} interface is identical
to what we have seen for lists in the previous
sections. All array classes have a method
\texttt{view()}\index{view()@{\texttt{view()}}} while only
snapshottable arrays have a method
\texttt{snapshot()}\index{snapshot()@{\texttt{snapshot()}}}, as for lists.
Both methods yield \texttt{@Exported}
storage arrays in constant time. All constructors of the
arrays classes require to specify the unmodifiable
size of the array. Moreover, it is possible to specify
a default value for the elements of the
array, that can be explicit or given as a supplier, possibly indexed.

Next section shows an example of use for \texttt{StorageTreeArray<E>}.

\subsection{A tic-tac-toe contract}\label{subsec:tic_tac_toe}

\begin{center}
(See the \texttt{io-hotmoka-tutorial-examples-tictactoe} project in \texttt{\hotmokaRepo{}})
\end{center}

Tic-tac-toe is a game where two players place, alternately,
a cross and a circle on a $3\times 3$ board, initially empty. The winner is the
player who places three crosses or three circles on the same row,
column or diagonal. For instance, in Fig.~\ref{fig:cross_wins} the player of
the cross wins.
%
\begin{figure}
\centering
\begin{subfigure}{.5\textwidth}
  \centering
  \myincludegraphics{.4\linewidth}{tictactoe_wins}
  \caption{Cross wins.}
  \label{fig:cross_wins}
\end{subfigure}%
\begin{subfigure}{.5\textwidth}
  \centering
  \myincludegraphics{.5\linewidth}{tictactoe_draw}
  \caption{A draw.}
  \label{fig:tictactoe_draw}
\end{subfigure}
\caption{The two ways a tic-tac-toe game can end: victory of one of the players or a draw.}
\end{figure}
%
Some games that end up in a draw, when the board is full but nobody wins,
as in Fig.~\ref{fig:tictactoe_draw}.

A natural representation of the tic-tac-toe board is a two-dimensional array
where indexes are distributed as shown in Fig.~\ref{fig:tictactoe_grid},
%
\begin{figure}
\centering
\begin{subfigure}{.5\textwidth}
  \centering
  \myincludegraphics{.5\linewidth}{tictactoe_grid}
  \caption{A two-dimensional representation of the board.}
  \label{fig:tictactoe_grid}
\end{subfigure}%
\begin{subfigure}{.5\textwidth}
  \centering
  \myincludegraphics{.42\linewidth}{tictactoe_grid_linear}
  \caption{A linear representation of the board.}
  \label{fig:tictactoe_linear}
\end{subfigure}
\caption{Two alternative representations of the board of the game.}
\end{figure}
%
implemented as a \texttt{StorageTreeArray<StorageTreeArray<Tile>>}, where \texttt{Tile} is
a class that enumerates the three possible tiles (empty, cross, circle). This is
possible but overkill. It is simpler and cheaper (also in terms of gas)
to use the previous diagram as a conceptual representation of the board
shown to the users, but use, internally,
a one-dimensional array of nine tiles, distributed as in
Fig.~\ref{fig:tictactoe_linear}.
This one-dimensional array can be implemented as a \texttt{StorageTreeArray<Tile>}.
There will be functions
for translating the conceptual representation into the internal one.

Create hence in Eclipse a new Maven Java 21 (or later) project. Use for this project the name
\texttt{io-hotmoka-tutorial-examples-tictactoe}.
You can do this by duplicating the project \texttt{io-hotmoka-tutorial-examples-family}.
Use the following \texttt{pom.xml}:
%
\begin{codebox}\begin{xmllst}
<project xmlns="http://maven.apache.org/POM/4.0.0"
    xmlns:xsi="http://www.w3.org/2001/XMLSchema-instance"
    xsi:schemaLocation="http://maven.apache.org/POM/4.0.0 http://maven.apache.org/xsd/maven-4.0.0.xsd">

  <modelVersion>4.0.0</modelVersion>
  <groupId>io.hotmoka</groupId>
  <artifactId>io-hotmoka-tutorial-examples-tictactoe</artifactId>
  <version>|\hotmokaVersion{}|</version>

  <properties>
    <project.build.sourceEncoding>UTF-8</project.build.sourceEncoding>
    <maven.compiler.release>21</maven.compiler.release>
  </properties>

  <dependencies>
    <dependency>
      <groupId>io.hotmoka</groupId>
      <artifactId>io-takamaka-code</artifactId>
      <version>|\takamakaVersion{}|</version>
    </dependency>
  </dependencies>

  <build>
    <plugins>
      <plugin>
        <groupId>org.apache.maven.plugins</groupId>
        <artifactId>maven-compiler-plugin</artifactId>
        <version>3.11.0</version>
      </plugin>
    </plugins>
  </build>

</project>
\end{xmllst}\end{codebox}
%
and the following \texttt{module-info.java}:
%
\begin{codebox}\begin{javalst}
module tictactoe {
  requires io.takamaka.code;
}
\end{javalst}\end{codebox}
%
Create package \texttt{io.hotmoka.tutorial.examples.tictactoe}
inside \texttt{src/main/java} and add
the following \texttt{TicTacToe.java} source inside that package:
%
\begin{codebox}\begin{javalst}
package io.hotmoka.tutorial.examples.tictactoe;

import static io.takamaka.code.lang.Takamaka.require;

import java.math.BigInteger;

import io.takamaka.code.lang.Contract;
import io.takamaka.code.lang.Exported;
import io.takamaka.code.lang.FromContract;
import io.takamaka.code.lang.Payable;
import io.takamaka.code.lang.PayableContract;
import io.takamaka.code.lang.Storage;
import io.takamaka.code.lang.StringSupport;
import io.takamaka.code.lang.View;
import io.takamaka.code.math.BigIntegerSupport;
import io.takamaka.code.util.StorageTreeArray;

public class TicTacToe extends Contract {

  @Exported
  public class Tile extends Storage {
    private final char c;

    private Tile(char c) {
      this.c = c;
    }

    @Override
    public String toString() {
      return String.valueOf(c);
    }

    private Tile nextTurn() {
      return this == CROSS ? CIRCLE : CROSS;
    }
  }

  private final Tile EMPTY = new Tile(' ');
  private final Tile CROSS = new Tile('X');
  private final Tile CIRCLE = new Tile('O');

  private final StorageTreeArray<Tile> board = new StorageTreeArray<>(9, EMPTY);
  private PayableContract crossPlayer;
  private PayableContract circlePlayer;
  private Tile turn = CROSS; // cross plays first
  private boolean gameOver;

  public @View Tile at(int x, int y) {
    require(1 <= x && x <= 3 && 1 <= y && y <= 3, "coordinates must be between 1 and 3");
    return board.get((y - 1) * 3 + x - 1);
  }

  private void set(int x, int y, Tile tile) {
    board.set((y - 1) * 3 + x - 1, tile);
  }

  public @Payable @FromContract(PayableContract.class) void play(long amount, int x, int y) {
    require(!gameOver, "the game is over");
    require(1 <= x && x <= 3 && 1 <= y && y <= 3, "coordinates must be between 1 and 3");
    require(at(x, y) == EMPTY, "the selected tile is not empty");

    PayableContract player = (PayableContract) caller();

    if (turn == CROSS)
      if (crossPlayer == null)
        crossPlayer = player;
      else
        require(player == crossPlayer, "it's not your turn");
    else
      if (circlePlayer == null) {
        require(crossPlayer != player, "you cannot play against yourself");
        long previousBet = BigIntegerSupport.subtract
          (balance(), BigInteger.valueOf(amount)).longValue();
        require(amount >= previousBet,
          () -> StringSupport.concat("you must bet at least ", previousBet, " coins"));
        circlePlayer = player;
      }
      else
        require(player == circlePlayer, "it's not your turn");

    set(x, y, turn);
    if (isGameOver(x, y))
      player.receive(balance());
    else
      turn = turn.nextTurn();
  }

  private boolean isGameOver(int x, int y) {
    if (at(x, 1) == turn && at(x, 2) == turn && at(x, 3) == turn) // column x
      return gameOver = true;

    if (at(1, y) == turn && at(2, y) == turn && at(3, y) == turn) // row y
      return gameOver = true;

    if (x == y && at(1, 1) == turn && at (2, 2) == turn && at(3, 3) == turn) // first diagonal
      return gameOver = true;

    if (x + y == 4 && at(1, 3) == turn && at(2, 2) == turn && at(3, 1) == turn) // second diagonal
      return gameOver = true;

    return gameOver = false;
  }

  @Override
  public @View String toString() {
    return StringSupport.concat(at(1, 1), "|", at(2, 1), "|", at(3, 1),
      "\n-----\n", at(1, 2), "|", at(2, 2), "|", at(3, 2),
      "\n-----\n", at(1, 3), "|", at(2, 3), "|", at(3, 3));
  }
}
\end{javalst}\end{codebox}

The internal class \texttt{Tile} represents the three alternatives that can be
put in the tic-tac-toe board. It overrides the default
\texttt{toString()} implementation, to yield the
usual representation for such alternatives; its \texttt{nextTurn()} method
alternates between cross and circle.

The board of the game is represented as a \texttt{new StorageTreeArray<>(9, EMPTY)}, whose
elements are indexed from zero to eight (inclusive) and are initialized to \texttt{EMPTY}.
It is also possible to construct the array as \texttt{new StorageTreeArray<>(9)}, but then
its elements would hold the default value \texttt{null} and the array would need to be initialized
inside a constructor for \texttt{TicTacToe}.

Methods \texttt{at()} and \texttt{set()} read and set the board element
at indexes $(x,y)$, respectively. They transform the two-dimensional conceptual representation
of the board into its internal one-dimensional representation. Since \texttt{at()} is public,
we defensively check the validity of the indexes there.

Method \texttt{play()} is the heart of the contract. Being called by the accounts
that play the game, it is annotated as \texttt{@FromContract}. It is also annotated as
\texttt{@Payable(PayableContract.class)} since players must bet money for
taking part in the game, at least for the first two moves, and receive
money if they win. The first
contract that plays is registered as \texttt{crossPlayer}. The second contract
that plays is registered as \texttt{circlePlayer}. Subsequent moves must
come, alternately, from \texttt{crossPlayer} and \texttt{circlePlayer}. The contract
uses a \texttt{turn} variable to keep track of the current turn.

Note the extensive use of \texttt{require()} to check all error situations:
%
\begin{enumerate}
\item It is possible to play only if the game is not over yet.
\item A move must be inside the board and identify an empty tile.
\item Players must alternate correctly.
\item The second player must bet at least as much as the first player.
\item It is not allowed to play against oneself.
\end{enumerate}

The \texttt{play()} method ends with a call to \texttt{gameOver()} that checks
if the game is over, that is, if the current player won.
In that case, the winner receives the full
jackpot. Note that the \texttt{gameOver()} method receives the coordinates
where the current player has moved. This allows it to restrict the
check for game over: the game is over only if the row or column
where the player moved contain the same tile; if the current player
played on a diagonal, the method checks the diagonals as well.
It is of course possible to check all rows, columns and diagonals, always,
but our solution is gas-thriftier.

The \texttt{toString()} method yields a string representation of the current board, such as
%
\begin{alltt}
X|O| 
-----
 |X|O
-----
 |X| 
\end{alltt}

\subsection{A more realistic tic-tac-toe contract}\label{subsec:more_realistic_tictactoe}

\begin{center}
(See the \texttt{io-hotmoka-tutorial-examples-tictactoe\_revised} project in \texttt{\hotmokaRepo{}})
\end{center}

The \texttt{TicTacToe.java} code implements the rules of a tic-tac-toe game, but has
a couple of drawbacks that make it still incomplete. Namely:
%
\begin{enumerate}
\item The creator of the game must spend gas to call its constructor,
  but has no direct incentive in doing so. He must be a benefactor,
  or hope to take part in the game after creation, if he is faster than
  any other potential player.
\item If the game ends in a draw, money gets stuck in the \texttt{TicTacToe} contract
  instance, for ever and ever.
\end{enumerate}

Replace hence the previous version of \texttt{TicTacToe.java} with the following
revised version. This new version solves
both problems at once. The policy is very simple: it imposes a minimum
bet, in order to avoid free games; if a winner emerges,
then the game forwards him only $90\%$ of the jackpot; the remaining $10\%$ goes to the
creator of the \texttt{TicTacToe} contract itself. If, instead, the game ends in a draw,
it forwards the whole jackpot to the creator.
Note that we added a \texttt{@FromContract} constructor, that takes
note of the \texttt{creator} of the game:
%
\begin{codebox}\begin{javalst}
package io.hotmoka.tutorial.examples.tictactoe;

import static io.takamaka.code.lang.Takamaka.require;

import java.math.BigInteger;

import io.takamaka.code.lang.Contract;
import io.takamaka.code.lang.Exported;
import io.takamaka.code.lang.FromContract;
import io.takamaka.code.lang.Payable;
import io.takamaka.code.lang.PayableContract;
import io.takamaka.code.lang.Storage;
import io.takamaka.code.lang.StringSupport;
import io.takamaka.code.lang.View;
import io.takamaka.code.math.BigIntegerSupport;
import io.takamaka.code.util.StorageTreeArray;

public class TicTacToe extends Contract {

  @Exported
  public class Tile extends Storage {
    private final char c;

    private Tile(char c) {
      this.c = c;
    }

    @Override
    public String toString() {
      return String.valueOf(c);
    }

    private Tile nextTurn() {
      return this == CROSS ? CIRCLE : CROSS;
    }
  }

  private final Tile EMPTY = new Tile(' ');
  private final Tile CROSS = new Tile('X');
  private final Tile CIRCLE = new Tile('O');

  private static final long MINIMUM_BET = 100L;

  private final StorageTreeArray<Tile> board = new StorageTreeArray<>(9, EMPTY);
  private final PayableContract creator;
  private PayableContract crossPlayer;
  private PayableContract circlePlayer;
  private Tile turn = CROSS; // cross plays first
  private boolean gameOver;

  public @FromContract(PayableContract.class) TicTacToe() {
    creator = (PayableContract) caller();
  }

  public @View Tile at(int x, int y) {
    require(1 <= x && x <= 3 && 1 <= y && y <= 3, "coordinates must be between 1 and 3");
    return board.get((y - 1) * 3 + x - 1);
  }

  private void set(int x, int y, Tile tile) {
    board.set((y - 1) * 3 + x - 1, tile);
  }

  public @Payable @FromContract(PayableContract.class) void play(long amount, int x, int y) {
    require(!gameOver, "the game is over");
    require(1 <= x && x <= 3 && 1 <= y && y <= 3, "coordinates must be between 1 and 3");
    require(at(x, y) == EMPTY, "the selected tile is not empty");

    PayableContract player = (PayableContract) caller();

    if (turn == CROSS)
      if (crossPlayer == null) {
        require(amount >= MINIMUM_BET, () -> "you must invest at least " + MINIMUM_BET + " coins");
         crossPlayer = player;
      }
      else
        require(player == crossPlayer, "it's not your turn");
    else
      if (circlePlayer == null) {
        require(crossPlayer != player, "you cannot play against yourself");
        long previousBet = BigIntegerSupport.subtract
          (balance(), BigInteger.valueOf(amount)).longValue();
        require(amount >= previousBet,
          () -> StringSupport.concat("you must bet at least ", previousBet, " coins"));
        circlePlayer = player;
      }
      else
        require(player == circlePlayer, "it's not your turn");

    set(x, y, turn);
    if (isGameOver(x, y)) {
      // 90% goes to the winner
      player.receive(BigIntegerSupport.divide
        (BigIntegerSupport.multiply(balance(), BigInteger.valueOf(9L)), BigInteger.valueOf(10L)));
      // the rest to the creator of the game
      creator.receive(balance());
    }
    else if (isDraw())
      // everything goes to the creator of the game
      creator.receive(balance());
    else
      turn = turn.nextTurn();
  }

  private boolean isGameOver(int x, int y) {
    if (at(x, 1) == turn && at(x, 2) == turn && at(x, 3) == turn) // column x
      return gameOver = true;

    if (at(1, y) == turn && at(2, y) == turn && at(3, y) == turn) // row y
      return gameOver = true;

    if (x == y && at(1, 1) == turn && at (2, 2) == turn && at(3, 3) == turn) // first diagonal
      return gameOver = true;

    if (x + y == 4 && at(1, 3) == turn && at(2, 2) == turn && at(3, 1) == turn) // second diagonal
      return gameOver = true;

    return gameOver = false;
  }

  private boolean isDraw() {
    for (var tile: board)
      if (tile == EMPTY)
        return false;

    return true;
  }

  @Override
  public @View String toString() {
    return StringSupport.concat(at(1, 1), "|", at(2, 1), "|", at(3, 1),
      "\n-----\n", at(1, 2), "|", at(2, 2), "|", at(3, 2),
      "\n-----\n", at(1, 3), "|", at(2, 3), "|", at(3, 3));
  }
}    
\end{javalst}\end{codebox}

\begin{commentbox}
  We have chosen to allow a \texttt{long amount} in the \texttt{@Payable}
  method \texttt{play()} since
  it is unlikely that users will want to invest huge quantities of money in this
  game. This gives us the opportunity to discuss why the computation of the
  previous bet has been written as
  \begin{center}\begin{alltt}
    long previousBet = BigIntegerSupport.subtract
      (balance(), BigInteger.valueOf(amount)).longValue()
  \end{alltt}\end{center}
  instead of the simpler
  \texttt{long previousBet = balance().longValue() - amount}.
  The reason is that, when that line is executed, both players have already paid
  their bet, that accumulates in the balance of the \texttt{TicTacToe} contract.
  Each single bet is a \texttt{long}, but their sum could overflow the size of a \texttt{long}.
  Hence, we have to deal with a computation on \texttt{BigInteger}. The same situation
  occurs later, when we have to compute the $90\%$ that goes to the winner:
  the jackpot might be larger than a \texttt{long} and we have to compute over
  \texttt{BigInteger}. As a final remark, note that in the line:
  \begin{center}\begin{alltt}
      BigIntegerSupport.divide(BigIntegerSupport.multiply
        (balance(), BigInteger.valueOf(9L)), BigInteger.valueOf(10L))
  \end{alltt}\end{center}
  we \emph{first multiply} by $9$ and \emph{then divide} by $10$. This reduces the
  approximation inherent to integer division. For instance, if the jackpot
  (\texttt{balance()}) were 209, we have (with Java's left-to-right evaluation)
  $209\cdot 9/10=1881/10=188$ while $209/10\cdot 9=20\cdot 9=180$.
\end{commentbox}

\subsection{Running the tic-tac-toe contract}\label{subsec:running_tictactoe}

Let us play with the \texttt{TicTacToe} contract. Go in
the \texttt{io-hotmoka-tutorial-examples-tictactoe} project,
compile it with Maven and store it in the Hotmoka node:
%
\index{moka!jars install@{\texttt{jars install}}}
\inputCommand{moka_jars_install_tictactoe_revised}
\inputOutput{moka_jars_install_tictactoe_revised}
%
Then we create an instance of the contract in the node:
%
\index{moka!objects create@{\texttt{objects create}}}
\inputCommand{moka_objects_create_tictactoe_revised}
\inputOutput{moka_objects_create_tictactoe_revised}

We use two of our accounts now, that we have already created in the previous section,
to interact with the contract: they will play, alternately, until the first player wins.
We will report the resulting of calling \texttt{toString()} on the contract, after each move.

The first player starts, by playing at (1,1), and bets 100:
%
\index{moka!objects call@{\texttt{objects call}}}
\inputCommand{moka_objects_call_tictactoe_play_1}
\inputOutput{moka_objects_call_tictactoe_play_1}
%
\inputCommand{moka_objects_call_tictactoe_toString_1}
\inputOutput{moka_objects_call_tictactoe_toString_1}
%
Note that the call to \texttt{toString()} does not require to provide the password of the key pair of the caller account,
since that method is a \texttt{@View} method: this means that moka runs a transaction to call it, it does not add a transaction.

The second player plays now, at (2,1), betting 100:
%
\inputCommand{moka_objects_call_tictactoe_play_2}
\inputOutput{moka_objects_call_tictactoe_play_2}
%
\inputCommand{moka_objects_call_tictactoe_toString_2}
\inputOutput{moka_objects_call_tictactoe_toString_2}
%
The first player replies, by playing at (1,2):
%
\inputCommand{moka_objects_call_tictactoe_play_3}
\inputOutput{moka_objects_call_tictactoe_play_3}
%
\inputCommand{moka_objects_call_tictactoe_toString_3}
\inputOutput{moka_objects_call_tictactoe_toString_3}
%
Then the second player plays at (2,2):
%
\inputCommand{moka_objects_call_tictactoe_play_4}
\inputOutput{moka_objects_call_tictactoe_play_4}
%
\inputCommand{moka_objects_call_tictactoe_toString_4}
\inputOutput{moka_objects_call_tictactoe_toString_4}
%
The first player wins by playing at (1,3):
%
\inputCommand{moka_objects_call_tictactoe_play_5}
\inputOutput{moka_objects_call_tictactoe_play_5}
%
\inputCommand{moka_objects_call_tictactoe_toString_5}
\inputOutput{moka_objects_call_tictactoe_toString_5}
%
We can verify that the game is over now:
%
\index{moka!objects show@{\texttt{objects show}}}
\inputCommand{moka_objects_show_tictactoe}
\inputOutput{moka_objects_show_tictactoe}
%
As you can see, the \texttt{gameOver} field holds true. Moreover, the balance of the contract is zero since it has been distributed to
the winner and to the creator of the game (that actually coincide to our first account, in this specific run).

If the second player attempts to play now, the transaction will be rejected, since the game is over:
%
\inputCommand{moka_objects_call_tictactoe_play_6}
\inputOutput{moka_objects_call_tictactoe_play_6}

\subsection{Specialized storage array classes}\label{subsec:specialized_storage_array_classes}

The \texttt{StorageTreeArray<E>} class is very general, since it can be used to hold
any type \texttt{E} of storage values. Since it uses generics,
primitive values cannot be held in a \texttt{StorageTreeArray<E>}, directly.
For instance, \texttt{StorageTreeArray<byte>} is not legal syntax in Java.
Instead, one could think to use \texttt{StorageTreeArray<Byte>}, where \texttt{Byte}
is the Java wrapper class \texttt{java.lang.Byte}. However, that class is not
currently allowed in storage, hence \texttt{StorageTreeArray<Byte>} will not work either.
One should hence define a new wrapper class for \texttt{byte}, that extends \texttt{Storage}.
That is possible, but highly discouraged:
the use of wrapper classes introduces a level of indirection
and requires the instantiation of many small objects, which costs gas. Instead,
Takamaka provides specialized storage classes implementing arrays of bytes,
without wrappers. The rationale is that such arrays arise
naturally when dealing, for instance, with hashes or encrypted data
(see next section for an example) and consequently deserve
a specialized and optimized implementation.
Such specialized array classes
can have their length specified at construction time, or fixed to
a constant (for best optimization and minimal gas consumption).

\begin{figure}[th]
  \begin{center}
    \myincludegraphics{0.8\textwidth}{bytes}
  \end{center}
  \caption{The hierarchy of specialized byte array classes.}
  \label{fig:byte_array_hierarchy}
\end{figure}

Fig.~\ref{fig:byte_array_hierarchy} shows the hierarchy of the specialized classes for arrays of bytes,
available in Takamaka.
The interface \texttt{StorageByteArrayView}\index{StorageByteArrayView@{\texttt{StorageByteArrayView}}}
defines the methods that read data from an array
of bytes, while the interface \texttt{StorageByteArray}\index{StorageByteArray@{\texttt{StorageByteArray}}}
defines the modification methods.
Class \texttt{StorageTreeByteArray}\index{StorageTreeByteArray@{\texttt{StorageTreeByteArray}}}
allows one to create byte arrays of any length, specified at construction time.
Classes \texttt{Bytes32}\index{Bytes32@{\texttt{Bytes32}}} and
\texttt{Bytes32Snapshot}\index{Bytes32Snapshot@{\texttt{Bytes32Snapshot}}}
have, instead, fixed length of $32$ bytes;
their constructors include one that allows one to specify such $32$ bytes,
which is useful for calling the constructor from outside the node,
since \texttt{byte} is a storage type.
While a \texttt{Bytes32} is modifiable, instances of class \texttt{Bytes32Snapshot}
are not modifiable after being created and are \texttt{@Exported}.
There are sibling classes for different, fixed sizes, such as
\texttt{Bytes64} and \texttt{Bytes8Snaphot}. For a full description of the methods
of these classes and interfaces, we refer to their JavaDoc.

\section{Storage maps}\label{sec:storage_maps}

Maps are dynamic associations of objects to objects. They are useful
for programming smart contracts, as their extensive use in Solidity proves.
However, most such uses are related to the withdrawal pattern, that is
not needed in Takamaka. Nevertheless, there are still situations when
maps are useful in Takamaka code, as we show below.

Java has many implementations of maps.
However, they are not storage objects and consequently cannot be
stored in a Hotmoka node. This section describes the Takamaka library classes
\texttt{io.takamaka.code.util.StorageTreeMap<K,V>}\index{StorageTreeMap@{\texttt{StorageTreeMap}}} and
\texttt{SnapshottableStorageTreeMap<K,V>}\index{SnapshottableStorageTreeMap@{\texttt{SnapshottableStorageTreeMap}}},
that extend \texttt{Storage} and
whose instances can then be held in the store of a node, if
keys \texttt{K} and values \texttt{V} can be stored in a node as well.

\begin{figure}[th]
  \begin{center}
	\begin{tikzpicture}[scale=1]
	\scriptsize
	\externalclasscolor
	\begin{interface}[text width=4cm]{Iterable\string<Entry\string<K V\string>{}\string>}{0,0}
	\end{interface}
	\interfacecolor
	\begin{interface}[text width=9cm]{StorageMapView\string<K V\string>}{0,-1.5}
		\inherit{Iterable\string<Entry\string<K V\string>{}\string>}
		\operation{@View size():int}
		\operation{@View isEmpty():boolean}
		\operation{@View get(key:Object):V}
		\operation{@View getOrDefault(key:Object, \string_default:V):V}
		\operation{getOrDefault(key:Object, \string_default:Supplier\string<? extends V\string>):V}
		\operation{@View containsKey(key:Object):boolean}
		\operation{@View min(), max():K}
		\operation{@View floorKey(key:K), ceilingKey(key:K):K}
		\operation{@View select(k:int):K}
		\operation{forEach(action:Consumer\string<? super Entry\string<K,V\string>{}\string>)}
		\operation{forEachKey(action:Consumer\string<? super K\string>)}
		\operation{forEachValue(action:Consumer\string<? super V\string>)}
	\end{interface}
	\begin{interface}[text width=6cm]{SnapshottableStorageMapView\string<K V\string>}{4.5,-7}
		\inherit{StorageMapView\string<K V\string>}
		\operation{snapshot():StorageMapView\string<K,V\string>}
	\end{interface}
	\begin{interface}[text width=9.7cm]{StorageMap\string<K V\string>}{-4,-7}
		\inherit{StorageMapView\string<K V\string>}
		\operation{put(key:K, value:V)}
		\operation{putIfAbsent(key:K, value:V):V}
		\operation{computeIfAbsent(key:K, supplier:Function\string<? super K, ? extends V\string>):V}
		\operation{removeMin(), removeMax()}
		\operation{remove(key:Object)}
		\operation{update(key:K, how:UnaryOperator\string<V\string>)}
		\operation{clear()}
		\operation{view():StorageMapView\string<K,V\string>}
	\end{interface}
	\begin{interface}[text width=6cm]{SnapshottableStorageMap\string<K V\string>}{4.5,-10}
		\inherit{StorageMap\string<K V\string>}
		\inherit{SnapshottableStorageMapView\string<K V\string>}
		\operation{view():SnapshottableStorageMapView\string<K,V\string>}
	\end{interface}
	\externalclasscolor
	\begin{class}[text width=2cm]{Storage}{-1,-11}
	\end{class}
	\classcolor
	\begin{class}[text width=6cm]{StorageTreeMap\string<K V\string>}{-5,-12}
		\inherit{Storage}
		\implement{StorageMap\string<K V\string>}
		\operation{StorageTreeMap()}
	\end{class}
	\begin{class}[text width=7.5cm]{SnapshottableStorageTreeMap\string<K V\string>}{3,-12}
		\inherit{Storage}
		\implement{SnapshottableStorageMap\string<K V\string>}
		\operation{SnapshottableStorageTreeMap()}
	\end{class}
	\end{tikzpicture}
  \end{center}
  \caption{The hierarchy of storage maps.}
  \label{fig:maps_hierarchy}
\end{figure}

We refer to the JavaDoc of \texttt{StorageTreeMap} and \texttt{SnapshottableStorageTreeMap}
for a full description of their methods,
that are similar to those of traditional Java maps. Here, we just observe
that a key is mapped into a value by calling method
\texttt{void put(K key, V value)}, while the value bound to a key is retrieved by calling
\texttt{V get(Object key)}. It is possible to yield a default value when a key is not
in the map, by calling \texttt{V getOrDefault(Object key, V \_default)} or
its sibling \texttt{V getOrDefault(Object key, Supplier<? extends V> \_default)}, that
evaluates the default value only if needed. Method \texttt{V putIfAbsent(K key, V value)}
binds the key to the value only if the key is unbound. Similarly for
its sibling \texttt{V computeIfAbsent(K key, Function<? super K, ? extends V> value)} that computes
the new value only if needed (these two methods differ for their
returned value, as in Java maps. Please refer to their JavaDoc).

Instances of \texttt{StorageTreeMap<K,V>} and \texttt{SnapshottableStorageTreeMap<K,V>}
keep keys in increasing order. Namely, if
type \texttt{K} has a natural order, that order is used. Otherwise, keys
(that must be storage objects) are kept ordered by increasing storage
reference. Consequently, methods \texttt{forEach(Consumer<? super Entry<K,V>{}> action)},
\texttt{forEachKey(Consumer<? super K> action)} and
\texttt{forEachValue(Consumer<? super V> action)}
perform an internal iteration of the elements of the map, in order.
%
\begin{commentbox}
Compare this with Solidity, where maps do not know the set of their keys nor the
set of their values, so that, in Solidity, it is impossible to iterate on maps.
\end{commentbox}

Fig.~\ref{fig:maps_hierarchy} shows the hierarchy of the storage map classes.
They implement the library interface \texttt{StorageMap<K,V>}\index{StorageMap@{\texttt{StorageMap}}},
that defines the methods that modify a map.
That interface extends the interface \texttt{StorageMapView<K,V>}\index{StorageMapView@{\texttt{StorageMapView}}}
that, instead, defines the methods that read data from a map, but do not modify it.
Methods \texttt{snapshot()}\index{snapshot()@{\texttt{snapshot()}}}
and \texttt{view()}\index{view()@{\texttt{view()}}} return an \texttt{@Exported}
\texttt{StorageMapView<K,V>}, in constant time.
Only instances of class \texttt{SnapshottableStorageTreeMap<K,V>} allow the creation of snapshots.
Because of that, they are slightly more expensive, in time, space and gas, than instances of
class \texttt{StorageTreeMap<K,V>}. Therefore, use always a \texttt{StorageMapView<K,V>} whenever
snapshots are not needed.

There are also specialized map classes, optimized
for specific primitive types of keys, such as
\texttt{StorageTreeIntMap<V>}\index{StorageTreeIntMap@{\texttt{StorageTreeIntMap}}},
whose keys are \texttt{int} values. We refer to their JavaDoc for further information.

Next section shows an example of use for \texttt{StorageTreeMap<K,V>}.

\subsection{A blind auction contract}\label{subsec:blind_auction}

\begin{center}
(See the \texttt{io-hotmoka-tutorial-examples-auction} project in \texttt{\hotmokaRepo{}})
\end{center}

This section exemplifies the use of class \texttt{StorageTreeMap} by writing a smart
contract that implements a \emph{blind auction}. That contract allows
a \emph{beneficiary} to sell an item to the buying contract that offers
the highest bid. Since data in blockchain is public, in a non-blind
auction it is possible that bidders eavesdrop the offers of other bidders
in order to place an offer that is only slightly higher than the current
best offer. A blind auction, instead, uses a two-phases
mechanism: in the initial \emph{bidding time}, bidders place bids, hashed, so that
they do not reveal their amount. After the bidding time expires, the second
phase, called \emph{reveal time}, allows bidders to
reveal the real values of their bids and the auction contract to determine
the actual winner.
This works since, to reveal a bid, each bidder provides the real data
of the bid. The auction contract then recomputes the hash from real data and
checks if the result matches the hash provided at bidding time.
If not, the bid is considered invalid. Bidders can even place fake offers
on purpose, in order to confuse other bidders.

Create in Eclipse a new Maven Java 21 (or later) project. Use for this project the name
\texttt{io-hotmoka-tutorial-examples-auction}.
You could do this for instance by duplicating the project \texttt{io-hotmoka-tutorial-examples-family}.
Use the following \texttt{pom.xml}:
%
\begin{codebox}\begin{xmllst}
<project xmlns="http://maven.apache.org/POM/4.0.0"
    xmlns:xsi="http://www.w3.org/2001/XMLSchema-instance"
    xsi:schemaLocation="http://maven.apache.org/POM/4.0.0 http://maven.apache.org/xsd/maven-4.0.0.xsd">

  <modelVersion>4.0.0</modelVersion>
  <groupId>io.hotmoka</groupId>
  <artifactId>io-hotmoka-tutorial-examples-auction</artifactId>
  <version>|\hotmokaVersion{}|</version>

  <properties>
    <project.build.sourceEncoding>UTF-8</project.build.sourceEncoding>
    <maven.compiler.release>21</maven.compiler.release>
  </properties>

  <dependencies>
    <dependency>
      <groupId>io.hotmoka</groupId>
      <artifactId>io-takamaka-code</artifactId>
      <version>|\takamakaVersion{}|</version>
    </dependency>
  </dependencies>

  <build>
    <plugins>
      <plugin>
        <groupId>org.apache.maven.plugins</groupId>
        <artifactId>maven-compiler-plugin</artifactId>
        <version>3.11.0</version>
      </plugin>
    </plugins>
  </build>

</project>
\end{xmllst}\end{codebox}
%
and the following \texttt{module-info.java}:
%
\begin{codebox}\begin{javalst}
module auction {
  requires io.takamaka.code;
}
\end{javalst}\end{codebox}

Create package \texttt{io.hotmoka.tutorial.examples.auction}
inside \texttt{src/main/java} and add
the following \texttt{BlindAuction.java} inside that package.
It is a Takamaka contract that implements
a blind auction. Since each bidder may place more bids and since such bids
must be kept in storage until reveal time, this code uses a map
from bidders to lists of bids. This smart contract has been inspired
by a similar Ethereum contract in Solidity available at
\url{https://docs.soliditylang.org/en/v0.8.33/solidity-by-example.html#blind-auction}.
Please note that the code below does not compile yet, since it misses two classes
that we will define in the next section.

%
\begin{codebox}\begin{javalst}
package io.hotmoka.tutorial.examples.auction;

import static io.takamaka.code.lang.Takamaka.event;
import static io.takamaka.code.lang.Takamaka.now;
import static io.takamaka.code.lang.Takamaka.require;

import java.math.BigInteger;
import java.security.NoSuchAlgorithmException;
import java.util.function.Supplier;

import io.takamaka.code.lang.Contract;
import io.takamaka.code.lang.Exported;
import io.takamaka.code.lang.FromContract;
import io.takamaka.code.lang.Payable;
import io.takamaka.code.lang.PayableContract;
import io.takamaka.code.lang.Storage;
import io.takamaka.code.lang.StringSupport;
import io.takamaka.code.math.BigIntegerSupport;
import io.takamaka.code.security.SHA256Digest;
import io.takamaka.code.util.Bytes32Snapshot;
import io.takamaka.code.util.StorageLinkedList;
import io.takamaka.code.util.StorageList;
import io.takamaka.code.util.StorageMap;
import io.takamaka.code.util.StorageTreeMap;

/**
* A contract for a simple auction. This class is derived from the Solidity
* code shown at https://docs.soliditylang.org/en/v0.8.33/
* solidity-by-example.html#blind-auction
* In this contract, bidders place bids together with a hash. At the end of
* the bidding period, bidders are expected to reveal if and which of their
* bids were real and their actual value. Fake bids are refunded. Real bids
* are compared and the bidder with the highest bid wins.
*/
public class BlindAuction extends Contract {

  /**
  * A bid placed by a bidder. The deposit has been payed in full.
  * If, later, the bid will be revealed as fake, then the deposit will
  * be fully refunded. If, instead, the bid will be revealed as real, but
  * for a lower amount, then only the difference will be refunded.
  */
  private static class Bid extends Storage {

    /**
    * The hash that will be regenerated and compared at reveal time.
    */
    private final Bytes32Snapshot hash;

    /**
    * The value of the bid. Its real value might be lower and known
    * at real time only.
    */
    private final BigInteger deposit;

    private Bid(Bytes32Snapshot hash, BigInteger deposit) {
      this.hash = hash;
      this.deposit = deposit;
    }

    /**
    * Recomputes the hash of a bid at reveal time and compares it
    * against the hash provided at bidding time. If they match,
    * we can reasonably trust the bid.
    * 
    * @param revealed the revealed bid
    * @param digest the hasher
    * @return true if and only if the hashes match
    */
    private boolean matches(RevealedBid revealed, SHA256Digest digest) {
      digest.update(BigIntegerSupport.toByteArray(revealed.value));
      digest.update(revealed.fake ? (byte) 0 : (byte) 1);
      digest.update(revealed.salt.toArray());
      byte[] arr1 = hash.toArray();
      byte[] arr2 = digest.digest();

      if (arr1.length != arr2.length)
        return false;

      for (int pos = 0; pos < arr1.length; pos++)
        if (arr1[pos] != arr2[pos])
          return false;

      return true;
    }
  }

  /**
  * A bid revealed by a bidder at reveal time. The bidder shows
  * if the corresponding bid was fake or real, and how much was the
  * actual value of the bid. This might be lower than previously
  * communicated.
  */
  @Exported
  public static class RevealedBid extends Storage {
    private final BigInteger value;
    private final boolean fake;

    /**
    * The salt used to strengthen the hashing.
    */
    private final Bytes32Snapshot salt;

    public RevealedBid(BigInteger value, boolean fake, Bytes32Snapshot salt) {
      this.value = value;
      this.fake = fake;
      this.salt = salt;
    }
  }

  /**
  * The beneficiary that, at the end of the reveal time, will receive
  * the highest bid.
  */
  private final PayableContract beneficiary;

  /**
  * The bids for each bidder. A bidder might place more bids.
  */
  private final StorageMap<PayableContract, StorageList<Bid>> bids = new StorageTreeMap<>();

  /**
  * The time when the bidding time ends.
  */
  private final long biddingEnd;

  /**
  * The time when the reveal time ends.
  */
  private final long revealEnd;

  /**
  * The bidder with the highest bid, at reveal time.
  */
  private PayableContract highestBidder;

  /**
  * The highest bid, at reveal time.
  */
  private BigInteger highestBid;

  /**
  * Creates a blind auction contract.
  * 
  * @param biddingTime the length of the bidding time
  * @param revealTime the length of the reveal time
  */
  public @FromContract(PayableContract.class) BlindAuction(int biddingTime, int revealTime) {
    require(biddingTime > 0, "Bidding time must be positive");
    require(revealTime > 0, "Reveal time must be positive");

    this.beneficiary = (PayableContract) caller();
    this.biddingEnd = now() + biddingTime;
    this.revealEnd = biddingEnd + revealTime;
  }

  /**
  * Places a blinded bid the given hash.
  * The sent money is only refunded if the bid is correctly
  * revealed in the revealing phase. The bid is valid if the
  * money sent together with the bid is at least "value" and
  * "fake" is not true. Setting "fake" to true and sending
  * not the exact amount are ways to hide the real bid but
  * still make the required deposit. The same bidder can place multiple bids.
  */
  public @Payable @FromContract(PayableContract.class) void bid(BigInteger amount, Bytes32Snapshot hash) {
    onlyBefore(biddingEnd);
    bids.computeIfAbsent((PayableContract) caller(),
     (Supplier<? extends StorageList<Bid>>) StorageLinkedList::new).add(new Bid(hash, amount));
  }

  /**
  * Reveals a bid of the caller. The caller will get a refund for all
  * correctly blinded invalid bids and for all bids except
  * for the totally highest.
  * 
  * @param revealed the revealed bid
  * @throws NoSuchAlgorithmException if the hashing algorithm is not available
  */
  public @FromContract(PayableContract.class) void reveal(RevealedBid revealed)
      throws NoSuchAlgorithmException {
    onlyAfter(biddingEnd);
    onlyBefore(revealEnd);
    PayableContract bidder = (PayableContract) caller();
    StorageList<Bid> bids = this.bids.get(bidder);
    require(bids != null && bids.size() > 0, "No bids to reveal");
    require(revealed != null, () -> "The revealed bid cannot be null");

    // any other hashing algorithm will do, as long as both
    // bidder and auction contracts use the same
    var digest = new SHA256Digest();
    // by removing the head of the list, it makes it impossible for the caller
    // to re-claim the same deposits
    bidder.receive(refundFor(bidder, bids.removeFirst(), revealed, digest));
  }

  public PayableContract auctionEnd() {
    onlyAfter(revealEnd);
    PayableContract winner = highestBidder;
		
    if (winner != null) {
      beneficiary.receive(highestBid);
      event(new AuctionEnd(winner, highestBid));
      highestBidder = null;
    }

    return winner;
  }

  /**
  * Checks how much of the deposit should be refunded for a given bid.
  * 
  * @param bidder the bidder that placed the bid
  * @param bid the bid, as was placed at bidding time
  * @param revealed the bid, as was revealed later
  * @param digest the hashing algorithm
  * @return the amount to refund
  */
  private BigInteger refundFor(PayableContract bidder, Bid bid, RevealedBid revealed,
                               SHA256Digest digest) {
    if (!bid.matches(revealed, digest))
      // the bid was not actually revealed: no refund
      return BigInteger.ZERO;
    else if (!revealed.fake && BigIntegerSupport.compareTo(bid.deposit, revealed.value) >= 0
             && placeBid(bidder, revealed.value))
      // the bid was correctly revealed and is the best up to now:
      // only the difference between promised and provided is refunded;
      // the rest might be refunded later if a better bid will be revealed
      return BigIntegerSupport.subtract(bid.deposit, revealed.value);
    else
      // the bid was correctly revealed and is not the best one:
      // it is fully refunded
      return bid.deposit;
  }

  /**
  * Takes note that a bidder has correctly revealed a bid for the given value.
  * 
  * @param bidder the bidder
  * @param value the value, as revealed
  * @return true if and only if this is the best bid, up to now
  */
  private boolean placeBid(PayableContract bidder, BigInteger value) {
    if (highestBid != null && BigIntegerSupport.compareTo(value, highestBid) <= 0)
      // this is not the best bid seen so far
      return false;

    // if there was a best bidder already, its bid is refunded
    if (highestBidder != null)
      // refund the previously highest bidder
      highestBidder.receive(highestBid);

    // take note that this is the best bid up to now
    highestBid = value;
    highestBidder = bidder;
    event(new BidIncrease(bidder, value));

    return true;
  }

  private static void onlyBefore(long when) {
    long diff = now() - when;
    require(diff <= 0, StringSupport.concat(diff, " ms too late"));
  }

  private static void onlyAfter(long when) {
    long diff = now() - when;
    require(diff >= 0, StringSupport.concat(-diff, " ms too early"));
  }
}
\end{javalst}\end{codebox}

Let us discuss this (long) code, by starting from the inner classes.

Class \texttt{Bid} represents a bid placed by a contract that takes part in the auction.
This information will be stored in blockchain at bidding time, hence
it is known to all other participants. An instance of \texttt{Bid} contains
the \texttt{deposit} paid at time of placing the bid. This is not necessarily
the real value of the offer but must be at least as large as the real offer,
or otherwise the bid will be considered as invalid and rejected at reveal time. Instances
of \texttt{Bid} contain a \texttt{hash} consisting of $32$ bytes. As already said, this will
be recomputed at reveal time and matched against the result.
Since arrays cannot be stored in blockchain, we use the storage class
\texttt{io.takamaka.code.util.Bytes32Snapshot} here, a library class that holds $32$ bytes, as a
traditional array (see Sec.~\ref{subsec:specialized_storage_array_classes}).
It is well possible to use a \texttt{StorageArray} of a wrapper
of \texttt{byte} here, but \texttt{Bytes32Snapshot} is much more compact and its methods consume less gas.

Class \texttt{RevealedBid} describes a bid revealed after bidding time.
It contains the real value of the bid, the salt used to strengthen the
hashing algorithm and a boolean \texttt{fake} that, when true, means that the
bid must be considered as invalid, since it was only placed in order
to confuse other bidders. It is possible to recompute and check the hash of
a revealed bid through method \texttt{matches()}, that uses a given
hashing algorithm (\texttt{digest}, a Java \texttt{java.security.MessageDigest}) to
hash value, fake mark and salt into bytes, finally compared
against the hash provided at bidding time.

The \texttt{BlindAuction} contract stores the \texttt{beneficiary} of the auction.
It is the contract that created the auction and is consequently
initialized, in the constructor of \texttt{BlindAuction}, to its caller.
The constructor must be annotated as \texttt{@FromContract} because of that.
The same constructor receives the length of bidding time and reveal time, in
milliseconds. This allows the contract to compute the absolute ending time
for the bidding phase and for the reveal phase, stored into fields
\texttt{biddingEnd} and \texttt{revealEnd}, respectively.
Note, in the constructor of \texttt{BlindAuction}, the
use of the static method \texttt{io.takamaka.code.lang.Takamaka.now()}, that yields the
current time, as with the traditional \texttt{System.currentTimeMillis()} of Java
(that instead cannot be used in Takamaka code). Method \texttt{now()}, in a blockchain, yields the
time of creation of the block of the current transaction, as seen by its miner.
That time is reported in the block and hence is independent from the
machine that runs the contract, which guarantees determinism.

Method \texttt{bid()} allows a caller (the bidder) to place a bid during the bidding phase.
An instance of \texttt{Bid} is created and added to a list, specific to each
bidder. Here is where our map comes to help. Namely, field
\texttt{bids} holds a \texttt{StorageTreeMap<PayableContract, StorageList<Bid>>},
that can be held in the store of a node since it is a storage map between storage keys
and storage values. Method \texttt{bid()} computes an empty list of bids if it is the
first time that a bidder places a bid. For that, it uses method
\texttt{computeIfAbsent()} of \texttt{StorageMap}. If it used method \texttt{get()}, it would
run into a null-pointer exception the first time a bidder places a bid.
That is, storage maps default to \texttt{null}, as all Java maps. (But differently to
Solidity maps, that provide a default value automatically when undefined.)

Method \texttt{reveal()} is called by each bidder during the reveal phase.
It accesses the \texttt{bids} placed by the bidder during the bidding time.
The method matches each revealed bid against the corresponding
list of bids for the player, by calling
method \texttt{refundFor()}, that determines how much of the deposit must be
refunded to the bidder. Namely, if a bid was fake or was not the best bid,
it must be refunded in full. If it was the best bid, it must be partially refunded
if the apparent \texttt{deposit} turns out to be higher than the actual value of the
revealed bid. While bids are refunded, method \texttt{placeBid} updates
the best bid information.

Method \texttt{auctionEnd()} is meant to be called after the reveal phase.
If there is a winner, it sends the highest bid to the beneficiary.

Note the use of methods \texttt{onlyBefore()} and \texttt{onlyAfter()} to guarantee
that some methods are only run at the right moment.

\subsection{Events}\label{subsec:events}\index{event}

\begin{center}
(See the \texttt{io-hotmoka-tutorial-examples-auction\_events} project in \texttt{\hotmokaRepo{}})
\end{center}

The code in the previous section does not compile since it misses two
classes \texttt{BidIncrease.java} and \texttt{AuctionEnd.java}, that we report below.
Namely, the code of the blind auction contract contains some lines that generate
\emph{events}, such as:
%
\begin{codebox}\begin{javalst}
event(new AuctionEnd(winner, highestBid));
\end{javalst}\end{codebox}

Events are milestones that are saved in the store of a Hotmoka node.
From outside the node, it is possible to subscribe to specific events and get
notified as soon as an event of that kind occurs,
to trigger actions when that happens. In terms of the
Takamaka language, events are generated through the
\texttt{io.takamaka.code.lang.Takamaka.event(Event event)}\index{event()@{\texttt{event()}}} method,
that receives a parameter
of type \texttt{io.takamaka.code.lang.Event}\index{Event@{\texttt{Event}}}. The latter is simply an abstract class that
extends \texttt{Storage}. Hence, events will
be stored in the node as part of the transaction that generated that event.
The constructor of class \texttt{Event} is annotated as \texttt{@FromContract}, which allows one
to create events from the code of contracts only. The creating contract is available
through method \texttt{creator()}\index{creator()@{\texttt{creator()}}} of class \texttt{Event}.

In our example, the \texttt{BlindAuction} class uses two events, that you can add
to the \texttt{auction} package and are defined as follows:
%
\begin{codebox}\begin{javalst}
package io.hotmoka.tutorial.examples.auction;

import java.math.BigInteger;

import io.takamaka.code.lang.FromContract;
import io.takamaka.code.lang.Event;
import io.takamaka.code.lang.PayableContract;
import io.takamaka.code.lang.View;

public class BidIncrease extends Event {
  public final PayableContract bidder;
  public final BigInteger amount;

  @FromContract BidIncrease(PayableContract bidder, BigInteger amount) {
    this.bidder = bidder;
    this.amount = amount;
  }

  public @View PayableContract getBidder() {
    return bidder;
  }

  public @View BigInteger getAmount() {
    return amount;
  }
}
\end{javalst}\end{codebox}
%
and
%
\begin{codebox}\begin{javalst}
package io.hotmoka.tutorial.examples.auction;

import java.math.BigInteger;

import io.takamaka.code.lang.FromContract;
import io.takamaka.code.lang.Event;
import io.takamaka.code.lang.PayableContract;
import io.takamaka.code.lang.View;

public class AuctionEnd extends Event {
  public final PayableContract highestBidder;
  public final BigInteger highestBid;

  @FromContract AuctionEnd(PayableContract highestBidder, BigInteger highestBid) {
    this.highestBidder = highestBidder;
    this.highestBid = highestBid;
  }

  public @View PayableContract getHighestBidder() {
    return highestBidder;
  }

  public @View BigInteger getHighestBid() {
  return highestBid;
  }
}
\end{javalst}\end{codebox}

Now that all classes have been completed, the project should compile.
Go inside the project \texttt{io-hotmoka-tutorial-examples-auction} and run
\texttt{mvn install}.

\subsection{Running the blind auction contract}\label{subsec:running_blind_auction}

\begin{center}
(See the \texttt{io-hotmoka-tutorial-examples-runs} project in \texttt{\hotmokaRepo{}})
\end{center}

This section presents a Java class that connects to a Hotmoka node and runs the blind auction
contract of the previous section. We could run it in the \texttt{\serverMokamint{}} server,
but that Hotmoka node is based on a proof of space consensus, that generates a block every ten
seconds \emph{on average}. This means that, for a transaction to be committed, one could have to wait
more, sometime up to one minute. This would make the test slow and would require larger windows
for the bidding and for the revealing phases. Instead, we use the \texttt{\serverTendermint{}} server,
that is a Hotmoka node based on a proof of stake consensus, that generates a block every four seconds.
This makes the test faster and the timings reliable. However, this means that we must first generate
some new accounts for our tests, since those that we generated before for
\texttt{\serverMokamint{}} do not exist in \texttt{\serverTendermint{}}. We do it as previously done,
but swapping the server we are talking to:
%
\index{moka!keys create@{\texttt{keys create}}}
\inputCommand{moka_keys_create_account4}
\inputOutput{moka_keys_create_account4}
%
\inputCommand{moka_keys_create_account5}
\inputOutput{moka_keys_create_account5}
%
\inputCommand{moka_keys_create_account6}
\inputOutput{moka_keys_create_account6}
%
\index{moka!accounts create@{\texttt{accounts create}}}
\inputCommand{moka_accounts_create_account4}
\inputOutput{moka_accounts_create_account4}
%
\inputCommand{moka_accounts_create_account5}
\inputOutput{moka_accounts_create_account5}
%
\inputCommand{moka_accounts_create_account6}
\inputOutput{moka_accounts_create_account6}

Go to the \texttt{io-hotmoka-tutorial-examples-runs} Eclipse project and add the following
class inside its package:
%
\begin{codebox}\begin{javalst}
package io.hotmoka.tutorial.examples.runs;

import static io.hotmoka.helpers.Coin.panarea;
import static io.hotmoka.node.StorageTypes.BIG_INTEGER;
import static io.hotmoka.node.StorageTypes.BOOLEAN;
import static io.hotmoka.node.StorageTypes.BYTE;
import static io.hotmoka.node.StorageTypes.BYTES32_SNAPSHOT;
import static io.hotmoka.node.StorageTypes.INT;
import static io.hotmoka.node.StorageTypes.PAYABLE_CONTRACT;
import static io.hotmoka.node.StorageValues.byteOf;

import java.math.BigInteger;
import java.net.URI;
import java.nio.file.Files;
import java.nio.file.Path;
import java.nio.file.Paths;
import java.security.KeyPair;
import java.security.MessageDigest;
import java.util.ArrayList;
import java.util.List;
import java.util.Random;
import java.util.function.Function;

import io.hotmoka.constants.Constants;
import io.hotmoka.crypto.api.Signer;
import io.hotmoka.helpers.GasHelpers;
import io.hotmoka.helpers.NonceHelpers;
import io.hotmoka.helpers.SignatureHelpers;
import io.hotmoka.helpers.api.GasHelper;
import io.hotmoka.helpers.api.NonceHelper;
import io.hotmoka.node.Accounts;
import io.hotmoka.node.ConstructorSignatures;
import io.hotmoka.node.MethodSignatures;
import io.hotmoka.node.StorageTypes;
import io.hotmoka.node.StorageValues;
import io.hotmoka.node.TransactionRequests;
import io.hotmoka.node.api.Node;
import io.hotmoka.node.api.requests.SignedTransactionRequest;
import io.hotmoka.node.api.signatures.ConstructorSignature;
import io.hotmoka.node.api.transactions.TransactionReference;
import io.hotmoka.node.api.types.ClassType;
import io.hotmoka.node.api.values.StorageReference;
import io.hotmoka.node.api.values.StorageValue;
import io.hotmoka.node.remote.RemoteNodes;

public class Auction {

  public final static int NUM_BIDS = 10; // number of bids placed
  public final static int BIDDING_TIME = 230_000; // in milliseconds
  public final static int REVEAL_TIME = 350_000; // in milliseconds

  private final static BigInteger _500_000 = BigInteger.valueOf(500_000);

  private final static ClassType BLIND_AUCTION
    = StorageTypes.classNamed("io.hotmoka.tutorial.examples.auction.BlindAuction");
  private final static ConstructorSignature CONSTRUCTOR_BYTES32_SNAPSHOT
    = ConstructorSignatures.of(BYTES32_SNAPSHOT,
      BYTE, BYTE, BYTE, BYTE, BYTE, BYTE, BYTE, BYTE,
      BYTE, BYTE, BYTE, BYTE, BYTE, BYTE, BYTE, BYTE,
      BYTE, BYTE, BYTE, BYTE, BYTE, BYTE, BYTE, BYTE,
      BYTE, BYTE, BYTE, BYTE, BYTE, BYTE, BYTE, BYTE);

  private final TransactionReference takamakaCode;
  private final StorageReference[] accounts;
  private final List<Signer<SignedTransactionRequest<?>>> signers = new ArrayList<>();
  private final String chainId;
  private final long start;  // the time when bids started being placed
  private final Node node;
  private final TransactionReference classpath;
  private final StorageReference auction;
  private final List<BidToReveal> bids = new ArrayList<>();
  private final GasHelper gasHelper;
  private final NonceHelper nonceHelper;

  public static void main(String[] args) throws Exception {
	try (Node node = RemoteNodes.of(new URI(args[0]), 20000)) {
      new Auction(node, Paths.get(args[1]),
        StorageValues.reference(args[2]), args[3],
        StorageValues.reference(args[4]), args[5],
        StorageValues.reference(args[6]), args[7]);
    }
  }

  /**
  * Class used to keep in memory the bids placed by each player,
  * that will be revealed at the end.
  */
  private class BidToReveal {
    private final int player;
    private final BigInteger value;
    private final boolean fake;
    private final byte[] salt;

    private BidToReveal(int player, BigInteger value, boolean fake, byte[] salt) {
      this.player = player;
      this.value = value;
      this.fake = fake;
      this.salt = salt;
    }

    /**
    * Creates in store a revealed bid corresponding to this object.
    * 
    * @return the storage reference to the freshly created revealed bid
    */
    private StorageReference intoBlockchain() throws Exception {
      StorageReference bytes32 = node.addConstructorCallTransaction(TransactionRequests.constructorCall
        (signers.get(player), accounts[player],
        nonceHelper.getNonceOf(accounts[player]), chainId, _500_000,
        panarea(gasHelper.getSafeGasPrice()), classpath, CONSTRUCTOR_BYTES32_SNAPSHOT,
        byteOf(salt[0]), byteOf(salt[1]), byteOf(salt[2]), byteOf(salt[3]),
        byteOf(salt[4]), byteOf(salt[5]), byteOf(salt[6]), byteOf(salt[7]),
        byteOf(salt[8]), byteOf(salt[9]), byteOf(salt[10]), byteOf(salt[11]),
        byteOf(salt[12]), byteOf(salt[13]), byteOf(salt[14]), byteOf(salt[15]),
        byteOf(salt[16]), byteOf(salt[17]), byteOf(salt[18]), byteOf(salt[19]),
        byteOf(salt[20]), byteOf(salt[21]), byteOf(salt[22]), byteOf(salt[23]),
        byteOf(salt[24]), byteOf(salt[25]), byteOf(salt[26]), byteOf(salt[27]),
        byteOf(salt[28]), byteOf(salt[29]), byteOf(salt[30]), byteOf(salt[31])));

      var CONSTRUCTOR_REVEALED_BID
        = ConstructorSignatures.of(
           StorageTypes.classNamed("io.hotmoka.tutorial.examples.auction.BlindAuction$RevealedBid"),
           BIG_INTEGER, BOOLEAN, BYTES32_SNAPSHOT);

      return node.addConstructorCallTransaction(TransactionRequests.constructorCall
        (signers.get(player), accounts[player],
        nonceHelper.getNonceOf(accounts[player]), chainId,
        _500_000, panarea(gasHelper.getSafeGasPrice()), classpath, CONSTRUCTOR_REVEALED_BID,
        StorageValues.bigIntegerOf(value), StorageValues.booleanOf(fake), bytes32));
    }
  }

  private Auction(Node node, Path dir, StorageReference account1, String password1,
      StorageReference account2, String password2, StorageReference account3, String password3)
      throws Exception {

    this.node = node;
    takamakaCode = node.getTakamakaCode();
    accounts = new StorageReference[] { account1, account2, account3 };
    var signature = node.getConfig().getSignatureForRequests();
    Function<? super SignedTransactionRequest<?>, byte[]> toBytes
      = SignedTransactionRequest<?>::toByteArrayWithoutSignature;
    signers.add(signature.getSigner(loadKeys(node, dir, account1, password1).getPrivate(), toBytes));
    signers.add(signature.getSigner(loadKeys(node, dir, account2, password2).getPrivate(), toBytes));
    signers.add(signature.getSigner(loadKeys(node, dir, account3, password3).getPrivate(), toBytes));
    gasHelper = GasHelpers.of(node);
    nonceHelper = NonceHelpers.of(node);
    chainId = node.getConfig().getChainId();
    classpath = installJar();
    auction = createContract();
    start = System.currentTimeMillis();

    StorageReference expectedWinner = placeBids();
    waitUntilEndOfBiddingTime();
    revealBids();
    waitUntilEndOfRevealTime();
    StorageValue winner = askForWinner();

    // show that the contract computes the correct winner
    System.out.println("expected winner: " + expectedWinner);
    System.out.println("actual winner: " + winner);
  }

  private StorageReference createContract() throws Exception {
    System.out.println("Creating contract");

    var CONSTRUCTOR_BLIND_AUCTION = ConstructorSignatures.of(BLIND_AUCTION, INT, INT);

    return node.addConstructorCallTransaction
      (TransactionRequests.constructorCall(signers.get(0), accounts[0],
      nonceHelper.getNonceOf(accounts[0]), chainId, _500_000, panarea(gasHelper.getSafeGasPrice()),
      classpath, CONSTRUCTOR_BLIND_AUCTION,
      StorageValues.intOf(BIDDING_TIME), StorageValues.intOf(REVEAL_TIME)));
  }

  private TransactionReference installJar() throws Exception {
    System.out.println("Installing jar");

    //the path of the user jar to install
    var auctionPath = Paths.get(System.getProperty("user.home")
      + "/.m2/repository/io/hotmoka/io-hotmoka-tutorial-examples-auction/"
      + Constants.HOTMOKA_VERSION
      + "/io-hotmoka-tutorial-examples-auction-" + Constants.HOTMOKA_VERSION + ".jar");

    return node.addJarStoreTransaction(TransactionRequests.jarStore
      (signers.get(0), // an object that signs with the payer's private key
      accounts[0], // payer
      nonceHelper.getNonceOf(accounts[0]), // payer's nonce
      chainId, // chain identifier
      BigInteger.valueOf(5_000_000), // gas limit: enough for this jar
      gasHelper.getSafeGasPrice(), // gas price: at least the current gas price of the network
      takamakaCode, // class path for the execution of the transaction
      Files.readAllBytes(auctionPath), // bytes of the jar to install
      takamakaCode)); // dependency
  }

  private StorageReference placeBids() throws Exception {
    var maxBid = BigInteger.ZERO;
    StorageReference expectedWinner = null;
    var random = new Random();
    var BID = MethodSignatures.ofVoid(BLIND_AUCTION, "bid", BIG_INTEGER, BYTES32_SNAPSHOT);

    int i = 1;
    while (i <= NUM_BIDS) { // generate NUM_BIDS random bids
      System.out.println("Placing bid " + i + "/" + NUM_BIDS);
      int player = 1 + random.nextInt(accounts.length - 1);
      var deposit = BigInteger.valueOf(random.nextInt(1000));
      var value = BigInteger.valueOf(random.nextInt(1000));
      boolean fake = random.nextInt(100) >= 80;
      var salt = new byte[32];
      random.nextBytes(salt); // random 32 bytes of salt for each bid

      // create a Bytes32 hash of the bid in the store of the node
      StorageReference bytes32 = codeAsBytes32(player, value, fake, salt);

      // keep note of the best bid, to verify the result at the end
      if (!fake && deposit.compareTo(value) >= 0)
        if (expectedWinner == null || value.compareTo(maxBid) > 0) {
          maxBid = value;
          expectedWinner = accounts[player];
        }
        else if (value.equals(maxBid))
          // we do not allow ex aequos, since the winner
          // would depend on the fastest player to reveal
          continue;

      // keep the explicit bid in memory, not yet in the node,
      // since it would be visible there
      bids.add(new BidToReveal(player, value, fake, salt));

      // place a hashed bid in the node
      node.addInstanceMethodCallTransaction(TransactionRequests.instanceMethodCall
        (signers.get(player), accounts[player],
        nonceHelper.getNonceOf(accounts[player]), chainId,
        _500_000, panarea(gasHelper.getSafeGasPrice()), classpath, BID,
        auction, StorageValues.bigIntegerOf(deposit), bytes32));

      i++;
    }

    return expectedWinner;
  }

  private void revealBids() throws Exception {
    var REVEAL = MethodSignatures.ofVoid
      (BLIND_AUCTION, "reveal",
       StorageTypes.classNamed("io.hotmoka.tutorial.examples.auction.BlindAuction$RevealedBid"));

    // we create the revealed bids in blockchain; this is safe now, since the bidding time is over
    int counter = 1;
    for (BidToReveal bid: bids) {
      System.out.println("Revealing bid " + counter++ + "/" + bids.size());
      int player = bid.player;
      StorageReference bidInBlockchain = bid.intoBlockchain();
      node.addInstanceMethodCallTransaction(TransactionRequests.instanceMethodCall
        (signers.get(player), accounts[player],
        nonceHelper.getNonceOf(accounts[player]), chainId, _500_000,
        panarea(gasHelper.getSafeGasPrice()),
        classpath, REVEAL, auction, bidInBlockchain));
    }
  }

  private StorageReference askForWinner() throws Exception {
    var AUCTION_END = MethodSignatures.ofNonVoid
      (BLIND_AUCTION, "auctionEnd", PAYABLE_CONTRACT);

    StorageValue winner = node.addInstanceMethodCallTransaction
      (TransactionRequests.instanceMethodCall
      (signers.get(0), accounts[0], nonceHelper.getNonceOf(accounts[0]),
      chainId, _500_000, panarea(gasHelper.getSafeGasPrice()),
      classpath, AUCTION_END, auction)).get();

    // the winner is normally a StorageReference,
    // but it could be a NullValue if all bids were fake
    return winner instanceof StorageReference sr ? sr : null;
  }

  private void waitUntilEndOfBiddingTime() {
    waitUntil(BIDDING_TIME + 10000, "Waiting until the end of the bidding time");
  }

  private void waitUntilEndOfRevealTime() {
    waitUntil(BIDDING_TIME + REVEAL_TIME + 10000, "Waiting until the end of the revealing time");
  }

  /**
  * Waits until a specific time after start.
  */
  private void waitUntil(long duration, String forWhat) {
    long msToWait = start + duration - System.currentTimeMillis();
    System.out.println(forWhat + " (" + msToWait + "ms still missing)");
	try {
      Thread.sleep(msToWait);
    }
    catch (InterruptedException e) {
      Thread.currentThread().interrupt();
    }
  }

  /**
  * Hashes a bid and put it in the store of the node, in hashed form.
  */
  private StorageReference codeAsBytes32(int player, BigInteger value, boolean fake, byte[] salt)
      throws Exception {
	// the hashing algorithm used to hide the bids
	var digest = MessageDigest.getInstance("SHA-256");
    digest.update(value.toByteArray());
    digest.update(fake ? (byte) 0 : (byte) 1);
    digest.update(salt);
    byte[] hash = digest.digest();
    return createBytes32(player, hash);
  }

  /**
  * Creates a Bytes32Snapshot object in the store of the node.
  */
  private StorageReference createBytes32(int player, byte[] hash) throws Exception {
    return node.addConstructorCallTransaction
      (TransactionRequests.constructorCall(
      signers.get(player),
      accounts[player],
      nonceHelper.getNonceOf(accounts[player]), chainId,
      _500_000, panarea(gasHelper.getSafeGasPrice()),
      classpath, CONSTRUCTOR_BYTES32_SNAPSHOT,
      byteOf(hash[0]), byteOf(hash[1]),
      byteOf(hash[2]), byteOf(hash[3]),
      byteOf(hash[4]), byteOf(hash[5]),
      byteOf(hash[6]), byteOf(hash[7]),
      byteOf(hash[8]), byteOf(hash[9]),
      byteOf(hash[10]), byteOf(hash[11]),
      byteOf(hash[12]), byteOf(hash[13]),
      byteOf(hash[14]), byteOf(hash[15]),
      byteOf(hash[16]), byteOf(hash[17]),
      byteOf(hash[18]), byteOf(hash[19]),
      byteOf(hash[20]), byteOf(hash[21]),
      byteOf(hash[22]), byteOf(hash[23]),
      byteOf(hash[24]), byteOf(hash[25]),
      byteOf(hash[26]), byteOf(hash[27]),
      byteOf(hash[28]), byteOf(hash[29]),
      byteOf(hash[30]), byteOf(hash[31])));
  }

  private static KeyPair loadKeys(Node node, Path dir, StorageReference account, String password)
      throws Exception {
    return Accounts.of(account, dir).keys(password,
      SignatureHelpers.of(node).signatureAlgorithmFor(account));
  }
}
\end{javalst}\end{codebox}

This test class is relatively long and complex. Let us start from its beginning.
The code specifies that the test will place 10 random bids, that the bidding phase
lasts $100$ seconds and that the reveal phase lasts $140$ seconds
(these timings are fine on a blockchain that creates a block every four seconds;
shorter block creation times would allow shorter timings):
%
\begin{codebox}\begin{javalst}
public final static int NUM_BIDS = 10;
public final static int BIDDING_TIME = 230_000;
public final static int REVEAL_TIME = 350_000;
\end{javalst}\end{codebox}

Some constant signatures follow,
that simplify the calls to methods and constructors later.
Method \texttt{main()} connects to a remote node and passes it
as a parameter to the constructor of class \texttt{Auction}, that
installs \texttt{io-hotmoka-tutorial-examples-auction-\hotmokaVersion{}.jar} inside it.
It stores the node in field \texttt{node}.
Then the constructor of \texttt{Auction} creates an \texttt{auction} contract in the node
and calls method \texttt{placeBids()} that
uses the inner class \texttt{BidToReveal} to keep track of the bids placed
during the test, in clear. Initially, bids are kept in
memory, not in the store of the node, where they could be publicly accessed.
Only their hashes are stored in the node.
Method \texttt{placeBids()} generates \texttt{NUM\_BIDS} random bids on behalf
of the \texttt{accounts.length - 1} players (the first element of the
\texttt{accounts} array is the creator of the auction):
%
\begin{codebox}\begin{javalst}
int i = 1;
while (i <= NUM_BIDS) {
  int player = 1 + random.nextInt(accounts.length - 1);
  var deposit = BigInteger.valueOf(random.nextInt(1000));
  var value = BigInteger.valueOf(random.nextInt(1000));
  var fake = random.nextInt(100) >= 80; // fake in 20% of the cases
  var salt = new byte[32];
  random.nextBytes(salt);
  ...
}
\end{javalst}\end{codebox}

Each random bid is hashed (including a random salt) and a \texttt{Bytes32Snapshot} object
is created in the store of the node, containing that hash:
%
\begin{codebox}\begin{javalst}
StorageReference bytes32 = codeAsBytes32(player, value, fake, salt);
\end{javalst}\end{codebox}
%
The bid, in clear, is added to a list \texttt{bids} that, at the end of the loop,
will contain all bids:
%
\begin{codebox}\begin{javalst}
bids.add(new BidToReveal(player, value, fake, salt));
\end{javalst}\end{codebox}
%
The hash is used instead to place a bid in the node:
%
\begin{codebox}\begin{javalst}
node.addInstanceMethodCallTransaction(TransactionRequests.instanceMethodCall
  (signers.get(player), accounts[player],
  nonceHelper.getNonceOf(accounts[player]), chainId,
  _500_000, panarea(gasHelper.getSafeGasPrice()), classpath, BID,
  auction, StorageValues.bigIntegerOf(deposit), bytes32));
\end{javalst}\end{codebox}
%
The loop takes also care of keeping track of the best bidder, that placed
the best bid, so that it can be compared at the end with the best bidder
computed by the smart contract (they should coincide):
%
\begin{codebox}\begin{javalst}
if (!fake && deposit.compareTo(value) >= 0)
  if (expectedWinner == null || value.compareTo(maxBid) > 0) {
    maxBid = value;
    expectedWinner = accounts[player];
  }
  else if (value.equals(maxBid))
    continue;
\end{javalst}\end{codebox}
%
As you can see, the test above avoids generating a bid that
is equal to the best bid seen so far. This avoids having two bidders
that place the same bid: the smart contract will consider as winner
the first bidder that reveals its bids. To avoid this tricky case, we prefer
to assume that the best bid is unique. This is just a simplification of the
testing code, since the smart contract deals perfectly with that case.

After all bids have been placed, the constructor of \texttt{Auction} waits until the end of
the bidding time:
%
\begin{codebox}\begin{javalst}
waitUntilEndOfBiddingTime();
\end{javalst}\end{codebox}
%
Then it calls method \texttt{revealBids()}, that reveals
the bids to the smart contract, in plain. It creates in the store of the node
a data structure
\texttt{RevealedBid} for each elements of the list \texttt{bids}, by calling
\texttt{bid.intoBlockchain()}.
This creates the bid in clear in the store of the node, but this is safe now,
since the bidding time is over and
they cannot be used to guess a winning bid anymore. Then method \texttt{revealBids()}
reveals the bids by calling method \texttt{reveal()} of the smart contract:
%
\begin{codebox}\begin{javalst}
for (BidToReveal bid: bids) {
  int player = bid.player;
  StorageReference bidInBlockchain = bid.intoBlockchain();
  node.addInstanceMethodCallTransaction(TransactionRequests.instanceMethodCall
    (signers.get(player), accounts[player],
    nonceHelper.getNonceOf(accounts[player]), chainId, _500_000,
    panarea(gasHelper.getSafeGasPrice()),
    classpath, REVEAL, auction, bidInBlockchain));
}
\end{javalst}\end{codebox}
%
Note that this is possible since the inner class \texttt{RevealedBid} of the
smart contract has been annotated as \texttt{@Exported}
(see its code in Sec.~\ref{subsec:blind_auction}), hence its instances can be
passed as argument to calls from outside the blockchain.

Subsequently, the constructor of \texttt{Auction} waits until the end of the reveal phase:
%
\begin{codebox}\begin{javalst}
waitUntilEndOfRevealTime();
\end{javalst}\end{codebox}
%
After that, method \texttt{askForWinner()}
signals to the smart contract that the auction is over and asks about the winner:
%
\begin{codebox}\begin{javalst}
StorageValue winner = node.addInstanceMethodCallTransaction
  (TransactionRequests.instanceMethodCall
  (signers.get(0), accounts[0], nonceHelper.getNonceOf(accounts[0]),
  chainId, _500_000, panarea(gasHelper.getSafeGasPrice()),
  classpath, AUCTION_END, auction)).get();
\end{javalst}\end{codebox}
  
The final two \texttt{System.out.println()}'ss in the constructor of
\texttt{Auction} allow one to verify that the smart contract
actually computes the right winner, since they will always print the identical storage
object (different at each run, in general), such as:
%
\begin{shellbox}\begin{alltt}
expected winner: \accountFiveShort{}
actual winner: \accountFiveShort{}
\end{alltt}\end{shellbox}

We can run class \texttt{Auction} now (please note that the execution of this test will take a few minutes):
%
\inputCommand{mvn_exec_blind_auction}
%
Its execution should print something like this on the console:
%
\inputOutput{mvn_exec_blind_auction}

\subsection{Listening to events}\label{subsec:listening_to_events}
%
\begin{center}
(See the \texttt{io-hotmoka-tutorial-examples-runs} project in \texttt{\hotmokaRepo{}})
\end{center}

The \texttt{BlindAuction} contract generates events during its execution. If an external tool, such
as a wallet, wants to listen to such events and trigger some activity when they occur,
it is enough for it to subscribe to the events of a node that is executing the contract,
by providing a handler that gets executed each time a new event gets generated.
Subscription requires to specify the creator of the events that should be forwarded to the
handler. In our case, this is the \texttt{auction} contract. Thus, clone the \texttt{Auction.java} class into
\texttt{Events.java} and modify its constructor as follows:
%
\begin{codebox}\begin{javalst}
...
import io.hotmoka.node.api.ClosedNodeException;
import io.hotmoka.node.api.UnknownReferenceException;
...
auction = createAuction();
start = System.currentTimeMillis();

try (var subscription = node.subscribeToEvents(auction, this::eventHandler)) {
  StorageReference expectedWinner = placeBids();
  waitUntilEndOfBiddingTime();
  revealBids();
  waitUntilEndOfRevealTime();
  StorageValue winner = askForWinner();

  System.out.println("expected winner: " + expectedWinner);
  System.out.println("actual winner: " + winner);

  waitUntilAllEventsAreFlushed();
}

private void waitUntilAllEventsAreFlushed() {
  waitUntil(BIDDING_TIME + REVEAL_TIME + 30000, "Waiting until all events are flushed");
}

private void eventHandler(StorageReference creator, StorageReference event) {
  try {
    System.out.println
      ("Seen event of class " + node.getClassTag(event).getClazz()
       + " created by contract " + creator);
  }
  catch (ClosedNodeException | UnknownReferenceException | TimeoutException e) {
    System.out.println("The node is misbehaving: " + e.getMessage());
  }
  catch (InterruptedException e) {
    Thread.currentThread().interrupt();
  }
}
\end{javalst}\end{codebox}
%
The event handler, in this case, simply prints on the console the class of the event and its creator
(that will coincide with \texttt{auction}). You can run the \texttt{Events} class now:
%
\inputCommand{mvn_exec_blind_auction_events}
%
You should see something like this on the console:
%
\inputOutput{mvn_exec_blind_auction_events}
%
\begin{commentbox}
The \texttt{subscribeToEvents()}\index{subscribeToEvents()@{\texttt{subscribeToEvents()}}}
method returns a \texttt{Subscription}\index{Subscription@{\texttt{Subscription}}} object that should be
closed when it is not needed anymore, in order to reduce the overhead on the node.
Since it is an \texttt{AutoCloseable} resource, the recommended technique is to use a
try-with-resource construct, as shown in the previous example.
Moreover, our code waits for a few seconds before closing the
subscription to the events, in order to give events the time to be forwarded to the client.
\end{commentbox}

In general, event handlers can perform arbitrarily complex operations and even access the
event object in the store of the node,
from its storage reference, reading its fields or calling its methods. Please remember, however,
that event handlers are run in a thread of the node. Hence, they should be fast and shouldn't hang.
It is good practice to let event handlers add events in a queue, in a non-blocking way.
A consumer thread, external to the node, then retrieves the events from the queue and processes them in turn.

It is possible to subscribe to \emph{all} events generated by a node,
by using \texttt{null} as creator in the \texttt{subscribeToEvents()} method. Think twice before doing that,
since your handler will be notified of \emph{all} events generated by \emph{any} application installed in
the node. It might be a lot.
\chapter{Tokens}\label{ch:tokens}

A popular class of smart contracts
implement a dynamic ledger of coin transfers between accounts. These
coins are not native tokens, but rather new, derived tokens.
In some sense, tokens are programmed money, whose rules are specified
by a smart contract and enforced by the underlying blockchain.

\begin{commentbox}
In this context, the term \emph{token}\index{token} is used
for the smart contract that tracks coin transfers, for the single coin units and for the category
of similar coins. This is sometimes confusing.
\end{commentbox}

Native and derived tokens can be categorized in many
ways~\cite{OliveiraZBS18,FreniFM20,Tapscott20}.
The most popular classification
is between \emph{fungible}\index{token!fungible} and \emph{non-fungible}\index{token!non-fungible} tokens.
Fungible tokens are interchangeable with each other, since they have an identical
nominal value that does not depend on each specific token instance.
Native tokens and traditional (\emph{fiat}) currencies are both examples of fungible tokens.
Their main application is in the area of crowdfunding and initial coin offers
to support startups.
On the contrary, non-fungible tokens have a value that depends on their specific instance.
Hence, in general, they are not interchangeable.
Their main application is currently in the art market, where they represent
a written declaration of author's rights concession to the holder.

A few standards have emerged for such tokens,
that should guarantee correctness,
accessibility, interoperability, management and security
of the smart contracts that run the tokens.
Among them, the Ethereum Requests for Comment \#20
(ERC-20~\cite{VogelstellerB15}\index{ERC20}) and \#721 (ERC-721~\cite{EntrikenSES18}\index{ERC721})
are the most popular, also outside Ethereum. They provide developers with
a list of rules required for the correct integration of tokens
with other smart contracts and with applications external to the blockchain,
such as wallets, block explorers, decentralized finance protocols and games.

The most popular implementations of the ERC-20 and ERC-721 standards are in Solidity,
by OpenZeppelin~\cite{OpenZeppelinERC20,OpenZeppelinERC721}\index{OpenZeppelin},
a team of programmers in the Ethereum community
who deliver useful and secure smart contracts and libraries, and by
ConsenSys, later deprecated in favor of OpenZeppelin's.
OpenZeppelin extends ERC-20 with snapshots, that is,
immutable views of the state of a token contract, that show
its ledger at a specific instant of time.
They are useful to investigate the consequences of an attack, to create forks of the token
and to implement mechanisms based on token balances such as weighted voting.

\section{Fungible tokens (ERC20)}\label{sec:erc20}

A fungible token ledger is a ledger that binds owners (contracts) to
the numerical amount of tokens they own. With this very high-level description,
it is an instance of the \texttt{IERC20View}\index{IERC20View@{\texttt{IERC20View}}} interface in Fig.~\ref{fig:erc20_hierarchy}.
The \texttt{balanceOf()} method tells how many tokens an \texttt{account} holds and the method
\texttt{totalSupply()} provides the total number of tokens in circulation.
The \texttt{UnsignedBigInteger}\index{UnsignedBigInteger@{\texttt{UnsignedBigInteger}}}
class is a Takamaka library class that wraps a \texttt{BigInteger}
and guarantees that its value is never negative. For instance, the subtraction of two
\texttt{UnsignedBigInteger}s throws an exception when the second is larger than the first.

\begin{figure}[th]
  \begin{center}
	\begin{tikzpicture}[scale=1]
	\scriptsize
	\interfacecolor
	\begin{interface}[text width=6cm]{IERC20View}{0,0}
	\operation{@View totalSupply():UBI}
	\operation{@View balanceOf(account:Contract):UBI}
	\operation{@View snapshot():IERC20View}
	\end{interface}
	\begin{interface}[text width=11cm]{IERC20}{0, -3}
	\inherit{IERC20View}
	\operation{@FromContract trasfer(to:Contract, amount: UBI|int|long):boolean}
	\operation{@View allowance(owner:Contract, spender:Contract):UBI}
	\operation{@FromContract approve(spender:Contract, amount:UBI):boolean}
	\operation{@FromContract transferFrom(from:Contract, to:Contract, amount:UBI):boolean}
	\operation{view():IERC20View}
	\end{interface}	
	\externalclasscolor
	\begin{abstractclass}[text width=2cm]{Contract}{-5,-6}
	\end{abstractclass}
	\classcolor
	\begin{class}[text width=6cm]{ERC20}{0,-7}
		\inherit{Contract}
		\implement{IERC20}
		\operation{ERC20(name:String, symbol:String)}
		\operation{@View name(), symbol():String}
		\operation{@View decimals():short)}
		\operation{\# \_mint(account:Contract, amount:UBI)}
		\operation{\# \_burn(account:Contract, amount:UBI)}
	\end{class}
	\begin{class}[text width=5cm]{ERC20Burnable}{-5,-10}
		\inherit{ERC20}
		\operation{ERC20Burnable(name:String, symbol:String)}
		\operation{@FromContract burn(amount:UBI)}
		\operation{@FromContract burnFrom(account:Contract, amount:UBI)}
	\end{class}
	\begin{class}[text width=5cm]{ERC20Capped}{1,-10}
		\inherit{ERC20}
		\operation{ERC20Capped(name:String, symbol:String, cap:UBI)}
	\end{class}
	\ourclasscolor
	\begin{class}[text width=5cm]{CryptoBuddy}{7,-10}
		\inherit{ERC20}
		\operation{@FromContract CryptoBuddy()}
		\operation{@FromContract mint(account:Contract, amount:UBI)}
		\operation{@FromContract burn(account:Contract, amount:UBI)}
	\end{class}
	\end{tikzpicture}
  \end{center}
  \caption{The hierarchy of the ERC20 token implementations. \texttt{UBI} is shorthand
  		   for the library class \texttt{UnsignedBigInteger}.}
  \label{fig:erc20_hierarchy}
\end{figure}

The \texttt{snapshot()}\index{snapshot()@{\texttt{snapshot()}}} method,
as already seen for collection classes, yields a read-only,
frozen view of the latest state of the token ledger.
Since it is defined in the topmost interface, all token classes
can be snapshotted. Snapshots are computable in constant time.
%
\begin{commentbox}
In the original ERC20 standard and implementation in Ethereum,
only specific subclasses allow snapshots, since their creation adds gas costs to all
operations, also for token owners that never performed any snapshot.
See the discussion and comparison in~\cite{CrosaraOST23}.
\end{commentbox}

An ERC20 ledger is typically modifiable. Namely, owners
can sell tokens to other owners
and can delegate trusted contracts to transfer tokens on their behalf.
Of course, these operations must be legal, in the sense that an owner cannot sell
more tokens than it owns and delegated contracts cannot transfer more tokens than the
cap to their delegation.
These modification operations are defined in the
\texttt{IERC20}\index{IERC20@{\texttt{IERC20}}} interface in Fig.~\ref{fig:erc20_hierarchy}. They are identical to the same
operations in the ERC20 standard for Ethereum, hence we refer to that standard for further detail.
The \texttt{view()}\index{view()@{\texttt{view()}}} method is used to yield a \emph{view} of the ledger, that is, an object
that reflects the current state of the original ledger, but without any modification operation.

The \texttt{ERC20}\index{ERC20@{\texttt{ERC20}}} implementation provides a standard implementation for the functions defined
in the \texttt{IERC20View} and \texttt{IERC20} interfaces. Moreover, it provides metadata information
such as name, symbol and number of decimals for the specific token implementation.
There are protected implementations for methods that allow one to mint or burn an amount
of tokens for a given owner (\texttt{account}). These are protected since one does not
want to allow everybody to print or burn money. Instead, subclasses can call into these
methods in their constructor, to implement an initial distribution of tokens,
and can also allow subsequent, controlled mint or burns.
For instance, the \texttt{ERC20Burnable}\index{ERC20Burnable@{\texttt{ERC20Burnable}}}
class is an \texttt{ERC20} implementation that
allows a token owner to burn its tokens only, or those it has been
delegated to transfer, but never those of another owner.
The \texttt{ERC20Capped}\index{ERC20Capped@{\texttt{ERC20Capped}}} implementation
allows the specification of a maximal cap to the
number of tokens in circulation. When new tokens get minted, it checks that the cap
is not exceeded and throws an exception otherwise.

\subsection{Implementing our own ERC20 token}\label{subsec:implementing_erc20}

\begin{center}
(See the \texttt{io-hotmoka-tutorial-examples-erc20} project in \texttt{\hotmokaRepo{}})
\end{center}

Let us define a token ledger class that allows only its creator the mint or burn tokens.
We will call it \texttt{CryptoBuddy}. As Fig.~\ref{fig:erc20_hierarchy} shows,
we plug it below the \texttt{ERC20} implementation, so that we inherit that implementation
and do not need to reimplement the methods of the \texttt{IERC20} interface.

Create in Eclipse a new Maven Java~21 (or later). Use, as name for this project, the string
\texttt{io-hotmoka-tutorial-examples-erc20}.
You can do this by duplicating the previous project \texttt{io-hotmoka-tutorial-examples-family}.
Use the following \texttt{pom.xml}:
%
\begin{codebox}\begin{xmllst}
<project xmlns="http://maven.apache.org/POM/4.0.0"
    xmlns:xsi="http://www.w3.org/2001/XMLSchema-instance"
    xsi:schemaLocation="http://maven.apache.org/POM/4.0.0 http://maven.apache.org/xsd/maven-4.0.0.xsd">

  <modelVersion>4.0.0</modelVersion>
  <groupId>io.hotmoka</groupId>
  <artifactId>io-hotmoka-tutorial-examples-erc20</artifactId>
  <version>|\hotmokaVersion{}|</version>

  <properties>
    <project.build.sourceEncoding>UTF-8</project.build.sourceEncoding>
    <maven.compiler.release>21</maven.compiler.release>
  </properties>

  <dependencies>
    <dependency>
      <groupId>io.hotmoka</groupId>
      <artifactId>io-takamaka-code</artifactId>
      <version>|\takamakaVersion{}|</version>
    </dependency>
  </dependencies>

  <build>
    <plugins>
      <plugin>
        <groupId>org.apache.maven.plugins</groupId>
        <artifactId>maven-compiler-plugin</artifactId>
        <version>3.11.0</version>
      </plugin>
    </plugins>
  </build>

</project>
\end{xmllst}\end{codebox}
%
and the following \texttt{module-info.java}:
%
\begin{codebox}\begin{javalst}
module erc20 {
  requires io.takamaka.code;
}
\end{javalst}\end{codebox}
%
Create package \texttt{io.hotmoka.tutorial.examples.erc20}
inside \texttt{src/main/java} and add
the following \texttt{CryptoBuddy.java} source inside that package:
%
\begin{codebox}\begin{javalst}
package io.hotmoka.tutorial.examples.erc20;

import static io.takamaka.code.lang.Takamaka.require;

import io.takamaka.code.lang.Contract;
import io.takamaka.code.lang.FromContract;
import io.takamaka.code.math.UnsignedBigInteger;
import io.takamaka.code.tokens.ERC20;

public class CryptoBuddy extends ERC20 {
  private final Contract owner;

  public @FromContract CryptoBuddy() {
    super("CryptoBuddy", "CB");
    owner = caller();
    var initialSupply = new UnsignedBigInteger("200000");
    var multiplier = new UnsignedBigInteger("10").pow(18);
    _mint(caller(), initialSupply.multiply(multiplier)); // 200000 * 10 ^ 18
  }

  public @FromContract void mint(Contract account, UnsignedBigInteger amount) {
    require(caller() == owner, "Lack of permission");
    _mint(account, amount);
  }

  public @FromContract void burn(Contract account, UnsignedBigInteger amount) {
    require(caller() == owner, "Lack of permission");
    _burn(account, amount);
  }
}
\end{javalst}\end{codebox}
%
The constructor of \texttt{CryptoBuddy} initializes the total supply by minting
a very large number of tokens. They are initially owned by the creator of the contract,
that is saved as \texttt{owner}. Methods \texttt{mint()} and \texttt{burn()} check
that the owner is requesting
the mint or burn and call the inherited protected methods in that case.

You can generate the jar archive and install that jar in the node, by letting our first account pay:
%
\index{moka!jars install@{\texttt{jars install}}}
\inputCommand{moka_jars_install_erc20}
\inputOutput{moka_jars_install_erc20}
%
Finally, you can create an instance of the token class, by always letting our first account pay for that:
%
\index{moka!objects create@{\texttt{objects create}}}
\inputCommand{moka_objects_create_erc20}
\inputOutput{moka_objects_create_erc20}
%
The new ledger instance has been installed in the storage of the node now, at the address
\texttt{\ercTwentyObjectShort{}}. It is possible to start interacting with that ledger instance, by transferring
tokens between accounts. For instance, this can be done with the \texttt{moka objects call} command,
that allows one to invoke the \texttt{transfer()} or \texttt{transferFrom()} methods of the ledger.
It is possible to show the state of the ledger with the \texttt{moka objects show} command, although specific
utilities will provide a more user-friendly view of the ledger in the future.

\subsection{Richer than expected}\label{subsec:richer_than_expected}

Every owner of an ERC20 token can decide to send some of its tokens to another
contract \emph{C}, that will become an owner itself, if it was not already.
This means that the ledger inside
an \texttt{ERC20} implementation gets modified and some tokens get registered for
the new owner \emph{C}. However, \emph{C} is not notified in any way of this transfer.
This means that our contracts could be richer than we expect, if somebody
has sent tokens to them, possibly inadvertently. In theory, we could
scan the whole memory of a Hotmoka node, looking for implementations
of the \texttt{IERC20} interface, and check if our contracts are registered inside them.
Needless to say, this is computationally irrealistic.
Moreover, even if we know that one of our contracts is waiting to receive
some tokens, we don't know immediately when this happens, since
the contract does not get notified of any transfer of tokens.

This issue is inherent to the definition of the ERC20 standard in Ethereum
and the implementation in Takamaka inherits this limitation, since it wants to stick
as much as possible to the Ethereum standard. A solution to the problem
would be to restrict the kind of owners that are allowed in Fig.~\ref{fig:erc20_hierarchy}.
Namely, instead of allowing all instances of \texttt{Contract}, the signature of the methods could
be restricted to owners of some interface type \texttt{IERC20Receiver}, with a single method
\texttt{onReceive()} that gets called by the \texttt{ERC20} implementation, every time
tokens get transferred to an \texttt{IERC20Receiver}.
In this way, owners of ERC20 tokens get notified when they receive new tokens.
This solution has never been implemented for ERC20 tokens in Ethereum, while
it has been used in the ERC721 standard for non-fungible tokens, as we will
show in the next section.

\section{Non-fungible tokens (ERC721)}\label{sec:non_fungible_tokens}

A non-fungible token\index{token!non-fungible} is implemented as a ledger that maps each token identifier to its owner.
Ethereum provides the ERC721 specification~\cite{EntrikenSES18}\index{ERC721} for non-fungible tokens.
There, a token identifier is an array of bytes. Takamaka uses, more generically,
a \texttt{BigInteger}. Note that a \texttt{BigInteger} can be constructed from an array of bytes
by using the constructor of class \texttt{BigInteger} that receives an array of bytes.
In the ERC721 specification, token owners are contracts, although the implementation will check
that only contracts implementing the
\texttt{IERC721Receiver}\index{IERC721Receiver@{\texttt{IERC721Receiver}}} interface are added
to an IERC721 ledger, or externally owned accounts.
%
\begin{commentbox}
The reason for allowing externally owned accounts is probably a simplification,
since Ethereum users own externally owned accounts and it is simpler
to use such accounts directly inside an ERC721 ledger, instead of creating
contracts of type \texttt{IERC721Receiver}. In any case, no other kind of contracts
is allowed in standard ERC721 implementations.
\end{commentbox}

The hierarchy of the Takamaka classes for the ERC721 standard is shown
in Fig.~\ref{fig:erc721_hierarchy}.
%
\begin{figure}[th]
  \begin{center}
	\begin{tikzpicture}[scale=1]
	\scriptsize
	\interfacecolor
	\begin{interface}[text width=7cm]{IERC721View}{0,0}
	\operation{@View ownerOf(tokenId:BigInteger):Contract}
	\operation{@View balanceOf(account:Contract):BigInteger}
	\operation{@View snapshot():IERC721View}
	\end{interface}
	\begin{interface}[text width=11cm]{IERC721}{0, -3}
	\inherit{IERC20View}
	\operation{@FromContract approve(spender:Contract, tokenId:BigInteger)}
	\operation{@View setApprovalForAll(operator:Contract, approved:boolean)}
	\operation{@FromContract transferFrom(from:Contract, to:Contract, tokenId:BigInteger)}
	\operation{@View getApproved(tokenId:BigInteger):Contract}
	\operation{@View isApprovedForAll(owner:Contract, operator:Contract):boolean}
	\operation{view():IERC20View}
	\end{interface}	
	\externalclasscolor
	\begin{abstractclass}[text width=2cm]{Contract}{-5,-6.5}
	\end{abstractclass}
	\classcolor
	\begin{class}[text width=6cm]{ERC721}{0,-7.5}
		\inherit{Contract}
		\implement{IERC721}
		\operation{ERC721(name:String, symbol:String)}
		\operation{@View name(), symbol():String}
		\operation{\# \_mint(to:Contract, tokenId:BigInteger)}
		\operation{\# \_burn(tokenId:BigInteger)}
	\end{class}
	\begin{class}[text width=7cm]{ERC721Burnable}{-4,-10.5}
		\inherit{ERC721}
		\operation{ERC721Burnable(name:String, symbol:String)}
		\operation{@FromContract burn(tokenId:BigInteger)}
	\end{class}
	\ourclasscolor
	\begin{class}[text width=8cm]{CryptoShark}{5,-10.5}
		\inherit{ERC721}
		\operation{@FromContract CryptoShark()}
		\operation{@FromContract mint(account:Contract, tokenId:BigInteger)}
		\operation{@FromContract burn(tokenId:BigInteger)}
	\end{class}
	\interfacecolor
	\begin{interface}[text width=11cm]{IERC721Receiver}{0,-13}
		\operation{onReceive(ledger:IERC721, operator:Contract, from:Contract, tokenId:BigInteger)}
		\end{interface}
	\end{tikzpicture}
  \end{center}
  \caption{The hierarchy of the ERC721 token implementations.}
  \label{fig:erc721_hierarchy}
\end{figure}

As in the case of the ERC20 tokens, the interface
\texttt{IERC721View}\index{IERC721View@{\texttt{IERC721View}}} contains
the read-only operations that implement a ledger of non-fungible
tokens: the \texttt{ownerOf()} method yields the owner of a given token and
the \texttt{balanceOf()} method returns the number of tokens held by a given \texttt{account}.
The \texttt{snapshot()}\index{snapshot()@{\texttt{snapshot()}}} method
yields a frozen, read-only view of the latest state of the ledger.

The \texttt{IERC721}\index{IERC721@{\texttt{IERC721}}} subinterface
adds methods for token transfers.
Please refer to their description given by the Ethereum standard.
We just say here that
the \texttt{transferFrom()} method moves a given token from its previous
owner to a new owner. The caller of this method can be the owner of the
token, but it can also be another contract, called \emph{operator},
as long as the latter has
been previously approved by the token owner, by using the
\texttt{approve()} method. It is also possible to approve an operator for all
one's tokens (or remove such approval), through the \texttt{setApprovalForAll()} method.
The \texttt{getApproved()} method tells who is the operator approved for a given token
(if any) and the \texttt{isApprovedForAll()} method tells if a given operator has been
approved to transfer all tokens of a given \texttt{owner}. The \texttt{view()}\index{view()@{\texttt{view()}}} method
yields a read-only view of the ledger, that reflects all future changes to the ledger.

The implementation \texttt{ERC721}\index{ERC721@{\texttt{ERC721}}} provides standard implementations for all the methods
of the interfaces \texttt{IERC721View} and \texttt{IERC721}, adding metadata information about the name and the
symbol of the token and protected methods for minting and burning
new tokens. These are meant to be called in subclasses, such as
\texttt{ERC721Burnable}\index{ERC721Burnable@{\texttt{ERC721Burnable}}}. Namely,
the latter adds a \texttt{burn()} method that allows the owner of a token
(or its approved operator) to burn the token.

As we have already said in Sec.~\ref{subsec:richer_than_expected},
the owners of the tokens are declared as contracts
in the \texttt{IERC721View} and \texttt{IERC721} interfaces, but the \texttt{ERC721} implementation
actually requires them to be
\texttt{IERC721Receiver}`s\index{IERC721Receiver@{\texttt{IERC721Receiver}}}
or externally owned accounts.
Otherwise, the methods of \texttt{ERC721} will throw an exception.
Moreover, token owners that implement the \texttt{IERC721Receiver} interface
get their \texttt{onReceive()} method called whenever new tokens are transferred to them.

\begin{commentbox}
The ERC721 standard requires \texttt{onReceive()} to return a special message, in order
to prove that the contract actually executed that method. This is a very technical
necessity of Solidity, whose first versions allowed one to call non-existent methods
without getting an error. It is a sort of security measure, since Solidity
has no \texttt{instanceof} operator and cannot check in any reliable
way that the token owners are actually instances of the interface \texttt{IERC721Receiver}.
The implementation in Solidity uses the ERC165
standard~\cite{ReitwiessnerJVBFE18}\index{ERC165} for interface detection,
but that standard is not a reliable replacement of \texttt{instanceof},
since a contract can always pretend to belong to any contract type.
Takamaka is Java and can use the \texttt{instanceof} operator, that works correctly.
As a consequence, the \texttt{onReceive()} method in Takamaka needn't return any value.
\end{commentbox}

\subsection{Implementing our own ERC721 token}\label{subsec:implementing_erc721}

\begin{center}
(See the \texttt{io-hotmoka-tutorial-examples-erc721} project in \texttt{\hotmokaRepo{}})
\end{center}

Let us define a ledger for non-fungible tokens
that only allows its creator the mint or burn tokens.
We will call it \texttt{CryptoShark}. As Fig.~\ref{fig:erc721_hierarchy} shows,
we plug it below the \texttt{ERC721} implementation, so that we inherit that implementation
and do not need to reimplement the methods of the \texttt{IERC721} interface.
The code is almost identical to that for the \texttt{CryptoBuddy} token defined
in Sec.~\ref{subsec:implementing_erc20}.

Create in Eclipse a new Maven Java~21 (or later) project. Use for this project the name
\texttt{io-hotmoka-tutorial-examples-erc721}.
You can do this by duplicating the previous project \texttt{io-hotmoka-tutorial-examples-family}.
Use the following \texttt{pom.xml}:
%
\begin{codebox}\begin{xmllst}
<project xmlns="http://maven.apache.org/POM/4.0.0"
    xmlns:xsi="http://www.w3.org/2001/XMLSchema-instance"
    xsi:schemaLocation="http://maven.apache.org/POM/4.0.0 http://maven.apache.org/xsd/maven-4.0.0.xsd">

  <modelVersion>4.0.0</modelVersion>
  <groupId>io.hotmoka</groupId>
  <artifactId>io-hotmoka-tutorial-examples-erc721</artifactId>
  <version>|\hotmokaVersion{}|</version>

  <properties>
    <project.build.sourceEncoding>UTF-8</project.build.sourceEncoding>
    <maven.compiler.release>21</maven.compiler.release>
  </properties>

  <dependencies>
    <dependency>
      <groupId>io.hotmoka</groupId>
      <artifactId>io-takamaka-code</artifactId>
      <version>|\takamakaVersion{}|</version>
    </dependency>
  </dependencies>

  <build>
    <plugins>
      <plugin>
        <groupId>org.apache.maven.plugins</groupId>
        <artifactId>maven-compiler-plugin</artifactId>
        <version>3.11.0</version>
      </plugin>
    </plugins>
  </build>

</project>
\end{xmllst}\end{codebox}
%
and the following \texttt{module-info.java}:
%
\begin{codebox}\begin{javalst}
module erc721 {
  requires io.takamaka.code;
}
\end{javalst}\end{codebox}
%
Create package \texttt{io.hotmoka.tutorial.examples.erc721}
inside \texttt{src/main/java} and add
the following \texttt{CryptoShark.java} source inside that package:
%
\begin{codebox}\begin{javalst}
package io.hotmoka.tutorial.examples.erc721;

import static io.takamaka.code.lang.Takamaka.require;

import java.math.BigInteger;

import io.takamaka.code.lang.Contract;
import io.takamaka.code.lang.FromContract;
import io.takamaka.code.tokens.ERC721;

public class CryptoShark extends ERC721 {
  private final Contract owner;

  public @FromContract CryptoShark() {
    super("CryptoShark", "SHK");
    owner = caller();
  }

  public @FromContract void mint(Contract account, BigInteger tokenId) {
    require(caller() == owner, "Lack of permission");
    _mint(account, tokenId);
  }

  public @FromContract void burn(BigInteger tokenId) {
    require(caller() == owner, "Lack of permission");
    _burn(tokenId);
  }
}
\end{javalst}\end{codebox}
%
The constructor of \texttt{CryptoShark} takes note of the creator of the token.
That creator is the only that is allowed to mint or burn tokens, as
you can see in methods \texttt{mint()} and \texttt{burn()}.

You can compile the file:
%
\begin{shellcommandbox}\begin{ttlst}
cd io-takamaka-code-examples-erc721
mvn clean install
cd ..
\end{ttlst}\end{shellcommandbox}
%
Then you can install that jar in the node and create an instance of the token
exactly as we did for the \texttt{CryptoBuddy} ERC20 token in Sec.~\ref{subsec:implementing_erc20}.
\chapter{Hotmoka blockchains}\label{ch:hotmoka_blockchains}

The experiments of the previous chapters have been performed on a
Hotmoka node already existing online, part of a blockchain installed by somebody else.
Namely, we have used the Hotmoka node at \url{\serverMokamint}, that is part
of a blockchain based on a proof of space consensus algorithm~\cite{Spoto25}\index{proof of space},
and the Hotmoka node at \url{\serverTendermint}, that is part of
another blockchain, this time based on a proof of stake consensus algorithm~\cite{Kwon14}\index{proof of stake}.
Both are able to execute Hotmoka transactions, store smart contracts and interact with them,
but they are different for the way they create new blocks and
synchronize their state among all the nodes
of the blockchain they belong to. Namely, in the case of proof of space, each node
of the blockchain increases the likelihood of winning the right to create a new block
if it allocates a larger data structure on disk (typically, on SSD) than the other nodes of the
same blockchain. Instead, in the case of proof of stake, only a restricted, modifiable set of nodes
(called \emph{validators}\index{validator}) has the right to create new blocks,
in a round-robin way, and the prize earned for the creation
of a new block is proportional to the shares of validation power\index{validation power}
owned by each validator. Other nodes can just verify the transactions.
We do not enter into the (sometime religious) discussion about which consensus algorithm
is the best. Probably none of them, as each of them (and others, that we do not even speak about here)
has its advantages and drawbacks. What is important to highlight is that Hotmoka is not
bound to a specific consensus algorithm, but is rather a software layer for the execution
of smart contract transactions, that can be plugged on top of different consensus engines,
giving rise to distinct blockchains with distinct consensus algorithms.

This chapter shows how to build new blockchains or connect to existing blockchains,
both for proof of space and for proof of stake consensus. It will present this by using docker,
so that you do not need to install anything on your machine beyond docker itself.
Namely, there exist preconfigured docker\index{docker} images that allow one to run Hotmoka nodes of
a blockchain without preliminary, complex configuration. They are meant to simplify the
life of users who just want to install nodes with a standard procedure that should cover
most typical use cases. For less standard use cases, such docker images can be partially configured
or Hotmoka nodes can be installed with Java code, as it will be shown later in Ch.~\ref{ch:hotmoka_nodes}.

\section{Hotmoka blockchains based on Mokamint (proof of space)}\label{sec:hotmoka_mokamint}

Hotmoka can run on top of a generic engine for proof of space consensus, called Mokamint~\cite{mokamint_github}.
The latter implements the networking and consensus layers of a blockchain, leaving the application
layer generic, so that it can be plugged on top of it. Hotmoka is one such application layer,
but others may exist as well. Mokamint implements a consensus algorithm based on proof of space,
derived from that of Burstcoin/Signum~\cite{Spoto25}. In such algorithm, the creation of a new
block required to identification of a data structure, called \emph{nonce}, from a set of
nonces precomputed on disk, in order to minimize the waiting time for the creation of the next block.
The more nonces are precomputed on disk, the higher the likelihood of winning the competition for
the creation of the next blocks. Nonces could in principle be recomputed on the fly at each
block creation, but their computation is so complex that it becomes impossible to perform it
in the short time between the creaton of successive blocks. From this point of view, proof of space
is similar to Bitcoin's proof of work~\cite{Nakamoto08}, but the resource used for the competition
is different: SSD instead of CPU. Consequently, proof of space has a minimal energy cost and can be
run a cheap hardware as well. In particular, you can mine for Mokamint-based blockchains with a desktop machine
or even with a mobile phone or tablet (see the Mokaminter Android app~\cite{mokaminter_google_play}).
This becomes possible for Hotmoka as well, if run on top of Mokamint, as shown in this section.

\begin{figure}[th]
  \begin{center}
    \begin{tikzpicture}[scale=1.1]
      \draw [fill=lightviolet,thick] (0,1) rectangle +(3,1.5);
      \draw[-, thick,dashed] (1.5,0.25) -- (1.5,1);
      \draw [fill=yellow,thick] (0,-0.75) rectangle +(3,1);
      \draw (1.5,-0.25) node {{\scriptsize peer1 (Hotmoka)}};
      \draw (1.5,2) node {{\scriptsize peer1 (Mokamint)}};
      \draw (1.5,1.5) node {{\scriptsize 8dE4pHgtvFAXgB\ldots}};
      \draw [fill=lightviolet,thick] (0,4) rectangle +(3,1.5);
      \draw[-, thick,dashed] (1.5,5.5) -- (1.5,6.25);
      \draw [fill=yellow,thick] (0,6.25) rectangle +(3,1);
      \draw (1.5,6.75) node {{\scriptsize peer2 (Hotmoka)}};
      \draw (1.5,5) node {{\scriptsize peer2 (Mokamint)}};

      \draw (4.3,6.75) node {\leftplug};
      \draw (4.3,6.25) node {{\scriptsize remote client}};
      \draw[-, thick] (3,6.75) -- (4,6.75);
      \draw (5.5,6.75) node {{\scriptsize port 80}};
      
      \draw (1.5,4.5) node {{\scriptsize AhYvJj737BtY24\ldots}};

      \draw [fill=lightred,thick] (4,4.85) rectangle +(4,0.8);
      \draw (5.5,5.25) node {{\scriptsize GaWKnFs1s2syow\ldots}};
      \draw (6,5.9) node {{\scriptsize local miner}};
      \draw (7.4,5.25) node {\ssd};
      \draw[-, thick] (3,5.25) -- (4,5.25);

      \draw (4.3,4.1) node {\leftplug};
      \draw (4.3,3.6) node {{\scriptsize remote miner}};
      \draw[-, thick] (3,4.1) -- (4,4.1);
      \draw (5.6,4.1) node {{\scriptsize port 8026}};

      \draw [fill=lightred,thick] (4,2) rectangle +(4,0.8);
      \draw (6,3.1) node {{\scriptsize local miner}};
      \draw (5.5,2.4) node {{\scriptsize FbXZjuFsZvQ1Ae\ldots}};
      \draw (7.4,2.4) node {\ssd};
      \draw[-, thick] (3,2.4) -- (4,2.4);

      \draw[-, thick, dashed] (1.5,2.5) -- (1.5,4);
      \draw (0.8,2.7) node {{\scriptsize port 8030}};
      \draw (0.8,3.7) node {{\scriptsize port 8032}};

      \draw (4.3,1.1) node {\leftplug};
      \draw (4.3,0.6) node {{\scriptsize remote miner}};
      \draw[-, thick] (3,1.1) -- (4,1.1);
      %\visible<1>{\draw (5.6,1.1) node {{\scriptsize port 8025}};}
      
      \draw (4.3,-0.2) node {\leftplug};
      \draw (4.3,-0.7) node {{\scriptsize remote client}};
      \draw[-, thick] (3,-0.2) -- (4,-0.2);
      \draw (5.6,-0.2) node {{\scriptsize port 8001}};

      \draw [fill=verylightgray,dashed] (6.3,0.2) rectangle +(6,1.6);
      \draw[-, thick, dashed] (4.8,1.1) -- (6.4,1.1);
      \draw (5.6,1.4) node {{\scriptsize port 8025}};
      \draw[-, thick] (7.15,1.1) -- (8.1,1.1);
      \draw [fill=lightred,thick] (8.2,0.7) rectangle +(4,0.8);
      \draw (6.8,1.1) node {\rightplug};
      \draw (10.1,0.4) node {{\scriptsize local miner in our machine}};
      \draw (9.5,1.1) node {{\scriptsize our public key}};
      \draw (11.6,1.1) node {\ssd};
    \end{tikzpicture}
  \end{center}
  \caption{The network architecture of a couple of Hotmoka nodes based on Mokamint.}\label{fig:hotmoka_mokamint_network}
\end{figure}

Fig.~\ref{fig:hotmoka_mokamint_network} shows the network architecture of a couple
of Hotmoka nodes based on Mokamint, that are two peers of the same blockchain. The two Mokamint instantiations
communicate between each other, in order to exchange blocks and synchronize their copy of the blockchain.
Each Mokamint instantiation runs an instance of Hotmoka, that specifies the semantics of the transactions
that can be executed on the blockchain. Each peer exports the node interface (Fig.~\ref{fig:node_hierarchy}),
respectively, in this example, at port 80 and 8001. Remote clients, such as moka or Mokito, can connect to such
ports and run transactions on the nodes, that will be automatically executed in all peers.
Fig.~\ref{fig:hotmoka_mokamint_network} shows that each peer runs a local miner\index{miner!local}, that is, a software
tool that finds nonces on disk in order to create the next blocks. It is well possible to have peers
without any miner, in which case they just synchronize their copy of the blockchain but do not contribute
new blocks. However, for simplicity, our docker images will always provide a local miner to a node.
As the same figure shows,
some peer might allow the connection of \emph{remote miners}\index{miner!remote},
that is, miners, identical to the local ones,
but running as a remote network service. This allows peers to exploit the willingness of remote users
to provide mining power to them by just plugging remote miners, without running a full node. If such remote miners
contribute to the creation of new blocks,
the prize for this creation will be shared between the peer and the remote miners.

\subsection{Start a remote miner for an existing node}\label{subsec:mining_hotmoka_mokamint}

The simplest scenario is that of a user who wants to provide mining power to an existing node of a blockchain
based on Mokamint, such as \url{\serverMokamint}. This user does not want to run a full node of the blockchain,
for instance because he does not want to perfom a long synchronization, nor hold a copy of the full blockchain
in his machine, or maybe because his internet connection is not stable enough for running a full node, or
just because he does not want to keep his machine always on, day and night. This user can contribute
to the creation of new blocks, and earn crypto for this, by running a remote miner
for an existing Hotmoka node based on Mokamint (Fig.~\ref{fig:hotmoka_mokamint_network}),
which can be done in two ways. The simplest is to run the
Mokaminter Android app~\cite{mokaminter_google_play} on a mobile phone or tablet. This should be so
simple that we do not really describe it here: just install the app and try starting mining. The other possibility
is to mine on a local machine, and we describe it here. In both cases, the miner is identified by a key pair,
that is where the crypto earned for mining gets sent. The key is created automatically by Mokaminter,
while it must be created manually when mining on a local machine.

There is a docker image for running a remote miner for Mokamint~\cite{mokamint_dockerhub}.
You first need to create a key pair that will identify you as a miner of the blockchain, then create a mining configuration,
compatible with the Mokamint node that you are going to mine for, that includes a plot file for mining.
The larger this plot file, the more likely it is that the miner will be selected for creating the next blocks
of the blockchain and more coins will be earned by your miner, on average.
For instance, in order to mine for the Mokamint node running the Hotmoka testnet
and published at \url{\serverMokamintMining}, you start with the creation of the key pair
that will identify you as a miner:
%
\inputCommand{docker_create_keys_for_mining}
\inputOutput{docker_create_keys_for_mining}
%
\begin{commentbox}
We create the key pair inside a docker container above and then export it outside.
Thus is because, in this chapter, we perform everything with the docker tool only.
You can of course do the same by invoking \texttt{moka keys create} directly if you
have installed on your local machine (Sec.~\ref{subsubsec:moka_local}).
\end{commentbox}
%
The command above asks for a password, that you will use to control the key pair,
it creates the key pair in your local file system, named \texttt{miner.pem},
and prints its public key.
Write down the password on paper, since it is not possible to recover it if lost.
Save \texttt{miner.pem} in a safe place as well.

You can create the mining configuration now:
%
\index{docker!docker config\_miner@{\texttt{config\_miner}}}
\inputCommand{docker_config_miner}
%
You can replace 5000 with any other positive number of nonces in order to increase or decrease the size of the plot file.

After a mining configuration has been created, you can start mining:
%
\index{docker!docker mine@{\texttt{mine}}}
\inputCommand{docker_mine}
%
You should see on the screen the flow of the logs of the miner.
You can terminate the miner by pressing the ENTER key.
Or you can leave the miner working in the background
by pressing CTRL+p followed by CTRL-q, as always with docker.

You can stop an already running miner by simply stopping its docker container (\texttt{docker stop miner}).
You can restart a stopped miner by running the above \texttt{mine} command again.
You normally do not recreate the mining configuration when you restart a miner: its creation is a one-shot action.
Nevertheless, you can recreate the configuration if, for instance, you want to change the size of the plot file.
For that, stop your miner (if running) and just rerun the docker command for \texttt{config-miner}.
Once the mining configuration has been recreated, you can restart your miner with \texttt{mine}.

Your miner will earn coins for the mining activity. You can check the current balance of the public key
that identifies your miner, as follows:
%
\index{docker!docker balance@{\texttt{balance}}}
\inputCommand{docker_check_balance}
%
When the balance will finally be non-zero, you can bind the public key to the actual account that holds its crypto coins, with
%
\index{moka!keys bind@{\texttt{keys bind}}}
\inputCommand{moka_keys_bind_miner}
%
that requires you to have \texttt{moka} installed in your machine (Sec.~\ref{subsubsec:moka_local})
and to be run in the directory where you have saved the key pair file \texttt{miner.pem}.

\subsection{Start a node of an existing blockchain}\label{subsec:join_hotmoka_mokamint}

This is the most typical situation. Namely, this occurs if you want to join an existing blockchain,
with a node that mines new blocks and receives blocks created by the other peers.
This docker image provides a script for this situation. This script includes the creation of
a local miner as well, or otherwise your node would not be able to mine new blocks.
Because of this, the scripts deal with two key pairs: the former identifies the node,
and is kept inside the machine running the script, and the latter identifies the miner,
can be stored elsewhere and only its public key is needed here.

The process is consequently split in two:
%
\begin{itemize}	
\item configure the node (\texttt{config-clone});
\item run the node (\texttt{go}).
\end{itemize}
%
Each phase is the execution of a script inside this docker image. The script
\texttt{config-clone} is meant to be run only once, while \texttt{go} can be run, stopped and run again,
whenever you want to start, stop and restart a node. You can also pause it and unpause it.
The reason for splitting the process in two scripts is that it allows one
to manually edit the configuration created by \texttt{config-clone} before running the node,
although we will not show this here. Moreover, having distinct scripts allows \texttt{go}
to be stopped and run again, repeatedly, whenever you want to stop and restart a node.

The following instructions assume that you have a reliable internet connection.
If the connection is too slow, or flickering, or if it disconnects for some time,
the synchronization of the node will likely fail. In that case, stop and restart \texttt{go} later, hoping
to complete the synchronization, eventually. Please consider that the synchronization of the node
requires to download all the history of the blockchain that you are joining, which might take time
(hours, days, even weeks, depending on the age of the blockchain and the quality of the internet
connection).

\subsubsection{Configure the node (\texttt{config-clone})}\label{subsubsec:mokamint_config_clone}

The \texttt{config-clone} script creates the configuration directory of a node
that joins an existing Hotmoka blockchain based on the Mokamint proof of space engine.
For that, you must specify the URI of a node of this blockchain, from where
the configuration information will be fetched. This is important, since all nodes of the
same blockchain must have the same consensus parameters.

The first thing to do is to create a key pair for the miner of the new node
that you want to start. You can do this as in Sec.~\ref{subsec:mining_hotmoka_mokamint},
hence generating \texttt{miner.pem}.

You can then run the script that configures the node. Use for it the base58-encoded public key
of the key pair that you have created above. Specify \url{\serverMokamintPublic}
as the URI of a node of the blockchain to join.
Specify the size of the plot file to use for the proof of space:
the larger, the more blocks will be created by your node, but also more disk space
will be allocated for mining. We will use two volumes: \texttt{chain} will contain
the actual blockchain data and \texttt{hotmoka\_mokamint} will contain the configuration
information created for the node. By using volumes, we can share that information
across successive invocations of docker:
%
\inputCommand{docker_mokamint_config_clone}
%
Note that the script above will create another key pair (and ask you about the relative password),
that will be kept \emph{inside} the container. This key pair identifies the node
and will be used to sign the blocks that the node will create. It must remain inside the container,
although you may want to extract a copy from the container to your local host.

\subsubsection{Run the node (\texttt{go})}\label{subsubsec:mokamint_go}

After configuring the node, you can run it with the go script:
%
\inputCommand{docker_mokamint_go}
%
The command above will start a node of the same blockchain of \url{\serverMokamint},
that allows connections to the ports:
%
\begin{description}
\item[8001:] this is the port where the Hotmoka node is published by default;
	it can be used to contact the node, install and run smart contracts;
\item[8025:] this is the port where mining services can connect by default;
	mining services help your node produce new blocks; note that this requires to open
	a remote miner in your node, listening at port 8025 (\texttt{mokamint-node miners add});
\item[8030:] this is the port where Mokamint can be reached for public queries, by default;
\item[8031:] this is the port where Mokamint can be reached for restricted operations, by default;
	note the we have restricted its access to localhost only,
	since we do not want our Mokamint node to be freely reconfigured remotely.
\end{description}
%
After the command above, you should see that the node will start up and begin synchronizing
from \url{\serverMokamintPublic}. This will take some time (hours, days, weeks\ldots) depending
on the age of the cloned blockchain and on the speed of your internet connection.
You can leave the container in the backrground by entering CTRL+p, CTRL+q, as always in docker.

You can then monitor the progress of the synchronization by entering the running container
and executing the \texttt{mokamint-node} command:
%
\begin{shellcommandbox}\begin{ttlst}
docker exec -it hotmoka /bin/bash
\end{ttlst}\end{shellcommandbox}
%
and then inside the container:
%
\begin{shellcommandbox}\begin{ttlst}
hotmoka@e41eda9afd3b: mokamint-node chain ls 10
\end{ttlst}\end{shellcommandbox}
%
You can also see the manifest of the node, that is identical to that of any other node of the joined blockchain:
%
\begin{shellcommandbox}\begin{ttlst}
hotmoka@e41eda9afd3b: moka nodes manifest show
\end{ttlst}\end{shellcommandbox}
%
Remember that you can always check the logs of a running docker container, for instance with:
%
\begin{shellcommandbox}\begin{ttlst}
docker logs -f hotmoka
\end{ttlst}\end{shellcommandbox}

\subsection{Start the first node of a new blockchain}\label{subsec:start_hotmoka_mokamint}

This situation is much rarer. It occurs when you want to start a brand new blockchain from scratch,
by minting its genesis block and initializing its store. Later, other nodes can join the new blockchain
with the technique described in Sec.~\ref{subsec:join_hotmoka_mokamint}.

The docker image of Hotmoka provides scripts for starting a brand new blockchain.
These scripts include the creation of a local miner as well, or otherwise your node
would not be able to mine new blocks. Because of this, the scripts deal with two key pairs:
the first identifies the node, and is kept inside the machine running the script,
and the second identifies the miner, can be stored elsewhere and only
its public key is needed for starting the node.

The process is consequently split in three:
%
\begin{itemize}
\item configure the node (\texttt{config-new})
\item initialize the node (\texttt{init})
\item run the node (\texttt{go})
\end{itemize}
%
Each phase is the execution of a script inside the docker image. The scripts \texttt{config-new}
and \texttt{init} are meant to be run only once, while \texttt{go} can be run, stopped and run again,
whenever you want to start or stop a node. You can also pause it and unpause it.
The reason for splitting the process in three scripts is that it allows one to manually edit
the configuration created by \texttt{config-new} before initializing and running the node,
although we will not show this here. Moreover, having distinct scripts allows \texttt{go}
to be stopped and run again, repeatedly, whenever you want to stop and restart a node.

\subsubsection{Configure the node (\texttt{config-new})}\label{subsubsec:mokamint_config_new}

The configuration requires the creation of a brand new key pair that will identify the miner of the node.
You can do this as in Sec.~\ref{subsec:mining_hotmoka_mokamint},
hence generating \texttt{miner.pem}.
Moreover, you will need another key pair, for the gamete\index{gamete} account.
This is an account of Hotmoka that holds all cryptocurrency minted at start-up (if any).
It can also be used as faucet of the node, if your node allows a free faucet.
In any case, you need a key pair for the gamete account as well.
Therefore, follow the instructions in Sec.~\ref{subsec:mining_hotmoka_mokamint} and create
\texttt{gamete.pem} as well:
%
\inputCommand{docker_create_keys_for_mokamint_gamete}
\inputOutput{docker_create_keys_for_mokamint_gamete}
%
You can now configure the node, by specifying the public key of the miner and that of the gamete:
%
\inputCommand{docker_mokamint_config_new}
%
Note that the public key of the miner is reported in base58,
while that of the gamete is reported in base64, currently.
The target block creation time, in milliseconds, is the average time between the creation of two successive blocks.
The chain identifier identifies the new network and must be used, for instance,
in the transaction requests sent to the Hotmoka nodes of the network.
The script above will prompt for the password of the key pair used for signing the new blocks, that it creates.
Enter your chosen password or leave it blank. The script will configure the node and create a plot file for its miner.

\subsubsection{Initialize the node (\texttt{init})}\label{subsubsec:mokamint_init_node}

The initialization of the node consists in the execution of a few initial transactions
that create the genesis block, the manifest and the gas station of the node. You can do this with:
%
\inputCommand{docker_mokamint_init}
%
This will take a few seconds. You should see the logs of the executed transactions,
until the script terminates.

\subsubsection{Run the node (\texttt{go})}\label{subsubsec:mokamint_go_new_node}

Once the node has been configured and initialized, you can run it.
Follow for this the instructions reported in Sec.~\ref{subsec:join_hotmoka_mokamint}
for the \texttt{go} script.

\section{Hotmoka blockchains based on Tendermint (proof of stake)}\label{sec:hotmoka_tendermint}

Tendermint~\cite{Tendermint}\index{Tendermint}, now Ignite, is a
Byzantine-fault tolerant engine for building blockchains, that
replicates a finite-state machine on a network of nodes across the world.
The finite-state machine is often referred to as a \emph{Tendermint app}.
The nice feature of Tendermint is that it takes care of all
issues related to networking and consensus, leaving to the
developer only the task to develop the Tendermint app.

\begin{figure}[th]
  \begin{center}
    \myincludegraphics{0.8\textwidth}{hotmoka_tendermint}
  \end{center}
  \caption{The architecture of the Hotmoka node based on Tendermint.}
  \label{fig:hotmoka_tendermint}
\end{figure}

There is a docker image that implements a Hotmoka node as a Tendermint app
for programming in Takamaka over Tendermint. We have already used that node
previously, since that installed at
\texttt{\serverTendermint{}} is a node of that type.
Fig.~\ref{fig:hotmoka_tendermint}
shows the architecture of a Tendermint Hotmoka node.
It consists of a few components.
The Hotmoka component is the Tendermint app that
implements the transactions on the state, that is the installation
of jars and the execution of code written in the Takamaka subset of Java. This part is the same in every
implementation of a Hotmoka node, not only for this one based on Tendermint.
The database that contains the state is implemented by
using the Xodus transactional database by IntelliJ.
What is specific here, however, is that transactions are put inside a blockchain
implemented by Tendermint. The communication occurs, internally, through the two TCP ports
26657 and 26658, that are the standard choice of Tendermint for communicating with an app.
Clients can contact the Hotmoka node
through any port, typically but not exclusively~8001 or~8002, as a websocket service.
The node can live alone but is normally integrated with other Hotmoka nodes based on Tendermint, so that
they execute and verify the same transactions, reaching the same state at the end. This happens through
the TCP port 26656, that allows Tendermint instances to \emph{gossip}:
they exchange transactions and information on peers and finally reach consensus.
Each node can be configured to use a different port to communicate with clients,
which is useful if, for instance, ports 8001 or 8002 (or both)
are already used by some other service.
Port 26656 must be the same for all nodes in the network, since they must communicate on
a standard port.

We can use \texttt{\serverTendermint{}} to play with
accounts and Takamaka contracts. However, we might want to
install our own node, part of the same blockchain network of \texttt{\serverTendermint{}}
or part of a brand new blockchain. In the former case, our own node will execute
the same transactions of \texttt{\serverTendermint{}}, so that we can be sure that they are
executed according to the rules. In the latter case, we can have our own blockchain that
executes our transactions only, instead of using a shared blockchain such as that
at \texttt{\serverTendermint{}}.

\subsection{Start a node of an existing blockchain}\label{subsec:join_hotmoka_tendermint}

As for Hotmoka nodes based on Mokamint~\ref{subsec:join_hotmoka_mokamint}, this task is split in two scripts,
\texttt{config-clone} and \text{go}.

\subsubsection{Configure the node (\texttt{config-clone})}\label{subsubsec:tendermint_config_clone}

The \texttt{config-clone} script creates the configuration directory of a node
that joins an existing Hotmoka blockchain based on the Tendermint byzantine
consensus engine. For that, you must specify the URI of a node of this blockchain,
from where the configuration information will be fetched, such as
\url{\serverTendermint}. This is important, since all nodes of the
same blockchain must have the same consensus parameters.
You must also specify the average block creation rate of the blockchain,
that in the case of \url{\serverTendermint} is $10,000$ milliseconds.
We will use two volumes: \texttt{chain} will contain the actual blockchain data
and \texttt{hotmoka\_tendermint} will contain the configuration information
created for the node. By using volumes, we can share that information
across successive invocations of docker.

You can then run the script that configures the node.
%
\inputCommand{docker_tendermint_config_clone}
%
The script above will create a key pair with empty password, that will be kept inside the container.
This key pair identifies the node and will be used to sign the blocks that the node will create,
if it will ever become a validator of the network. It must remain inside the container,
although you may want to extract a copy from the container to your local host.

\subsubsection{Run the node (\texttt{go})}\label{subsubsec:tendermint_go}

After configuring the node, you can run it with the go script:
%
\inputCommand{docker_tendermint_go}
%
The command above will start a node of the same blockchain of \url{\serverTendermint},
that allows connections to the ports:
%
\begin{description}
\item[8001:] this is the port where the Hotmoka node is published by default;
	it can be used to contact the node, install and run smart contracts;
\item[26656:] this is the port where the Tendermint engine communicates with its peers.
\end{description}
%
After the command above, you should see that the node will start up and begin synchronizing
from \url{\serverTendermint}. This will take some time (hours, days, weeks\ldots) depending
on the age of the cloned blockchain and on the speed of your internet connection.
You can leave the container in the backrground by entering CTRL+p, CTRL+q, as always in docker.

You can then enter the running container and check the manifest of the node,
that is identical to that of any other node of the joined blockchain:
%
\begin{shellcommandbox}\begin{ttlst}
docker exec -it hotmoka /bin/bash
\end{ttlst}\end{shellcommandbox}
%
and then inside the container:
%
\begin{shellcommandbox}\begin{ttlst}
hotmoka@e41eda9afd3b: moka nodes manifest show
\end{ttlst}\end{shellcommandbox}
%
Remember that you can always check the logs of a running docker container, for instance with:
%
\begin{shellcommandbox}\begin{ttlst}
docker logs -f hotmoka
\end{ttlst}\end{shellcommandbox}

The resulting node will not be a validator. For that, you need to buy some validation power
from a validator who is willing to sell it to you.
Check the \texttt{moka nodes tendermint validators}\index{moka!nodes tendermint validators@{\texttt{nodes tendermint validators}}} commands.

\subsection{Start a node of a new blockchain}\label{subsec:start_hotmoka_tendermint}

As in Sec.~\ref{subsec:start_hotmoka_mokamint}, this task is split in three scripts,
\texttt{config-new}, \texttt{init} and \texttt{go}.

\subsubsection{Configure the node (\texttt{config-new})}\label{subsubsec:tendermint_config_new}

You will need a key pair, for the gamete\index{gamete} account.
This is an account of Hotmoka that holds all cryptocurrency minted at start-up (if any).
It can also be used as faucet of the node, if your node allows a free faucet.
You can follow the instructions in Sec.~\ref{subsec:mining_hotmoka_mokamint} and create
\texttt{gamete.pem} as well:
%
\inputCommand{docker_create_keys_for_tendermint_gamete}
\inputOutput{docker_create_keys_for_tendermint_gamete}
%
You can now configure the node, by specifying the public key of the gamete:
%
\inputCommand{docker_tendermint_config_new}
%
Note that the public key of the gamete is reported in base64, currently.
The target block creation time, in milliseconds, is the average time between
the creation of two successive blocks. The chain identifier identifies
the new network and must be used, for instance, in the transaction
requests sent to the Hotmoka nodes of the network.

\subsubsection{Initialize the node (\texttt{init})}\label{subsubsec:tendermint_init_node}

The initialization of the node consists in the execution of a few initial transactions
that create the genesis block, the manifest and the gas station of the node. You can do this with:
%
\inputCommand{docker_tendermint_init}
%
This will take a few seconds. You should see the logs of the executed transactions,
until the script terminates.

\subsubsection{Run the node (\texttt{go})}\label{subsubsec:tendermint_go_new_node}

Once the node has been configured and initialized, you can run it.
Follow for this the instructions reported in Sec.~\ref{subsec:join_hotmoka_tendermint}
for the \texttt{go} script.

\subsection{Manifest and validators}\label{subsec:manifest_and_validators}

The information reported by \texttt{moka nodes manifest show} refers to two accounts that have been
created during the initialization of the node:
%
\begin{shellbox}\begin{ttlst}
gamete: fcc6ad9a4cd4109dcb554df6f1d951f770d42cdb4c2caeb8fa713c1d4189b4fa#0
  balance: 1000000000000000000000000000000000
  maxFaucet: 0

validator #0: b437832a688dbb89b145d380b7cb3eb841d7cb09fb600d27be43cd670e8b43f9#0
  id: 030203B36BA4EDF0182D7D40D7EA7FE34A9415B4
  balance: 1568
  staked: 4704
  power: 1000000
\end{ttlst}\end{shellbox}
%
We already know the first one, that is, the gamete\index{gamete}.
Its private key is not stored in the docker container
but must be available to the person who started the container.
Normally, it is the key that was created before starting the node (with \texttt{moka keys create}
or similar) and
that is later bound to the storage address of the gamete (with \texttt{moka keys bind}). If you
followed the instructions in the previous sections, you should have a
\texttt{gamete.pem} file in your file system, for the gamete. With that pem file,
you have \emph{superuser} rights,
in the sense that you can, for instance,
open and close the faucet\index{faucet} (but only if you configured the new blockchain
with the \texttt{ALLOWS\_UNSIGNED\_FAUCET} option set to true).
Moreover, the gamete owns all cryptocurrency initially minted for the node (if any).
With that, you can create and fund as many new accounts
as you want and in general run any transaction you like.

There is a second account that has been created. Namely, the \emph{validator}\index{validator} account.
This is an externally owned account that gets remunerated for every
non-\texttt{@View}\index{View@{\texttt{View}}}
transaction run in the node and included in blockchain. At
the beginning, the balance of a validator is 0. This increases for each non-\texttt{@View}
transaction executed in the blockchain. Namely, the gas
consumed for such transactions gets forwarded to the validators of the blockchain, at its current price,
in proportion to their validation power\index{validation power}.
If there is only a single validator, everything goes to it.
It is important to note that only a portion of this prize lands, immediately, in full,
in the balance of the validators: the rest is \emph{staked}\index{staking}
for each validator, that is, kept in the validators contract
as a motivation for the validators to behave correctly. In the future, if a validator misbehaves
(that is, if it does not validate the transactions correctly or does not validate them at all) then
its stake will be reduced by a percent that is called \emph{slashing}\index{slashing}.
This is by default $1\%$ for validators that do not validate correctly and $0.5\%$ for
validators that do not validate at all (for instance, they are down).
The staked amount will be forwarded to a validator only when it will sell all its
validation power to another validator and stop being a validator.

The power of a validator expresses also the weight of its votes: in order to validate a transaction,
at least two thirds of the validation power must agree on the outcome of the transaction, or
otherwise the network will hang.
This mechanism is inherited from the underlying Tendermint engine and might be different
in other Hotmoka nodes in the future. However, the idea that power represents the weight
of a validator will likely remain.

Validators also receive a prize at the end of the creation of each block of the blockchain,
when coins are minted and distributed to the validators in proportion to their validation power.

Therefore, validator accounts receives payments for the validation of transactions
and the creation of new blocks.
But who controls this validators? It turns out that the \texttt{config-init} command
has created the key pair of the validator inside the configuration directory of the node.
You can see it if you access the \texttt{hotmoka\_tendermint} volume where the container operates.
You must be root to do that:
%
\begin{shellcommandbox}\begin{ttlst}
sudo ls /var/lib/docker/volumes/hotmoka_tendermint/_data
\end{ttlst}\end{shellcommandbox}
%
\begin{shellbox}\begin{ttlst}
consensus_config.toml  local_config.toml  tendermint_config  validator.pem
\end{ttlst}\end{shellbox}
%
In alternative, you can use the \texttt{docker exec} command to run a command inside the container.
You do not need to be root, but need to know the id of the running container (\texttt{docker ps} might help you):
%
\begin{shellcommandbox}\begin{ttlst}
docker exec c1407e499ad67465318704da1fcb6e9b88ee94faceb7c4b86e00ab4775590b3f /bin/ls hotmoka_tendermint
\end{ttlst}\end{shellcommandbox}
%
\begin{shellbox}\begin{ttlst}
consensus_config.toml
local_config.toml
tendermint_config
validator.pem
\end{ttlst}\end{shellbox}
%
Who owns that key controls the validator. Therefore, you might wanto to move it in your host machine:
%
\begin{shellcommandbox}\begin{ttlst}
docker cp c1407e499:/home/hotmoka/hotmoka_tendermint/validator.pem .
\end{ttlst}\end{shellcommandbox}
%
\begin{shellbox}\begin{ttlst}
Successfully copied 2.05kB
\end{ttlst}\end{shellbox}

As a final remark about the key of the validator, note that it \emph{must} be the same
key that the underlying Tendermint engine uses in order to identify the node in the network
and vote for validation. If that is not the case, the validator account in the
manifest will not be recognized as a working validator and will be slashed for
not behaving. Eventually, it will be expulsed from the set of validators.
Tendermint stores the key that it uses to identify the node in another file, inside
its configuration, and in JSON format:
%
\begin{shellcommandbox}\begin{ttlst}
docker exec c1407e499 /bin/ls hotmoka_tendermint/tendermint_config/config
\end{ttlst}\end{shellcommandbox}
%
\begin{shellbox}\begin{ttlst}
config.toml
genesis.json
node_key.json
priv_validator_key.json
\end{ttlst}\end{shellbox}
%
This file must remain in the node, or otherwise Tendermint cannot vote for validation.
The docker \texttt{config-clone} and \texttt{config-new} scripts magically ensure that,
correctly, this file contains the same key
as \texttt{validator.pem}, although in a different format.

\subsection{Become a validator}\label{subsec:become_validator}

The configuration of a Hotmoka Tendermint node, created thorugh the \texttt{config-clone}
or \texttt{config-new} commands,
contains a \texttt{validator.pem} key pair file, as said above. You can use that file to create an actual
account object, a candidate for becoming a validator of the network. You can do this with the
\texttt{moka nodes tendermint validators create}\index{moka!nodes tendermint validators create@{\texttt{nodes tendermint validators create}}}
command, which is similar to
\texttt{moka accounts create} but creates instances of
\texttt{TendermintED25519Validator}\index{TendermintED25519Validator@{\texttt{TendermintED25519Validator}}}.
Once you have a validator object for your \texttt{validator.pem}, you can become an actual validator
when one of the existing validators decides to sell (part of) its shares and you buy it.
The seller validator creates a sale offer with the
\texttt{moka nodes tendermint validators sell}\index{moka!nodes tendermint validators sell@{\texttt{nodes tendermint validators sell}}}
command. The sale offer will be visible in the manifest of the node and can be bought with
the \texttt{moka nodes tendermint validators buy}\index{moka!nodes tendermint validators buy@{\texttt{nodes tendermint validators buy}}}
command. After that, the buyer will be a new validator of the network.

\begin{commentbox}
It is important to note that a validator node is assumed to be reachable from the outside world
and up, also overnight. Most home desktop computers, connected through a modem, have no public IP
that can be used to reach them from the outside world (the public IP of the modem reaches the modem
itself, not the computer). Therefore, if you try to use a home machine to become a validator, your
machine will be immediately slashed for being uneachable and it will be removed from the set of validators,
immediately after becoming one. You need a machine with a public IP for being a validator, such as
for instance a machine on a rental service.
\end{commentbox}
\chapter{Hotmoka nodes}\label{ch:hotmoka_nodes}

A Hotmoka node is a device that implements an interface for running Java code
remotely. It can be any kind of device, such as a device of an IoT network,
but also a node of a blockchain. We have already used instances of Hotmoka nodes,
namely, instances of \texttt{RemoteNode}. But there are other examples of nodes, that we
will describe in this chapter.

The interface \texttt{io.hotmoka.node.api.Node}\index{Node@{\texttt{Node}}}
is shown in the topmost part of Fig.~\ref{fig:node_hierarchy}.
That interface can be split in five parts:
%
\begin{enumerate}
\item A \emph{get} part, that includes methods for querying the
	state of the node and for accessing the objects contained in its store.
\item An \emph{add} part, that expands the store of the node with the result of a transaction.
\item A \emph{run} part, that runs transactions that execute
	\texttt{@View}\index{View@{\texttt{View}}} methods and hence do not
   	expand the store of the node.
\item A \emph{post} part, that expands the store of the node with the result of a transaction,
	without waiting for its result; instead, a future is returned.
\item A \emph{contextual} part, that allows users to subscribe listeners of events generated during
	the execution of the transactions, or to subscribe listeners called when the node gets closed, or
	to close the node itself.
\end{enumerate}

\begin{figure}[th!]
  \begin{center}
	\begin{tikzpicture}[scale=1]
	\scriptsize
	\externalclasscolor
	\begin{interface}[text width=2.2cm]{Autocloseable}{-4,0}
		\operation{close()}
	\end{interface}
	\begin{interface}[text width=6.6cm]{OnCloseHandlersContainer}{3,0}
		\operation{addOnCloseHandler(OnCloseHandler handler)}
		\operation{removeOnCloseHandler(OnCloseHandler handler)}
	\end{interface}
	\interfacecolor
	\begin{interface}[text width=16cm]{Node}{0,-2.2}
		\inherit{Autocloseable}
		\inherit{OnCloseHandlersContainer}
		\operation{getInfo():NodeInfo}
		\operation{getConfig():ConsensusConfig<?,?>}
		\operation{getTakamakaCode():TransactionReference}
		\operation{getManifest():StorageReference}
		\operation{getClassTag(object:StorageReference):ClassTag}
		\operation{getState(StorageReference object):Stream<Update>}
		\operation{getIndex(StorageReference object):Stream<TransactionReference>}
		\operation{getRequest(reference:TransactionReference):TransactionRequest<?>}
		\operation{getResponse(reference:TransactionReference):TransactionResponse}
		\operation{getPolledResponse(reference:TransactionReference):TransactionResponse}
		\operation{addJarStoreInitialTransaction(request:JarStoreInitialTransactionRequest):TransactionReference}
		\operation{addGameteCreationTransaction(request:GameteCreationTransactionRequest):StorageReference}
		\operation{addInitializationTransaction(request:InitializationTransactionRequest)}
		\operation{addJarStoreTransaction(request:JarStoreTransactionRequest):TransactionReference}
		\operation{addConstructorCallTransaction(request:ConstructorCallTransactionRequest):StorageReference}
		\operation{addInstanceMethodCallTransaction(request:InstanceMethodCallTransactionRequest):Optional<StorageValue>}
		\operation{addStaticMethodCallTransaction(request:StaticMethodCallTransactionRequest):Optional<StorageValue>}
		\operation{runInstanceMethodCallTransaction(request:InstanceMethodCallTransactionRequest):Optional<StorageValue>}
		\operation{runStaticMethodCallTransaction(request:StaticMethodCallTransactionRequest):Optional<StorageValue>}
		\operation{postJarStoreTransaction(request:JarStoreTransactionRequest):JarFuture}
		\operation{postConstructorCallTransaction(request:ConstructorCallTransactionRequest):ConstructorFuture}
		\operation{postInstanceMethodCallTransaction(request:InstanceMethodCallTransactionRequest):MethodFuture}
		\operation{postStaticMethodCallTransaction(request:StaticMethodCallTransactionRequest):MethodFuture}
		\operation{subscribeToEvents(creator:StorageReference, handler:BiConsumer<StorageReference, StorageReference>):Subscription}
	\end{interface}
	\packagecolor
	\begin{package}{Local implementations}
		\begin{interface}[text width=2.7cm]{MokamintNode}{-6.5,-12.5}
			\interfacecolor
			\inherit{Node}
		\end{interface}
		\begin{interface}[text width=2.7cm]{TendermintNode}{-3.4,-12.5}
			\interfacecolor
			\inherit{Node}
		\end{interface}
		\begin{interface}[text width=2.3cm]{DiskNode}{-0.5,-12.5}
			\interfacecolor
			\inherit{Node}
		\end{interface}
	\end{package}
	\packagecolor
	\begin{package}{Adaptors}
		\begin{interface}[text width=2.5cm]{RemoteNode}{7,-12.5}
			\interfacecolor
			\inherit{Node}
		\end{interface}
	\end{package}
	\packagecolor
	\begin{package}{Decorators}
		\begin{interface}[text width=3.5cm]{InitializedNode}{-6,-15}
			\interfacecolor
			\inherit{Node}
			\operation{gamete():StorageReference}
		\end{interface}
		\begin{interface}[text width=5cm]{AccountsNode}{-0.6,-15}
			\interfacecolor
			\inherit{Node}
			\operation{accounts():Stream<StorageReference>}
			\operation{privateKeys():Stream<PrivateKey>}
			\operation{account(i:int):StorageReference}
			\operation{privateKey(i:int):PrivateKey}
		\end{interface}
		\begin{interface}[text width=5cm]{JarsNode}{5.5,-15}
			\interfacecolor
			\inherit{Node}
			\operation{jars():Stream<TransactionReference>}
			\operation{jar(i:int):TransactionReference}
		\end{interface}
	\end{package}
	\end{tikzpicture}
  \end{center}
  \caption{The hierarchy of Hotmoka nodes.}
  \label{fig:node_hierarchy}
\end{figure}

If a node belongs to a blockchain, then all nodes of the blockchain have the same vision
of the state, so that it is equivalent to call a method on a node or on any other node of the
network. The only methods that are out of consensus, since they deal with information specific
to each node, are \texttt{getInfo()}\index{getInfo()@{\texttt{getInfo()}}}, that returns
specific information about the node, and the four contextual methods
\texttt{subscribeToEvents()}\index{subscribeToEvents()@{\texttt{subscribeToEvents()}}},
\texttt{addOnCloseHandler()}\index{addOnCloseHandler()@{\texttt{addOnCloseHandler()}}},
\texttt{removeOnCloseHandler()}\index{removeOnCloseHandler()@{\texttt{removeOnCloseHandler()}}}
and \texttt{close()}\index{close()@{\texttt{close()}}}.

Looking at Fig.~\ref{fig:node_hierarchy}, it is possible to see that
the \texttt{Node} interface has many implementations, that we describe below.

\begin{description}
\item[Local implementations.]\index{node!local}
These are actual nodes that run on the machine
where they have been started. For instance, they can be a node
of a larger blockchain network. Among them,
\texttt{MokamintNode}\index{MokamintNode@{\texttt{MokamintNode}}}
implements a node of a Mokamint\index{Mokamint} blockchain (Sec.~\ref{sec:hotmoka_mokamint});
\texttt{TendermintNode}\index{TendermintNode@{\texttt{TendermintNode}}}
implements a node of a Tendermint\index{Tendermint} blockchain (Sec.~\ref{sec:hotmoka_tendermint})
and will be presented in Sec.~\ref{sec:tendermint_nodes};
\texttt{DiskNode}\index{DiskNode@{\texttt{DiskNode}}}
implements a single-node blockchain in disk memory: this
is useful for debugging, testing and learning, since it allows
one to inspect the content of blocks, transactions and store;
it will be presented in Sec.~\ref{sec:disk_nodes}.
Local nodes can be instantiated through the static
factory methods of their supplier classes
\texttt{MokamintNodes}\index{MokamintNodes@{\texttt{MokamintNodes}}},
\texttt{TendermintNodes}\index{TendermintNodes@{\texttt{TendermintNodes}}} and
\texttt{DiskNodes}\index{DiskNodes@{\texttt{DiskNodes}}}.
Those methods requires to specify
parameters that are specific to the given node
of the network that is being started
and can be different from node to node
(\texttt{MokamintNodeConfig}\index{MokamintNodeConfig@{\texttt{MokamintNodeConfig}}} and similar).
Some implementations have to ability to \emph{resume}.
This means that they recover the state at the end of a previous execution, reconstruct the
consensus parameters from that state and resume the execution from there, downloading
and verifying blocks already processed by the network.
%
\item[Decorators.]\index{node!decorator}
The \texttt{Node}\index{Node@{\texttt{Node}}} interface is implemented by some decorators as well.
Typically, these decorators run some transactions on the decorated node,
to simplify some tasks, such as the initialization of the node, the installation of jars into the node
or the creation of accounts in the node. These decorators are views of the decorated node, in the sense
that any method of the \texttt{Node} interface, invoked on the decorator, is forwarded
to the decorated node, with the exception of the contextual methods that are executed locally
on the specific node where they are invoked.
We will discuss them in Sec.~\ref{sec:node_decorators}.
%
\item[Adaptors.]\index{node!adaptor}
Very often, one wants to \emph{publish} a node online,
so that he (and other programmers who need its service) can use it concurrently.
This should be possible for all implementations of the
\texttt{Node}\index{Node@{\texttt{Node}}} interface,
such as \texttt{DiskNode}\index{DiskNode@{\texttt{DiskNode}}},
\texttt{MokamintNode}\index{MokamintNode@{\texttt{MokamintNode}}},
\texttt{TendermintNode}\index{TendermintNode@{\texttt{TendermintNode}}}
and all present and future implementations.
In other words, one would like to publish \emph{any}
Hotmoka node as a service, accessible through the internet. This will be the subject
of Sec.~\ref{sec:node_services}.
Conversely, once a Hotmoka node has been published at some URI, say
\url{ws://my.company.com}, it will be accessible through a network connection. This complexity
might make it complex, for a programmer, to use the published node.
In that case, we can create an instance of the node that operates as
a proxy to the network service, helping programmers integrate
their software to the service in a seamless way. This \emph{remote} node still implements
the \texttt{Node} interface, but simply forwards all its calls to the remote service
(with the exception of the contextual methods, that are executed locally on
the remote node itself). By programming against
the same \texttt{Node} interface, it becomes easy for a programmer
to swap a local node with a remote node, or vice versa. This mechanism is described
in Sec.~\ref{sec:remote_nodes},
where the adaptor interface \texttt{RemoteNode} in Fig.~\ref{fig:node_hierarchy} is presented.
\end{description}

\section{Tendermint nodes}\label{sec:tendermint_nodes}

This section shows how you can start your own Hotmoka Tendermint node,
consisting of a single validator node, hence part of its own blockchain.
The process is not difficult but is a bit tedious,
because it requires one to install Tendermint and to create
its configuration files. Sec.~\ref{sec:hotmoka_tendermint}
provides a simpler alternative for reaching the same goal, by using docker.

\begin{commentbox}
We strongly suggest you to use docker to install Hotmoka nodes, instead of the instructions
in this section, hence please
follow the instructions in Sec.~\ref{sec:hotmoka_tendermint}.
The current section only exists in order to understand what happens inside the docker container.
If you are not a developer, or if you are not interested in the topic, you can safely
skip this section.
\end{commentbox}

In order to use a Tendermint Hotmoka node, the Tendermint executable must be
installed in our machine, or our experiments will fail. The Hotmoka node
works with Tendermint version \tendermintVersion{}, that can be downloaded in executable
form from \url{https://github.com/tendermint/tendermint/releases/tag/v\tendermintVersion}.
Be sure that you download the executable for the architecture of your computer
and install it at a place that is
part of the command-line path of your computer. You can then verify that Tendermint is
correctly installed:
%
\inputCommand{tendermint_version}
\inputOutput{tendermint_version}

Before starting a local node of a Hotmoka blockchain based on Tendermint, you
need to create the Tendermint configuration file. For instance, in order
to run a single validator node with no non-validator nodes, you can create
its configuration files as follows:
%
\inputCommand{tendermint_testnet}
\inputOutput{tendermint_testnet}
%
This has created a directory \texttt{mytestnet/node0}
for a single Tendermint node, that includes the configuration
of the node and its private and public validator keys.

Once this is done, you can create a key pair for the \emph{gamete}\index{gamete}
of the node that you are going to start. This is an account that holds all initial
crypto coins, if any. You perform this with moka:
%
\index{moka!keys create@{\texttt{keys create}}}
\inputCommand{moka_keys_create_gamete}
\inputOutput{moka_keys_create_gamete}

You can start now a Hotmoka node based on Tendermint,
that uses the Tendermint configuration
directory that you have just created, and with a gamete controlled
by the \texttt{gamete.pem} key pair,
by using the \texttt{moka nodes tendermint init} command. You need
to specify the jar of the runtime of Takamaka, that will
be stored inside the node as \texttt{takamakaCode}: we use the
local Maven's cache for that but you can alternatively download the
\texttt{io.takamaka-code-\takamakaVersion{}.jar} file from Maven and refer to
it in the following command line:
%
\index{moka!nodes tendermint init@{\texttt{nodes tendermint init}}}
\inputCommand{moka_nodes_tendermint_init}
\inputOutput{moka_nodes_tendermint_init}

This command has done a lot! It has created an instance
of \texttt{TendermintNode}; it has stored the
\texttt{io-takamaka-code-\takamakaVersion{}.jar} file
inside it; it has created
a Java object, called manifest, that contains other objects, including
an externally-owned account gamete, whose public key is
that provided with \texttt{-{}-public-key-of-gamete};
it has initialized the balance of the gamete to
the a default initial supply. Finally, this command
has published an internet service at the URI \url{ws://localhost:8001},
reachable through websocket connections, that exports the API of the node.

\begin{commentbox}
By default, \texttt{moka nodes tendermint init} publishes the service at port 8001.
This can be changed with its \texttt{-{}-port} option. Moreover, it uses a default
initial supply for the gamete that can be changed with the \texttt{-{}-initial-supply} option.
Finally, it uses a Hotmoka chain identifier identical to that of the underlying Tendermint
network, specified inside the Tendermint configuration files created by \texttt{tendermint testnet}.
If you want to override it, you can either
edit such configuration files or use the \texttt{-{}-chain-id} option (in the latter case,
the chain identifiers of Hotmoka and Tendermint may be different, which is perfectly fine).
\end{commentbox}

In order to use the gamete, you should bind its key to its actual storage
reference in the node, on your local machine. Open another shell,
move inside the directory holding the keys of the gamete and run:
%
\index{moka!keys bind@{\texttt{keys bind}}}\index{key!binding}
\inputCommand{moka_keys_bind_gamete}
\inputOutput{moka_keys_bind_gamete}
%
This operation has created a pem file whose name is that of the storage reference of the gamete.
With this file, it is possible to run transactions on behalf of the gamete.
%
\begin{commentbox}
You do not need the \texttt{-{}-uri} option above, since \url{ws://localhost:8001} is the default
URI for moka.
\end{commentbox}

Your computer exports a Hotmoka node now, running on Tendermint. You can verify this with
\texttt{moka nodes manifest show}.
Moreover, if your computer is reachable at some address \texttt{my.machine}
and if its 8001 port is open to the outside world,
then anybody can contact
your node at \url{ws://my.machine:8001}, query your node and run transactions on it.
However, what has been created is a Tendermint node where all initial coins are inside
the gamete. By using the gamete, \emph{you} can fill the node with objects
and accounts now, and in general run all transactions you want.
However, other users, who do not know the keys of the gamete,
will not be able to run any non-\texttt{@View} transaction on your node.
If you want to open a faucet\index{faucet}, so that other users can gain droplets of coins
(see examples in Sec.~\ref{sec:creation_account}),
you must add the \texttt{-{}-open-unsigned-faucet} option to the \texttt{moka nodes tendermint init}
command above. If you do that, you can then go into another shell (since the previous one is busy with the
execution of the node), in a directory holding the key pair file of the gamete, and type:
%
\index{moka!nodes faucet@{\texttt{nodes faucet}}}
\inputCommand{moka_nodes_faucet}
\inputOutput{moka_nodes_faucet}
%
This set the maximal amount of coins that
the faucet is willing to give away at each request (its \emph{flow}). You can re-run the
\texttt{moka nodes faucet}
command many times, in order to change the flow of the faucet, or close it completely.
Needless to say, only the owner of the keys of the gamete can run the \texttt{moka nodes faucet} command,
which is why the key pair file of the gamete must be in the directory where you run it.

After opening a faucet with a sufficient flow, anybody can
re-run, for instance, the examples of Ch.~\ref{ch:getting_started_with_hotmoka} by replacing
\texttt{\serverMokamint{}} with \url{ws://my.machine:8001}: your computer will serve
the requests and run the transactions.

If you turn off your Hotmoka node based on Tendermint, its state remains saved inside the
\texttt{chain} directory: the \texttt{chain/tendermint} subdirectory is where Tendermint stores the blocks
of the chain; while \texttt{chain/hotmoka} contains the Xodus database,
consisting of the storage objects created in blockchain.
Try for instance to stop the Tendermint node that we initialized before
(press enter in the window where it was running).
You can subsequently resume that node from its latest state, by typing:
%
\index{moka!nodes tendermint resume@{\texttt{nodes tendermint resume}}}
\inputCommand{moka_nodes_tendermint_resume}
\inputOutput{moka_nodes_tendermint_resume}

There is a log file that can be useful to check the state of our Hotmoka-Tendermint node.
Namely, \texttt{tendermint.log} contains the log of Tendermint itself. It can be interesting
to inspect which blocks are committed and when:
%
\begin{shellbox}\begin{ttlst}
I[2025-06-11|10:13:24.143] Version info, module=main
  tendermint_version=@tendermint_version block=11 p2p=8
I[2025-06-11|10:13:24.169] Started node module=main
  nodeInfo="{ProtocolVersion:{P2P:8 Block:11 App:0}
I[2025-06-11|10:13:25.234] executed block module=state
  height=630 num_valid_txs=2 num_invalid_txs=0
I[2025-06-11|10:13:25.408] committed state module=state
  height=630 num_txs=0
  app_hash=A30F89457141AB7E94F71456871396FD9D30CA8E9F66998C6E3E3079D40849F
\end{ttlst}\end{shellbox}
%
In this log, the block height increases and the application hash changes,
reflecting the fact that the state has been modified.

\subsection{Shared entities}\label{subsec:shared_entities}

This section describes how the set of validators
is implemented in Hotmoka. Namely, the validation power\index{validation power} of the network is expressed as a
total quantity shared among all validator nodes. For instance, when we have shown the manifest of the
nodes (\texttt{moka nodes manifest show}), we have seen information about the only validator in the subsequent form:
%
\begin{shellbox}\begin{ttlst}
validator #0: b437832a688dbb89b145d380b7cb3eb841d7cb09fb600d27be43cd670e8b43f9#0
  id: 030203B36BA4EDF0182D7D40D7EA7FE34A9415B4
  balance: 1568
  staked: 4704
  power: 1000000
\end{ttlst}\end{shellbox}
%
This means that validator \#0 has a \emph{power} of 1000000. If it were the only validator of the network,
also the total power of the validators of the network would be 1000000.
A validator can decide to sell part of its power or all its power to
another validator, resulting in a network with a single (different) validator or with more validators.
For instance, it might sell 200000 units of power to another validator \#1, resulting
in a network with two validators: validator \#0 with 800000 units of power and
validator \#1 with 200000 units of power.

What said above means that the set of validators are a sort of \emph{entity} that shares validation
power among the single validators. Validation power can be sold and bought. The number of
validators and their power is consequently dynamic. In some sense, this mechanism
resembles the market of shares of a corporation.

\begin{figure}[th]
  \begin{center}
    \myincludegraphics{0.8\textwidth}{entities}
  \end{center}
  \caption{The hierarchy of entities and validators classes.}
  \label{fig:entities_hierarchy}
\end{figure}

Hotmoka has an interface that represents entities whose shares can be dynamically sold and bought
among \emph{shareholders}. Fig.~\ref{fig:entities_hierarchy} shows this
\texttt{SharedEntity}\index{SharedEntity@{\texttt{SharedEntity}}}\index{shared entity}
interface. As you can see in the figure, the notion of validators\index{validator}
is just a special case of shared entity (see also~\cite{SpotoMGB23}).
It is possible to use shared entities to represent other concepts, such as
a distributed autonomous organization, or a voting community.
Here, however, we focus on their use to represent the set
of the validators of a proof of stake blockchain.

In general, two concepts are specific to each implementation of shared entities:
who are the potential shareholders and how offers for selling shares work.
Therefore, two generic types specify the interface \texttt{SharedEntity<S,O>}:
\texttt{S} is the type of the shareholders and \texttt{O} is the type of the sale offers of shares.
The \texttt{SharedEntityView}\index{SharedEntityView@{\texttt{SharedEntityView}}}
interface at the top of the hierarchy in Fig.~\ref{fig:entities_hierarchy} defines
the read-only operations on a shared entity. This view is static, in the sense
that it does not specify the operations for transfers of shares. Therefore, its
only type parameter is \texttt{S}: any contract can play the role of the type for the
shareholders of the entity. Method \texttt{getShares()} yields the current
shares of the entity (who owns how much).
Method \texttt{isShareholder()} checks if an object is a shareholder. Method
\texttt{sharesOf()} yields the number of shares that a shareholder owns. As typical in Takamaka,
the \texttt{snapshot()} method allows one to create a frozen read-only copy of an entity
(in constant time), useful when an entity must be queried from a client without
the risk of race conditions if another client is modifying the same entity concurrently.

The \texttt{SharedEntity} subinterface adds methods for transfer of shares.
It includes an inner class \texttt{Offer}\index{Offer@{\texttt{Offer}}}
that models sale offers: it specifies
who is the seller of the shares, how many shares are being sold, the requested
price and the expiration of the offer. Method \texttt{isOngoing()} checks if an offer has
not expired yet. Implementations can subclass \texttt{Offer} if they need more specific
offers. Offers can be placed on sale by calling the \texttt{place()} method with a sale
offer. This method is annotated as \texttt{@FromContract} since the caller must be
identified as the owner of the shares
(or otherwise anybody could sell the shares of anybody else) and as
\texttt{@Payable} so that implementations can require to pay a ticket to place shares on
sale. The sale offer is passed as a parameter to \texttt{place()}, hence it must have been
created before calling that method. The set of all sale offers is available through
\texttt{getOffers()}. Method \texttt{sharesOnSaleOf()} yields the cumulative number of shares on
sale for a given shareholder. Who wants to buy shares calls method \texttt{accept()} with
the accepted offer and with itself as \texttt{buyer}
and becomes a new shareholder or increases its cumulative number of shares
(if it was already a shareholder). Also this method is \texttt{@Payable}, since its caller
must pay \texttt{ticket >= offer.cost} coins to the seller. This means that shareholders
must be able to receive payments and that is why \texttt{S extends PayableContract}.
The \texttt{SimpleSharedEntity}\index{SimpleSharedEntity@{\texttt{SimpleSharedEntity}}}
class implements the shared entity algorithms, that subclasses can redefine if they want.

Hotmoka models validator nodes as
objects of class \texttt{Validator}\index{Validator@{\texttt{Validator}}},
that are externally owned accounts with an extra identifier
(Fig.~\ref{fig:entities_hierarchy}).
In the specific case of a Hotmoka blockchain built over Tendermint, validators
are instances of the subclass
\texttt{TendermintED25519Validator}\index{TendermintED25519Validator@{\texttt{TendermintED25519Validator}}},
whose identifier is derived from their
ed25519 public key. This identifier is public information, reported
in the blocks or easily eavesdropped.
The \texttt{Validators}\index{Validators@{\texttt{Validators}}}
interface in Fig.~\ref{fig:entities_hierarchy}
extends the \texttt{SharedEntity} interface, fixes the shareholders
to be instances of \texttt{Validator} and adds a method
\texttt{getStake()} that yields the amount of coins
at stake for each given validator (if the validator misbehaves, its stake will
be slashed).

The \texttt{AbstractValidators}\index{AbstractValidators@{\texttt{AbstractValidators}}}
class implements the set of validators and the distribution
of the reward and is a subclass of \texttt{SimpleSharedEntity} (Fig.~\ref{fig:entities_hierarchy}).
Shares are voting power in this case. It has a subclass
for each kind of Hotmoka node, so that the distribution of the reward at each
block creation can be implemented differently in each node type.
Namely, there exist subclasses
\texttt{TendermintValidators}\index{TendermintValidators@{\texttt{TendermintValidators}}},
\texttt{MokamintValidators}\index{MokamintValidators@{\texttt{MokamintValidators}}} and
\texttt{DiskValidators}\index{DiskValidators@{\texttt{DiskValidators}}}.
The first subclass restricts the type of the validators to be
\texttt{TendermintED25519Validator}. At each
block committed, Hotmoka calls the reward method of such subclasses in order
to reward the validators (if any) that behaved correctly and slash those that
misbehaved, possibly removing them from the set of validators. Moreover, such methods
are responsible for minting new coins at each block creation.
In the case of Tendermint nodes, at block creation time,
Hotmoka calls method \texttt{getShares()} and informs
the underlying Tendermint engine about the identifiers of the validator nodes
for the next blocks. Tendermint expects such validators to mine and vote the
subsequent blocks, until a change in the set of validators occurs.

\section{Disk nodes}\label{sec:disk_nodes}

The Hotmoka nodes of the previous sections form a real blockchain.
They are perfect for deploying a blockchain where we can program smart contracts in
Takamaka. Nevertheless, they are slow for debugging: transactions are committed every few seconds,
by default. Hence, if we want to see the result of a transaction,
we have to wait for some seconds at least.
Moreover, Tendermint does not allow one to see the effects of each single transaction,
in a simple way. For testing, debugging and didactical purposes, it would be simpler to have a light node
that behaves like a blockchain, allows access to blocks and transactions as text files,
but is not a blockchain. This is the goal of the \texttt{DiskNode}s.\index{DiskNode@{\texttt{DiskNode}}}
They are not part of an actual blockchain since they do not propagate transactions
in a peer-to-peer network, on which consensus is imposed. But they are very
handy because they allow one to inspect, very easily, the requests sent to
the node and the corresponding responses.

You can start a disk Hotmoka node, with an open faucet, exactly as you did,
in the previous sections for a Mokamint or Tendermint node, but using the \texttt{moka nodes disk init}
command instead of \texttt{moka nodes [mokamint|tendermint] init}. You do not need any Mokamint or Tendermint configuration
this time, but still need a key to control the gamete of the node, that you can create
exactly as for the previous Hotmoka nodes.
You then specify the Base58-encoded public key when starting the node:
%
\index{moka!nodes disk init@{\texttt{nodes disk init}}}
\inputCommand{moka_nodes_disk_init}
\inputOutput{moka_nodes_disk_init}
%
Then, in another shell, you can bind the gamete and open the flow of the faucet, as we did
for the other kinds of Hotmoka nodes.

You should have noticed any apparent difference with the previous kinds of Hotmoka nodes,
but for the fact that this node is faster,
its default chain identifier is the empty string and it has no validators. Blocks and transactions are
inside the \texttt{chain} directory, that this time contains a nice textual representation of requests and
responses:
%
\inputCommand{tree_chain}
\inputOutput{tree_chain}

\begin{commentbox}
The exact ids and the number of these transactions will be different in your computer.
\end{commentbox}

There are blocks \texttt{b0},\ldots,\texttt{b7}, each containing a variable number of transactions.
Each transaction is reported with its id and the pair request/response that the node has computed
for it. They are text files, that you can open to understand what is happening inside the node.

The transactions shown above are those that have initialized the node and
opened the faucet. The last transaction inside the last block is a \emph{reward}
transaction, that distributes the earnings of the block to the (zero, for disk nodes) validators
and increases block height and number of transactions in the manifest.

Spend some time looking at the \texttt{request.txt} and \texttt{response.txt} files.
For instance, the transaction inside \texttt{b2} should be the one that created the gamete
account. Print its \texttt{request.txt} file:
%
\index{transaction!request}
\inputCommand{cat_gamete_creation_request}
\inputOutput{cat_gamete_creation_request}
%
You can see that this is a request to create a gamete: it specifies the initial amount of crypto coins
held in the gamete and the public key of the gamete, which is what we passed when we initialized the node
(in base64 format, since it is more compact, in general). Print its response now:
%
\index{transaction!response}
\inputCommand{cat_gamete_creation_response}
\inputOutput{cat_gamete_creation_response}
%
You can see that this response reports the storage reference of the gamete that has been created.
Moreover, responses typically report a set of \emph{updates}, as in this case.
Updates are the side-effects on the state of the node,
induced by the transaction. Each update is a triple, that specifies a change in the value
of a field of a storage object. In this case, the updates describe the initial
state of the gamete object; for instance, an update states that
the balance of the gamete has been set to the initial supply for the node;
another states that the \texttt{maxFaucet} field of the gamete has been set to 0: this might be modified
later through a transaction triggered by the \texttt{moka nodes faucet} command.

\section{Logs}\label{sec:logs}

The moka tool generates a \texttt{hotmoka.log.*}\index{hotmoka.log@{\texttt{hotmoka.log}}}\index{logs}
log file. Therefore, that file is generated also
for the \texttt{moka nodes} commants that initialize or resume a Hotmoka node.
In that case, the logs will report which transactions have been
processed, together with potential errors.
%
\begin{commentbox}
The \texttt{hotmoka.log} file is normally rotated across successive or very long executions of moka.
Therefore, look for files such as \texttt{hotmoka.log.0} or \texttt{hotmoka.log.1} to find the logs of the
specific execution of moka that you are interested in.
\end{commentbox}

The content of the logs, in the case of the execution of a Hotmoka node, might look like:
%
\inputCommand{cat_hotmoka_log}
\inputOutput{cat_hotmoka_log}

If you want to follow in real time what is happening inside your node,
you can run for instance \texttt{tail -f hotmoka.log.0}: this
will hang and print the new log entries as they are generated.
Assuming that you have a local node running in your machine, try for instance in another shell
to run \texttt{moka nodes manifest show}: you will see in the log all new entries related
to the execution of the methods to access
the information on the node printed by the last command.
%
\begin{commentbox}
Hotmoka nodes started with Docker disable the generation of the log files and dump
logs to the standard output, where they can be accessed with the \texttt{docker logs} command.
Therefore, they do not generate any \texttt{hotmoka.log} file. See next chapter for information.
\end{commentbox}

\section{Node decorators}\label{sec:node_decorators}

\begin{center}
(See the \texttt{io-hotmoka-tutorial-examples-runs} project in \texttt{\hotmokaRepo{}})
\end{center}

There are some frequent actions that can be performed in code on a Hotmoka node.
Typically, these actions consist in a sequence of transactions. A few examples are:
%
\begin{enumerate}
\item The creation of an externally owned account\index{account!creation}. This requires the creation
	of its private and public keys and the instantiation of an
	\texttt{io.takamaka.code.lang.ExternallyOwnedAccount}. It is not a difficult
	procedure, but it is definitely tedious and occurs frequently.
\item The installation of a jar\index{jar!installation} in a node. This requires a transaction for installing
	code in the node. It requires also to parse the jar into bytes and identify the
	number of gas units for the transaction, depending on the size of the jar.
\item The initialization of a node\index{node!initialization}. Namely, local nodes start empty, that is,
	their store does not contain anything at the beginning, not even their manifest
	object. This initialization is rather technical and detail might change in future
	versions of Hotmoka. Performing this initialization by hand leads to fragile
	and error-prone code.
\end{enumerate}

In all these examples, Hotmoka provides decorators\index{node!decorator},
that is, implementations of the
\texttt{Node} interface built from an existing \texttt{Node} object. A decorator is just an alias
of the decorated node, but adds some functionality or performs some action on it.
Fig.~\ref{fig:node_hierarchy} shows that there are decorators for each of the three
situations enumerated above.

In order to understand the use of node decorators and appreciate their existence,
let us write a Java class that creates a \texttt{DiskNode}, initially empty;
then it initializes that node; subsequently it installs our \texttt{io-hotmoka-tutorial-examples-family-\hotmokaVersion{}.jar}
file from Sec.~\ref{sec:calling_method} in the node and finally creates two accounts in the node.
We stress the fact that these actions
can be performed in code by using calls to the node interface (Fig.~\ref{fig:node_hierarchy}) or
through the moka tool. Here, however, we want to perform them
in code, simplified with the use of node decorators.

Create the following \texttt{Decorators.java} class inside the
\texttt{io.hotmoka.tutorial.examples.runs} package of the
\texttt{io-hotmoka-tutorial-examples-runs} project:
%
\index{DiskNodes@{\texttt{DiskNodes}}}
\index{DiskInitializedNodes@{\texttt{DiskInitializedNodes}}}
\index{JarsNodes@{\texttt{JarsNodes}}}
\index{AccountsNode@{\texttt{AccountsNodes}}}
\begin{codebox}\begin{javalst}
package io.hotmoka.tutorial.examples.runs;

import static io.hotmoka.constants.Constants.HOTMOKA_VERSION;
import static io.takamaka.code.constants.Constants.TAKAMAKA_VERSION;

import java.math.BigInteger;
import java.nio.file.Paths;
import java.security.KeyPair;

import io.hotmoka.crypto.Entropies;
import io.hotmoka.crypto.SignatureAlgorithms;
import io.hotmoka.helpers.AccountsNodes;
import io.hotmoka.helpers.JarsNodes;
import io.hotmoka.node.ConsensusConfigBuilders;
import io.hotmoka.node.disk.DiskInitializedNodes;
import io.hotmoka.node.disk.DiskNodeConfigBuilders;
import io.hotmoka.node.disk.DiskNodes;

public class Decorators {
 
  public static void main(String[] args) throws Exception {
    var config = DiskNodeConfigBuilders.defaults().build();

    // the path of the runtime Takamaka jar, inside Maven's cache
    var takamakaCodePath = Paths.get
      (System.getProperty("user.home")
      + "/.m2/repository/io/hotmoka/io-takamaka-code/" + TAKAMAKA_VERSION
      + "/io-takamaka-code-" + TAKAMAKA_VERSION + ".jar");

    // the path of the user jar to install
    var familyPath = Paths.get(System.getProperty("user.home")
      + "/.m2/repository/io/hotmoka/io-hotmoka-tutorial-examples-family/"
      + HOTMOKA_VERSION
      + "/io-hotmoka-tutorial-examples-family-" + HOTMOKA_VERSION + ".jar");

    // create a key pair for the gamete
    var signature = SignatureAlgorithms.ed25519();
    var entropy = Entropies.random();
    KeyPair keys = entropy.keys("mypassword", signature);
    var consensus = ConsensusConfigBuilders.defaults()
   	  .setInitialSupply(BigInteger.valueOf(1_000_000_000))
   	  .setPublicKeyOfGamete(keys.getPublic()).build();

	 try (var node = DiskNodes.init(config)) {
      // first decorator: store the io-takamaka-code jar
      // and create manifest and gamete
      var initialized = DiskInitializedNodes.of(node, consensus, takamakaCodePath);

      // second decorator: store the family jar: the gamete will pay for that
      var nodeWithJars = JarsNodes.of(node, initialized.gamete(), keys.getPrivate(), familyPath);

      // third decorator: create two accounts, the first with 10,000,000 coins
      // and the second with 20,000,000 units of coin; the gamete will pay
      var nodeWithAccounts = AccountsNodes.of
        (node, initialized.gamete(), keys.getPrivate(),
        BigInteger.valueOf(10_000_000), BigInteger.valueOf(20_000_000));

      System.out.println("manifest: " + node.getManifest());
      System.out.println("family jar: " + nodeWithJars.jar(0));
      System.out.println("account #0: " + nodeWithAccounts.account(0) +
        "\n  with private key " + nodeWithAccounts.privateKey(0));
      System.out.println("account #1: " + nodeWithAccounts.account(1) +
        "\n  with private key " + nodeWithAccounts.privateKey(1));
    }
  }
}
\end{javalst}\end{codebox}
%
Run class \texttt{Decorators}:
%
\inputCommand{mvn_exec_decorators}
\inputOutput{mvn_exec_decorators}
%
As you can see, the use of decorators has avoided us the burden of
programming transaction requests, explicitly, and makes our code more robust,
since future versions of Hotmoka will update the implementation of the decorators,
while their interface will remain untouched, shielding our code from modifications.

As we have already said, decorators are
views of the same node, just seen through different lenses
(Java interfaces). Hence, further transactions can be run on
\texttt{node} or \texttt{initialized} or \texttt{nodeWithJars} or \texttt{nodeWithAccounts},
with the same effects. Moreover, it is not necessary to close all such nodes: closing \texttt{node} at
the end of the try-with-resource will actually close all of them, since they are the same node.

There exist classes for initializing other kinds of Hotmoka nodes in code, such as the classes
\texttt{MokamintInitializedNodes}\index{MokamintInitializedNodes@{\texttt{MokamintInitializedNodes}}}
and \texttt{TendermintInitializedNodes}\index{TendermintInitializedNodes@{\texttt{TendermintInitializedNodes}}}.

\section{Node services}\label{sec:node_services}

\begin{center}
(See the \texttt{io-hotmoka-tutorial-examples-runs} project in \texttt{\hotmokaRepo{}})
\end{center}

This section shows how we can publish a Hotmoka node online, by using Java code,
so that it becomes a
network service that can be used, concurrently, by many remote clients.
Namely, we will show how to publish a blockchain node based on Tendermint, but the code
is similar if you want to publish a memory Hotmoka node or any other Hotmoka node.

Remember that we have already published our nodes online, as network services,
by using the command \texttt{moka nodes [disk|mokamint|tendermint] init}.
Here, however, we want to do the same operation in code.

Create a class \texttt{Publisher.java} inside the package
\texttt{io.hotmoka.tutorial.examples.runs}
of the \texttt{io-hotmoka-tutorial-examples-runs} project.
Use the following code for the class:
%
\index{TendermintNodes@{\texttt{TendermintNodes}}}
\index{NodeServices@{\texttt{NodeServices}}}
\begin{codebox}\begin{javalst}
package io.hotmoka.tutorial.examples.runs;

import java.math.BigInteger;
import java.nio.file.Paths;
import java.security.KeyPair;

import io.hotmoka.crypto.Entropies;
import io.hotmoka.crypto.SignatureAlgorithms;
import io.hotmoka.node.service.NodeServices;
import io.hotmoka.node.tendermint.TendermintConsensusConfigBuilders;
import io.hotmoka.node.tendermint.TendermintInitializedNodes;
import io.hotmoka.node.tendermint.TendermintNodeConfigBuilders;
import io.hotmoka.node.tendermint.TendermintNodes;
import io.takamaka.code.constants.Constants;

public class Publisher {
  public static void main(String[] args) throws Exception {
    var config = TendermintNodeConfigBuilders.defaults().build();

    // the path of the runtime Takamaka jar, inside Maven's cache
    var takamakaCodePath = Paths.get
      (System.getProperty("user.home") +
      "/.m2/repository/io/hotmoka/io-takamaka-code/" + Constants.TAKAMAKA_VERSION +
      "/io-takamaka-code-" + Constants.TAKAMAKA_VERSION + ".jar");

    // create a key pair for the gamete
    var signature = SignatureAlgorithms.ed25519();
    var entropy = Entropies.random();
    KeyPair keys = entropy.keys("password", signature);
    var consensus = TendermintConsensusConfigBuilders.defaults()
      .setPublicKeyOfGamete(keys.getPublic())
      .setInitialSupply(BigInteger.valueOf(100_000_000))
      .build();

    try (var original = TendermintNodes.init(config);
      // uncomment the next line if you want to publish an initialized node
      // var initialized = TendermintInitializedNodes.of(original, consensus, takamakaCodePath);
      var service = NodeServices.of(original, 8001)) {

      System.out.println("\nPress ENTER to turn off the server and exit this program");
      System.in.read();
    }
  }
}
\end{javalst}\end{codebox}
%
We have already seen that \texttt{original} is a Hotmoka node based on Tendermint.
The following line makes the feat:
%
\begin{codebox}\begin{javalst}
var service = NodeServices.of(original, 8001);
\end{javalst}\end{codebox}
%
Variable \texttt{service} holds a Hotmoka \emph{node service}, that is, an actual network service that adapts
the \texttt{original} node to a web API that is published at localhost, at port 8001.
The service is an \texttt{AutoCloseable} object: it starts when it is created and gets shut down 
when its \texttt{close()} method is invoked, which occurs, implicitly, at the end of the
scope of the try-with-resources. Hence, this service remains online until the user
presses the ENTER key and terminates the service (and the program).

Run class \texttt{Publisher}:
%
\inputCommand{mvn_exec_publisher}
%
It should work for a few seconds and then start waiting for the ENTER key. Do not press such key yet!
Since \texttt{original} is not initialized yet, it has no manifest and no gamete. Its store is just empty
at the moment. You can verify that by running \texttt{moka nodes manifest show}, whose result should be:
%
\begin{shellbox}\begin{ttlst}
The remote service is misbehaving: are you sure that it is actually published at ws://localhost:8001 and that it is initialized and accessible?
\end{ttlst}\end{shellbox}
%
since it cannot find a manifest in the node.

Therefore, let us initialize the node before publishing it, so that it is already
initialized when published. Press ENTER to terminate the service, then modify
the \texttt{Publisher.java} class by uncommenting the use of the \texttt{InitializedNode} decorator,
whose goal is to create manifest and gamete of the node
and install the basic classes of the Takamaka runtime inside the node.
%
\begin{commentbox}
Note that we have published \texttt{original}:
%
\begin{codebox}\begin{javalst}
var service = NodeServices.of(original, 8001);
\end{javalst}\end{codebox}
%
but we could have published \texttt{initialized} instead:
%
\begin{codebox}\begin{javalst}
var service = NodeServices.of(initialized, 8001);
\end{javalst}\end{codebox}
%
The result would be the same, since both are views of the same node object.
Moreover, note that we have initialized the node inside the try-with-resources,
before publishing the service as the last of the three resources.
This ensures that the node, when published, is already initialized.
In principle, publishing an uninitialized node, as done previously, exposes
to the risk that somebody else might initialize the node, hence taking its control
since he will set the keys of the gamete.
\end{commentbox}

If you re-run class \texttt{Publisher} now and retry the \texttt{moka nodes manifest show} command, you should see
the manifest of the now initialized node on the screen.
%
\begin{commentbox}
A Hotmoka node, once published, can be accessed by many
users, \emph{concurrently}. This is not a problem, since Hotmoka nodes are thread-safe and can
be used in parallel by many users. Of course, this does not mean that there are no
race conditions at the application level. As a simple example, if two users operate
with the same paying externally owned account, their wallets might suffer from race
conditions on the nonce of the account and they might see requests
rejected because of an incorrect nonce. The situation is the same here as in Ethereum,
for instance. In practice, each externally owned account should be controlled
by a single wallet at a time.
\end{commentbox}

\section{Remote nodes}\label{sec:remote_nodes}

A service can be published and its methods can be called through JSON queries.
This is relatively easy for methods such as \texttt{getManifest()} and
\texttt{getConfig()} of the interface \texttt{Node}. However, it
becomes harder if we want to call methods of \texttt{Node} that need parameters, such
as \texttt{getState()} or the many add/post/run methods for scheduling transactions on
the node. Parameters should be passed as JSON payload of the websockets connection, in a format
that is hard to remember, easy to get wrong and possibly changing in the future.
Moreover, the JSON responses must be parsed back.
In principle, this can be done by hand or through software that builds the
requests for the server and interprets its responses.
Nevertheless, it is not the suggested way to proceed.

A typical solution to this problem is to provide a software SDK, that is, a library
that takes care of serializing the requests into JSON and deserializing
the responses from JSON. Roughly speaking, this is the approach taken in Hotmoka.
More precisely, we can forget about the details of the JSON serialization
and deserialization of requests and responses and only program against the \texttt{Node} interface,
by using an adaptor of a published Hotmoka service into a \texttt{Node}. This adaptor is called
a \emph{remote} Hotmoka node\index{node!remote}.

We have used remote nodes from the very beginning of this tutorial.
Namely, if you go back to Sec.~\ref{sec:jar_installation},
you will see that we have built a Hotmoka node from a remote service:
%
\index{RemoteNodes@{\texttt{RemoteNodes}}}
\begin{codebox}\begin{javalst}
try (var node = RemoteNodes.of(new URI(args[0]), 150000)) {
  ...
}
\end{javalst}\end{codebox}
%
The \texttt{RemoteNodes.of(...)} method adapts a remote service into a Hotmoka node,
so that we can call all its methods (Fig.~\ref{fig:node_hierarchy}). The
\texttt{150000} parameter is the timeout, in milliseconds, for connecting to the service
and for the methods called on the remote node.

\section{Sentry nodes}\label{sec:sentry_nodes}

We have seen that a \texttt{Node} can be published as a Hotmoka service:
in a machine \texttt{my.validator.com} we can execute:
%
\begin{codebox}\begin{javalst}
TendermintNodeConfig config = TendermintNodeConfigBuilders.defaults().build();

try (Node original = TendermintNodes.init(config);
  NodeService service = NodeServices.of(original, 8001)) {
  ...
}
\end{javalst}\end{codebox}
%
The service will be published on the internet at \url{ws://my.validator.com:8001}.
Moreover, in another machine \texttt{my.sentry.com},
that Hotmoka service can be adapted into a remote node
that, itself, can be published on that machine:
%
\begin{codebox}\begin{javalst}
try (Node validator = RemoteNodes.of(URI.create("ws://my.validator:8001"), 80000);
  NodeService service = NodeServices.of(validator, 8001)) {
  ...
}
\end{javalst}\end{codebox}
%
The service will be published at \url{ws://my.sentry.com:8001}.

We can continue this process as much as we want, but let us stop at this point.
Programmers can connect to the service published at
\url{ws://my.sentry.com:8001} and send requests to it. That service is just a bridge
that forwards everything to the service at \url{ws://my.validator.com:8001}.
It might not be immediately clear why this intermediate step could be useful
or desirable. The motivation is that we could keep the (precious) validator
machine under a firewall that allows connections with \texttt{my.sentry.com} only.
As a consequence, in case of DOS attacks, the sentry node will receive
the attack and possibly crash, while the validator continues to operate as usual:
it will continue to interact with the other validators and take part in the validation
of blocks. Moreover, since many sentries can be connected to a single validator, the latter
remains accessible through the other sentries, if needed.
This is an effective way to mitigate the problem of DOS attacks to validator nodes.

The idea of using sentry nodes against DOS attacks is not new for proof-of-stake networks,
whose validators are considered as precious resources that must be protected. It is used, for
instance, in Cosmos networks~\cite{CosmosSentry}.
However, note how it is easy, with Hotmoka, to build such a network architecture
by using network services and remote nodes.

\section{Signatures and quantum-resistance}\label{sec:quantum}

Hotmoka is agnostic \wrt{} the algorithm used for signing requests. This means that it is
possible to deploy Hotmoka nodes that sign requests with distinct signature algorithms.
Of course, if nodes must re-execute the same transactions, such as in the case of a
blockchain, then all nodes of the blockchain must use the same algorithm for
the transactions signed by each given account, or otherwise
they will not be able to reach consensus.
Yet, any fixed algorithm can be chosen for each account. In principle, it is even possible to use
an algorithm that does not sign the transactions, if the identity of the callers of the
transactions needn't be verified. However, this might be sensible in private networks only.

The default signature algorithm used by a node is specified at construction time, as a configuration
parameter. For instance, the code\index{setSignatureForRequests()@{\texttt{setSignatureForRequests()}}}
%
\begin{codebox}\begin{javalst}
var config = TendermintNodeConfigBuilders.defaults().build();
var consensus = TendermintConsensusConfigBuilders.defaults()
  .setPublicKeyOfGamete(keys.getPublic())
  .setInitialSupply(SUPPLY)
  ...
  .setSignatureForRequests(SignatureAlgorithms.ed25519()) // this is the default
  .build();

try (var node = TendermintNodes.init(config);
     var initialized = TendermintInitializedNodes.of(node, consensus, takamakaCodePath)) {
  ...
}
\end{javalst}\end{codebox}
%
starts a Tendermint-based blockchain node that uses the ed25519 signature algorithm
as default signature algorithm for the requests.
Requests sent to that node can be signed as follows:
%
\begin{codebox}\begin{javalst}
// recover the algorithm used by the node
SignatureAlgorithm signature = node.getConfig().getSignatureForRequests();

// create a key pair for that algorithm
KeyPair keys = signature.getKeyPair();

// create a signer object with the private key of the key pair
Signer<SignedTransactionRequest<?>> signer = signature.getSigner
  (keys.getPrivate(), SignedTransactionRequest<?>::toByteArrayWithoutSignature);

// create an account having public key keys.getPublic()
var account = ....

// create a transaction request on behalf of the account
ConstructorCallTransactionRequest request
  = TransactionRequests.constructorCall(signer, account, ...);

// send the request to the node
node.addConstructorCallTransaction(request);
\end{javalst}\end{codebox}
%
In the example above, we have explicitly specified
to use ed25519 as default signature algorithm. That is what is chosen
if nothing is specified at configuration-time.
Consequently, there is no need to specify that algorithm in the
configuration object and that is why we never did it in the previous chapters.
But it is possible to configure nodes with other default signature algorithms.
For instance:
%
\begin{codebox}\begin{javalst}
var consensus = TendermintConsensusConfigBuilders.defaults()
  .setPublicKeyOfGamete(keys.getPublic())
  .setInitialSupply(SUPPLY)
  ...
  .setSignatureForRequests(SignatureAlgorithms.sha256dsa()) // this replaces the default
  .build();
\end{javalst}\end{codebox}
%
configures a node that uses sha256dsa as default signature algorithm, while
%
\begin{codebox}\begin{javalst}
var consensus = TendermintConsensusConfigBuilders.defaults()
  .setPublicKeyOfGamete(keys.getPublic())
  .setInitialSupply(SUPPLY)
  ...
  .setSignatureForRequests(SignatureAlgorithms.empty())
  .build();
\end{javalst}\end{codebox}
%
configures a node that uses the empty signature as default signature algorithm; it is an
algorithm that accepts all signatures, in practice disabling signature checking.

It is possible to specify a quantum-resistant signature algorithm\index{quantum resistance} as default,
that is, one that belongs to
a family of algorithms that are expected to be immune from attacks performed through
a quantistic computer. For instance,
%
\begin{codebox}\begin{javalst}
var consensus = TendermintConsensusConfigBuilders.defaults()
  .setPublicKeyOfGamete(keys.getPublic())
  .setInitialSupply(SUPPLY)
  ...
  .setSignatureForRequests(SignatureAlgorithms.qtesla1())
  .build();
\end{javalst}\end{codebox}
%
configures a node that uses the quantum-resistant qtesla-p-I algorithm as default signature algorithm,
while
%
\begin{codebox}\begin{javalst}
var consensus = TendermintConsensusConfigBuilders.defaults()
  .setPublicKeyOfGamete(keys.getPublic())
  .setInitialSupply(SUPPLY)
  ...
  .setSignatureForRequests(SignatureAlgorithms.qtesla3())
  .build();
\end{javalst}\end{codebox}
%
configures a node that uses the quantum-resistant qtesla-p-III
algorithm as default signature algorithm, that is expected to be more resistant than
qtesla-p-I but has larger signatures than qtesla-p-I.

Quantum-resistance is an important aspect of future-generation blockchains.
However, at the time of this writing, a quantum attack is mainly a theoretical
possibility, while the large size of quantum-resistant keys and signatures is
already a reality and a node using a qtesla signature algorithm \emph{as default}
might exhaust the disk space of your computer very quickly. In practice, it is better
to use a quantum-resistant signature algorithm only for a subset of the transactions, whose
quantum-resistance is deemed important. Instead, one should use a lighter algorithm
(such as the default ed25519) for all other transactions. This is possible because
Hotmoka nodes allow one to mix transactions signed with distinct algorithms.
Namely, one can use ed25519 as default algorithm, for all transactions signed
by instances of \texttt{ExternallyOwnedAccount}s,
with the exception of those transactions that are signed by instances of
the interface \texttt{AccountQTESLA1}\index{AccountQTESLA1@{\texttt{AccountQTESLA1}}},
such as the class \texttt{ExternallyOwnedAccountQTESLA1}\index{ExternallyOwnedAccountQTESLA1@{\texttt{ExternallyOwnedAccountQTESLA1}}},
or of the interface \texttt{AccountQTESLA3}\index{AccountQTESLA3@{\texttt{AccountQTESLA3}}},
such as the class \texttt{ExternallyOwnedAccountQTESLA3}\index{ExternallyOwnedAccountQTESLA3@{\texttt{ExternallyOwnedAccountQTESLA3}}},
or of the interface \texttt{AccountSHA256DSA}\index{AccountSHA256DSA@{\texttt{AccountSHA256DSA}}},
such as the class \texttt{ExternallyOwnedAccountSHA256DSA}\index{ExternallyOwnedAccountSHA256DSA@{\texttt{ExternallyOwnedAccountSHA256DSA}}}
(see Fig.~\ref{fig:contract_hierarchy}).
Namely, if the caller of a transaction is an \texttt{AccountQTESLA1}, then the
request of the transaction must be signed with the qtesla-p-I algorithm.
If the caller of a transaction is an \texttt{AccountQTESLA3}, then the
request of the transaction must be signed with the qtesla-p-III algorithm.
If the caller of a transaction is an \texttt{AccountSHA256DSA}, then the
request of the transaction must be signed with the sha256dsa algorithm.
If the caller of a transaction is an \texttt{AccountED25519}, then the
request of the transaction must be signed with the ed25519 algorithm.
In practice, this allows specific transactions to override the default signature
algorithm for the node.

Let us for instance create an account that uses the default signature algorithm for the node.
We charge its creation to the faucet of the node:
%
\index{moka!keys create@{\texttt{keys create}}}
\inputCommand{moka_keys_create_account7}
\inputOutput{moka_keys_create_account7}
%
\index{moka!accounts create@{\texttt{accounts create}}}
\inputCommand{moka_accounts_create_account7}
\inputOutput{moka_accounts_create_account7}
%
You can check the class of the new account with the \texttt{moka objects show} command:
%
\index{ExternallyOwnedAccountED25519@{\texttt{ExternallyOwnedAccountED25519}}}
\index{moka!objects show@{\texttt{objects show}}}
\inputCommand{moka_objects_show_account7}
\inputOutput{moka_objects_show_account7}
%
As you can see, an account has been created, that uses the default ed25519
signature algorithm of the node.
Assume that we want to create an account now, that always uses the sha256dsa signature algorithm instead,
regardless of the default signature algorithm of the node. We can specify that when invoking the
\texttt{moka accounts create} command:
%
\index{moka!keys create@{\texttt{keys create}}}
\inputCommand{moka_keys_create_account8}
\inputOutput{moka_keys_create_account8}
%
\index{moka!accounts create@{\texttt{accounts create}}}
\inputCommand{moka_accounts_create_account8}
\inputOutput{moka_accounts_create_account8}
%
This creation has been more expensive than for the previous account, because the public key of the
sha256dsa algorithm is much longer than that for the ed25519 algorithm.
You can verify this with the \texttt{moka objects show} command:
%
\index{moka!objects show@{\texttt{objects show}}}
\inputCommand{moka_objects_show_account8}
\inputOutput{moka_objects_show_account8}
%
Note that the class of the account is \texttt{ExternallyOwnedAccountSHA256DSA} this time.

Let us create an account that uses the qtesla-p-I signature algorithm now:
%
\index{moka!keys create@{\texttt{keys create}}}
\inputCommand{moka_keys_create_account9}
\inputOutput{moka_keys_create_account9}
%
\index{moka!accounts create@{\texttt{accounts create}}}
\inputCommand{moka_accounts_create_account9}
\inputOutput{moka_accounts_create_account9}
%
The creation of this account has been still more expensive, since this kind of quantum-resistant
keys are very large. Again, you can use the \texttt{moka object show}
command to verify that it has class \texttt{ExternallyOwnedAccountQTESLA1}.

Finally, let us use the previous qtesla-p-I account to create a qtesla-p-III account:
%
\index{moka!keys create@{\texttt{keys create}}}
\inputCommand{moka_keys_create_account10}
\inputOutput{moka_keys_create_account10}
%
\index{moka!accounts create@{\texttt{accounts create}}}
\inputCommand{moka_accounts_create_account10}
\inputOutput{moka_accounts_create_account10}
%
Note, again, the extremely high gas cost of this creation.

Regardless of the kind of account, their use is always the same.
The only difference is to use the right signature algorithm when signing
a transaction, since it must match that of the caller account. This is automatic, if we
use the moka tool. For instance, let us use our qtesla-p-I account to install
the \texttt{io-hotmoka-tutorial-examples-family-\hotmokaVersion{}.jar} file from
Sec.~\ref{sec:calling_method} in the node:
%
\index{moka!jars install@{\texttt{jars install}}}
\inputCommand{moka_jars_install_family_quantum}
\inputOutput{moka_jars_install_family_quantum}
%
The moka tool has understood that the payer is an account that signs with the
qtesla-p-I algorithm and has signed the request accordingly.
\input{9 code_verification}

\bibliographystyle{plain}

\begin{warpprint}   % For print output ...
\cleardoublepage    % ... a common method to place index entry into TOC.
\phantomsection
\addcontentsline{toc}{chapter}{\indexname}
\printindex
\cleardoublepage    % ... a common method to place bibliography entry into TOC.
\phantomsection
\addcontentsline{toc}{chapter}{Bibliography}
\end{warpprint}

\begin{warpHTML}
\ForceHTMLPage      % HTML index will be on its own page.
\ForceHTMLTOC       % HTML index will have its own toc entry.
\printindex
\ForceHTMLPage      % HTML bibliography will be on its own page.
\ForceHTMLTOC       % HTML bibliography will have its own toc entry.
\end{warpHTML}

\bibliography{../../src/main/latex/biblio}

\end{document}
